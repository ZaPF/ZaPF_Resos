\documentclass[DIV=calc]{scrartcl}
\usepackage[utf8]{inputenc}
\usepackage[T1]{fontenc}
\usepackage[ngerman]{babel}
\usepackage{graphicx}
\usepackage[draft, markup=underlined]{changes}
\usepackage{csquotes}
\usepackage{eurosym}

\usepackage{ulem}
%\usepackage[dvipsnames]{xcolor}
\usepackage{paralist}
%\usepackage{fixltx2e}
%\usepackage{ellipsis}
\usepackage[tracking=true]{microtype}

\usepackage{lmodern}              % Ersatz fuer Computer Modern-Schriften
%\usepackage{hfoldsty}

%\usepackage{fourier}             % Schriftart
\usepackage[scaled=0.81]{helvet}     % Schriftart

\usepackage{url}
%\usepackage{tocloft}             % Paket für Table of Contents
\def\UrlBreaks{\do\a\do\b\do\c\do\d\do\e\do\f\do\g\do\h\do\i\do\j\do\k\do\l%
\do\m\do\n\do\o\do\p\do\q\do\r\do\s\do\t\do\u\do\v\do\w\do\x\do\y\do\z\do\0%
\do\1\do\2\do\3\do\4\do\5\do\6\do\7\do\8\do\9\do\-}%

\usepackage{xcolor}
\definecolor{urlred}{HTML}{660000}

\usepackage{hyperref}
\hypersetup{colorlinks=false}

%\usepackage{mdwlist}     % Änderung der Zeilenabstände bei itemize und enumerate
% \usepackage[scale=0.8,colorspec=0.9]{draftwatermark} % Wasserzeichen ``Entwurf''
% \SetWatermarkText{Vorbehaltlich\\redaktioneller\\Änderungen}

\parindent 0pt                 % Absatzeinrücken verhindern
\parskip 12pt                 % Absätze durch Lücke trennen

\setlength{\textheight}{23cm}
\usepackage{fancyhdr}
\pagestyle{fancy}
\fancyhead{} % clear all header fields
\cfoot{}
\lfoot{Zusammenkunft aller Physik-Fachschaften}
\rfoot{www.zapfev.de\\stapf@zapf.in}
\renewcommand{\headrulewidth}{0pt}
\renewcommand{\footrulewidth}{0.1pt}
\newcommand{\gen}{*innen}
\addto{\captionsngerman}{\renewcommand{\refname}{Quellen}}

%%%% Mit-TeXen Kommandoset
\usepackage[normalem]{ulem}
\usepackage{xcolor}
\usepackage{xspace} 

\newcommand{\replace}[2]{
    \sout{\textcolor{blue}{#1}}~\textcolor{blue}{#2}}
\newcommand{\delete}[1]{
    \sout{\textcolor{red}{#1}}}
\newcommand{\add}[1]{
    \textcolor{blue}{#1}}

\newif\ifcomments
\commentsfalse
%\commentstrue

\newcommand{\red}[1]{{\ifcomments\color{red} {#1}\else{#1}\fi}\xspace}
\newcommand{\blue}[1]{{\ifcomments\color{blue} {#1}\else{#1}\fi}\xspace}
\newcommand{\green}[1]{{\ifcomments\color{green} {#1}\else{#1}\fi}\xspace}

\newcommand{\repl}[2]{{\ifcomments{\color{red} \sout{#1}}{\color{blue} {\xspace #2}}\else{#2}\fi}}
%\newcommand{\repl}[2]{{\color{red} \sout{#1}\xspace{\color{blue} {#2}}\else{#2}\fi}\xspace}

\newcommand{\initcomment}[2]{%
	\expandafter\newcommand\csname#1\endcsname{%
		\def\thiscommentname{#1}%
		\definecolor{col}{rgb}{#2}%
		\def\thiscommentcolor{col}%
}}

% initcomment Name RGB-color
\initcomment{Philipp}{0, 0.5, 0}

%\renewcommand{\comment}[1]{{\ifcomments{\color{red} {#1}}{}\fi}\xspace}

\renewcommand{\comment}[2][\nobody]{
	\ifdefined#1
	{\ifcomments{#1 \expandafter\color{\thiscommentcolor}{\thiscommentname: #2}}{}\fi}\xspace
	\else
	{\ifcomments{\color{red} {#2}}{}\fi}\xspace
	\fi
}

\newcommand{\zapf}{ZaPF\xspace}

\let\oldgrqq=\grqq
\def\grqq{\oldgrqq\xspace}

\setlength{\parskip}{.6em}
\setlength{\parindent}{0mm}

%\usepackage{geometry}
%\geometry{left=2.5cm, right=2.5cm, top=2.5cm, bottom=3.5cm}

% \renewcommand{\familydefault}{\sfdefault}


\begin{document}

\hspace{0.87\textwidth}
\begin{minipage}{120pt}
	\vspace{-1.8cm}
	\includegraphics[width=80pt]{../logos/logo.pdf}
	\centering
	\small Zusammenkunft aller Physik-Fachschaften
\end{minipage}

\begin{center}
  \huge{Positionspapier zum Einsatz von KI-Tools in der Lehre}\vspace{.25\baselineskip}\\
  \normalsize
\end{center}
\vspace{1cm}

%%%% Metadaten %%%%

%\paragraph{Adressierte:} xyx


%\paragraph{Antragstellende:} xxx

%%%% Text des Antrages zur veröffentlichung %%%%

%\section*{Antragstext}

Die ZaPF sieht folgende Punkte als wichtige Grundlage für einen verantwortungsvollen Umgang mit KI in der Lehre:
\begin{enumerate}
    \item Generelle Verbote von KI-Tools an Hochschulen sind ebenso kontraproduktiv wie kurzsichtig. In Hinblick auf die zunehmende Relevanz von KI in verschiedenen Berufsfeldern würden Studierende einer wichtigen Möglichkeit der Talentförderung und des Lernens beraubt. Obgleich ein generelles Verbot abgelehnt wird, sollten die ethischen Aspekte einer Verwendung, insbesondere - aber nicht ausschließlich - in Bezug auf die Frage des Diebstahls von geistigem Eigentum, jederzeit reflektiert und optimiert werden.
    \item Der Einsatz von KI-Tools darf keinesfalls das Erlernen der notwendigen Fähigkeiten ersetzen, die in einem Modul erlernt werden sollen.
    \item Die Verwendung von KI-Tools im Kontext von fortgeschrittenen Praktika, Hausarbeiten oder Prozessoptimierung bietet großes Potenzial, insbesondere um repetitive, redundante Methoden, die aus subkulturellen Gepflogenheiten entstanden sind, zu optimieren. Dabei ist eine korrekte Kennzeichnung der verwendeten Methode unerlässlich und sollte im Curriculum verankert, sowie in dafür vorgesehenen Übungen gefestigt werden. Besonders der Vorteil der Nutzung von KI-Tools für Personen mit Lern- und Ausdrucksschwierigkeiten wird unterstrichen.
    \item In Bezug auf die Frage der erlaubten und explizit der verbotenen KI-Tools in Modulen sind die Fakultäten der Studierendenschaft eine verständliche Kennzeichnung schuldig, um die Rechtssicherheit für Studierende zu gewährleisten. Dies erfolgt am besten über die Kennzeichnung in Modulhandbüchern.\newpage
    \item Sofern KI-Tools für eine Studienleistung zugelassen wurden, darf die Verwendung auf keinen Fall eine Herabstufung der Leistungen und der Benotung von Studierenden zur Folge haben, es sei denn, dies wurde eindeutig als Teil einer Benotungsstufe angemerkt.
    \item Trotz der Vorteile von KI-Tools in der Lehre ist es unerlässlich, den gesetzlichen Rahmen des Datenschutzes einzuhalten, sowie die ethischen Implikationen von KI zu berücksichtigen.
\end{enumerate}

%\section*{Begründung}
%yyyy

%\vspace{1cm} 
%
\vfill
\begin{flushright}
	Verabschiedet am 31. Oktober 2023 \\
	auf der ZaPF in Düsseldorf.
\end{flushright}

\end{document}
