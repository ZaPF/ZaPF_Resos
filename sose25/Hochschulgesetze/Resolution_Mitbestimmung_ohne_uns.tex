\documentclass[DIV=calc]{scrartcl}
\usepackage[utf8]{inputenc}
\usepackage[T1]{fontenc}
\usepackage[ngerman]{babel}
\usepackage{graphicx}
\usepackage[draft, markup=underlined]{changes}
\usepackage{csquotes}
\usepackage{eurosym}

\usepackage{ulem}
%\usepackage[dvipsnames]{xcolor}
\usepackage{paralist}
%\usepackage{fixltx2e}
%\usepackage{ellipsis}
\usepackage[tracking=true]{microtype}

\usepackage{lmodern}              % Ersatz fuer Computer Modern-Schriften
%\usepackage{hfoldsty}

%\usepackage{fourier}             % Schriftart
\usepackage[scaled=0.81]{helvet}     % Schriftart

\usepackage{url}
%\usepackage{tocloft}             % Paket für Table of Contents
\def\UrlBreaks{\do\a\do\b\do\c\do\d\do\e\do\f\do\g\do\h\do\i\do\j\do\k\do\l%
\do\m\do\n\do\o\do\p\do\q\do\r\do\s\do\t\do\u\do\v\do\w\do\x\do\y\do\z\do\0%
\do\1\do\2\do\3\do\4\do\5\do\6\do\7\do\8\do\9\do\-}%

\usepackage{xcolor}
\definecolor{urlred}{HTML}{660000}

\usepackage{hyperref}
\hypersetup{colorlinks=false}

%\usepackage{mdwlist}     % Änderung der Zeilenabstände bei itemize und enumerate
%\usepackage[scale=0.8,colorspec=0.9]{draftwatermark} % Wasserzeichen ``Entwurf''
%\SetWatermarkText{Vorbehaltlich\\redaktioneller\\Änderungen}

\parindent 0pt                 % Absatzeinrücken verhindern
\parskip 12pt                 % Absätze durch Lücke trennen

\setlength{\textheight}{23cm}
\usepackage{fancyhdr}
\pagestyle{fancy}
\fancyhead{} % clear all header fields
\cfoot{}
\lfoot{Zusammenkunft aller Physik-Fachschaften}
\rfoot{www.zapfev.de\\stapf@zapf.in}
\renewcommand{\headrulewidth}{0pt}
\renewcommand{\footrulewidth}{0.1pt}
\newcommand{\gen}{*innen}
\addto{\captionsngerman}{\renewcommand{\refname}{Quellen}}

%%%% Mit-TeXen Kommandoset
\usepackage[normalem]{ulem}
\usepackage{xcolor}
\usepackage{xspace} 

\newcommand{\replace}[2]{
    \sout{\textcolor{blue}{#1}}~\textcolor{blue}{#2}}
\newcommand{\delete}[1]{
    \sout{\textcolor{red}{#1}}}
\newcommand{\add}[1]{
    \textcolor{blue}{#1}}

\newif\ifcomments
\commentsfalse
%\commentstrue

\newcommand{\red}[1]{{\ifcomments\color{red} {#1}\else{#1}\fi}\xspace}
\newcommand{\blue}[1]{{\ifcomments\color{blue} {#1}\else{#1}\fi}\xspace}
\newcommand{\green}[1]{{\ifcomments\color{green} {#1}\else{#1}\fi}\xspace}

\newcommand{\repl}[2]{{\ifcomments{\color{red} \sout{#1}}{\color{blue} {\xspace #2}}\else{#2}\fi}}
%\newcommand{\repl}[2]{{\color{red} \sout{#1}\xspace{\color{blue} {#2}}\else{#2}\fi}\xspace}

\newcommand{\initcomment}[2]{%
	\expandafter\newcommand\csname#1\endcsname{%
		\def\thiscommentname{#1}%
		\definecolor{col}{rgb}{#2}%
		\def\thiscommentcolor{col}%
}}

% initcomment Name RGB-color
%###############
\initcomment{Philipp}{0, 0.5, 0}

%\renewcommand{\comment}[1]{{\ifcomments{\color{red} {#1}}{}\fi}\xspace}

\renewcommand{\comment}[2][\nobody]{
	\ifdefined#1
	{\ifcomments{#1 \expandafter\color{\thiscommentcolor}{\thiscommentname: #2}}{}\fi}\xspace
	\else
	{\ifcomments{\color{red} {#2}}{}\fi}\xspace
	\fi
}

\newcommand{\zapf}{ZaPF\xspace}

\let\oldgrqq=\grqq
\def\grqq{\oldgrqq\xspace}

\setlength{\parskip}{.6em}
\setlength{\parindent}{0mm}

%\usepackage{geometry}
%\geometry{left=2.5cm, right=2.5cm, top=2.5cm, bottom=3.5cm}

% \renewcommand{\familydefault}{\sfdefault}

\begin{document}

\hspace{0.87\textwidth}
\begin{minipage}{120pt}
	\vspace{-1.8cm}
	\includegraphics[width=80pt]{../logos/logo.pdf}
	\centering
	\small Zusammenkunft aller Physik-Fachschaften
\end{minipage}

\begin{center}
  \huge{Resolution: Mitbestimmung ohne uns?}
  \vspace{.25\baselineskip}
  \normalsize
\end{center}
\vspace{1cm}

%%%% Metadaten %%%%

%\paragraph{Adressierte:} Fraktionen aller Länder, MeTaFa, VerDi, DGB, GEW, TVStud


%\paragraph{Antragstellende:} xxx

%%%% Text des Antrages zur veröffentlichung %%%%


%\section*{Antragstext}
In Brandenburg wurden letztes Jahr bei der Novellierung des
Personalvertretungsgesetzes studentische Personalräte nach dem Vorbild Berlins eingeführt. Wir begrüßen dies und fordern nun die anderen Bundesländer auf, diesem Beispiel zu folgen und vollwertige, gleichberechtigte Personalräte für studentisch Beschäftigte einzuführen.

Studentisch Beschäftigte stellen im öffentlichen Dienst die größte Repräsentationslücke dar. Im uneinheitlich föderalen System werden sie nur teilweise von bestehenden Personalräten vertreten und noch seltener an ihnen beteiligt.

Auch die aktuell bestehenden, eigenständigen Interessenvertretungen der studentisch Beschäftigten, wie sie in Hessen, NRW und Thüringen existieren, haben kaum gesetzliche Grundlage, um ihren Mitsprache- bzw. Vertretungsanspruch geltend zu machen und sind damit so gut wie handlungsunfähig.

Als ein signifikanter Anteil der Beschäftigten an den Hochschulen darf es nicht sein, dass studentisch Beschäftigte wenig bis keinen Einfluss auf ihre eigenen Arbeitsbedingungen haben. Zudem bieten Personalräte eine wichtige Beratungsstelle zwischen Kolleg*innen.

Die einfache Eingliederung studentisch Beschäftigter in bestehende Personalräte ist keine Alternative zu studentischen Personalräten, da aufgrund kurzer Beschäftigungsdauer und langer Amtszeiten studentisch Beschäftigten die aktive Mitarbeit in Personalräten verwehrt wird, da bei Unterbrechung des Arbeitsverhältnisses die Mitgliedschaft im Personalrat entfällt.

In der Regel werden Angestellte mit Personalverantwortung von dem passiven Wahlrecht zu Personalvertretungen ausgeschlossen. In den Hochschulen haben die wissenschaftlichen Mitarbeiter*innen meist aber auch eine Weisungsbefugnis gegenüber studentischen und wissenschaftlichen Hilfskräften. Das führt in der Realität zu Interessenkonflikten bei der Repräsentation.

Aus diesen Gründen fordern wir die Einrichtung vollwertiger, gleichberechtigter studentischer Personalräte an allen öffentlichen Hochschulen. Dazu gehört explizit das Recht auf Mitbestimmung bei der Einstellung. Weiterhin fordern wir die
Einbeziehung des Personalrates bei Befristung von Arbeitsverhältnissen und in Ausschreibungsprozesse von Stellen.


Für die studentischen Personalräte sollte die Länge der Amtszeiten der Realität der Beschäftigungsdauer entsprechen und es muss einen Schutz vor frühzeitigem Ausscheiden aus dem Personalrat durch Auslaufen einer befristeten Beschäftigung, analog zu §94 Abs. 6 des Brandenburgischen LPersVG, geben.

Die Zeit für Mitbestimmung ist jetzt!




%\section*{Begründung}


\vspace{1cm} 
%
\vfill
\begin{flushright}
	Verabschiedet am 04. Mai 2025 \\
	auf der ZaPF in Erlangen.
\end{flushright}

\end{document}