\documentclass[DIV=calc]{scrartcl}
\usepackage[utf8]{inputenc}
\usepackage[T1]{fontenc}
\usepackage[ngerman]{babel}
\usepackage{graphicx}
\usepackage[draft, markup=underlined]{changes}
\usepackage{csquotes}

\usepackage{ulem}
%\usepackage[dvipsnames]{xcolor}
\usepackage{paralist}
\usepackage{fixltx2e}
%\usepackage{ellipsis}
\usepackage[tracking=true]{microtype}
\usepackage[a4paper,top=3cm,bottom=2cm,left=3cm,right=3cm]{geometry}

\usepackage{lmodern}              % Ersatz fuer Computer Modern-Schriften
%\usepackage{hfoldsty}

%\usepackage{fourier}             % Schriftart
\usepackage[scaled=0.81]{helvet}     % Schriftart

\usepackage{url}
\usepackage{tocloft}             % Paket für Table of Contents

%\usepackage{xcolor}
\usepackage{ulem,xspace,xcolor}
\definecolor{urlred}{HTML}{660000}

\usepackage{hyperref}
\hypersetup{
    colorlinks=true,
    linkcolor=black,    % Farbe der internen Links (u.a. Table of Contents)
    urlcolor=black,    % Farbe der url-links
    citecolor=black} % Farbe der Literaturverzeichnis-Links

\usepackage{mdwlist}     % Änderung der Zeilenabstände bei itemize und enumerate
\usepackage{draftwatermark} % Wasserzeichen ``Entwurf''
\SetWatermarkText{}

\parindent 0pt                 % Absatzeinrücken verhindern
\parskip 12pt                 % Absätze durch Lücke trennen

\setlength{\textheight}{23cm}
\usepackage{fancyhdr}
\pagestyle{fancy}
\fancyhead{} % clear all header fields
\cfoot{}
\lfoot{Zusammenkunft aller Physik-Fachschaften}
\rfoot{www.zapfev.de\\stapf@zapf.in}
\renewcommand{\headrulewidth}{0pt}
\renewcommand{\footrulewidth}{0.1pt}
\newcommand{\gen}{*innen}
\addto{\captionsngerman}{\renewcommand{\refname}{Quellen}}

%%%% Mit-TeXen Kommandoset
\usepackage[normalem]{ulem}
\usepackage{xcolor}

\usepackage{enumitem}
\setenumerate[1]{label=\thesection.\arabic*.}

\def\zapf{ZaPF}

\newif\ifcomments
\commentsfalse
\commentstrue

\begin{document}
  \hspace{0.87\textwidth}
  \begin{minipage}{120pt}
  	\vspace{-1.8cm}
  	\includegraphics[width=80pt]{../logo.pdf}
  	\centering
  	\small Zusammenkunft aller Physik-Fachschaften
  \end{minipage}
  \begin{center}
    \huge{Positionspapier zur Bibliotheks- und Raumentwicklung}\vspace{.25\baselineskip}\\
  	\normalsize
  \end{center}
  \vspace{1cm}
    %
    % \begin{center}
    %     \normalsize
    % \end{center}
    % \vspace{1cm}

%%%% Text des Antrags %%%%

Die Digitalisierung bedeutet einen Umbruch für die Bibliotheks- und Raumlandschaft.
Vielerorts unterliegen Strukturen, die zuvor über Jahrzehnte in kleinen Schritten aufgebaut wurden, innerhalb von kürzester Zeit einem starken Wandel.

\section{Problembeschreibung}

Obwohl die Nutzung der Bibliotheken an sehr vielen Orten erheblich steigt und es teilweise Einlasssperren wegen Überfüllung gibt, treten trotzdem oft folgende problematische Entwicklungen auf:

%1
\begin{enumerate}
%1
\item Sukzessives Verschwinden von Printmedien aus den Bibliotheken.\\
\\Da die Aufbewahrung von Printmedien viel Platz einnimmt, tendieren viele Bibliotheken dazu, keine neuen Printmedien mehr zu kaufen bzw. bestehende auszulagern.
%2
\item Abbau und drohende Schließung von dezentralen Bibliotheken. \\
%
\\ Unserer Erfahrung nach werden vor allem kleinere Bibliotheken oder Zweigstellen zur vermeintlichen Ressourceneinsparung hinsichtlich Öffnungszeiten, Mitarbeitenden und Medienbeständen abgebaut oder ganz geschlossen, weil gedruckte Medien und damit Bibliotheken als veraltet angesehen werden.

\end{enumerate}
In dieser Situation können viele positive Aspekte für die nächsten Jahrzehnte in die Wege geleitet werden. Es besteht allerdings die Gefahr, etablierte Infrastrukturen nachhaltig zu sch"adigen. Die Debatte um die Weiterentwicklung der Bibliotheken wird dabei oft zu verkürzt geführt.\\ \\

\section{Folgen von vermeintlichen Ressourceneinsparungen}

\textbf{Es trifft nicht zu, dass Ressourcen durch Abbau von Printmedien und Bibliotheken ohne Beeinträchtigung des Studiums eingespart werden können:}

%2
\begin{enumerate}
%1
\item Wenn Medien vermehrt nur noch elektronisch zur Verfügung stehen, m"ussen entsprechend viele elektronische Endgeräte von Studierenden selbst angeschafft werden, was vor allem finanzschwache Studierende beeintr'chtigt.
Ansonsten muessten solche Geraete von der Hochschule zum Verleih zur Verf"ugung gestellt werden.

Außerdem werden dadurch Studierende gezwungen, mehr Zeit am Bildschirm zu verbringen, was insbesondere Menschen benachteiligt, die nicht über längere Zeit an einem Bildschirm arbeiten können.


%2
\item Durch die Sortierung von Printmedien in den Regalen nach Kategorien werden Studierende in Bibliotheken dazu angeregt, unabhängig von einem bestimmten Titel zu stöbern und sich umfangreicher informieren zu können, ohne große Wege zurückzulegen.
Bei einer reinen Recherche über den Online-Katalog der Bibliothek fällt dieser Punkt weg.
%3
\item Die zunehmende Auslagerung von Printmedien in zentrale Lager, die auf Anfrage an  Zweigstellen geliefert werden, trägt nicht zu einem guten Arbeitsklima bei.
Das Anliefern kostet Zeit für Studierende und Mitarbeitende.
Außerdem müssten für den Transport der Medien weitere Ressourcen und Infrastruktur aufgewendet werden.
Jegliche Auslagerung von Printmedien (z.B. zur Schaffung von Arbeitsplätzen),  sollte in allen Fällen unter Einbeziehung aller betroffenen Statusgruppen besprochen und beschlossen werden.
%4
\item Bibliotheken stellen oft einen Gro"steil der studentischen Arbeitsr"aume dar.
Im Falle einer Schließung würden diese Arbeitsräume wegfallen und müssten ersetzt werden.
Daraus resultieren keine effektiven Ressourceneinsparungen.
%5
\item Beratung und Unterstützung bei der Literaturrecherche kann nur durch geschultes Personal geschehen.
Durch den resultierenden Stellenabbau könnte nicht mehr sichergestellt werden, dass jederzeit Ansprechpersonen bei Problemen zur Verfügung stehen.
%6
\item Die Anzahl der Bibliotheksstandorte vergrößert nur scheinbar die Zahl der insgesamt zu kaufenden Printmedien.
Die tatsächlich benötigte Anzahl der Exemplare ist jedoch abhängig von der Anzahl der Studierenden, nicht der Bibliotheken.
%7
\item Durch das Zusammenlegen von Teil- oder Fachbereichsbibliotheken entstehen kaum Redundanzen in der vorzuhaltenden Literatur.
Oft f"uhrt eine gro"se Distanz zur n"achsten Bibliothek zum Entstehen nicht professionell erschlossener und f"ur Studierende unzug"anglicher \glqq Privatbibliotheken\grqq\ einzelner Lehrst"uhle.
\end{enumerate}

\section{Gestaltungsmöglichkeiten}

\textbf{Die ZaPF spricht sich dafür aus, dass Hochschulen insbesondere folgende Punkte bei der zukünftigen Gestaltung von Bibliotheken und Lernräumen berücksichtigen:} \\

%3
\begin{enumerate}
%1
\item Bibliotheken sind oft das „Herz der Infrastruktur vor Ort“.
Ihr Wandel zieht somit Veränderungen der Raum- und Infrastruktur in ihrer Umgebung nach sich, die erhebliche Auswirkungen auf die Kultur im Fachbereich haben können.
Die Weiterentwicklung der Bibliothekslandschaft darf deshalb nicht losgelöst von der Weiterentwicklung der Raum- und Infrastruktur als Ganzes betrachtet und betrieben werden.
%2
\item Die steigende Zahl der Studierenden und der zugehörigen Mittel sollten genutzt werden, um die dezentrale Infrastruktur auszubauen.
%3
\item Fachbibliotheken und Lernr"aume bieten Raum f"ur Begegnung.
Dort treffen sich Studierende und tauschen sich aus.\\
Die studentische Vernetzung an gro"sen Hochschulen ist aufgrund der hohen Studierendenzahl schwieriger als bei kleinen.
Dadurch sind sie auf solche Räume angewiesen, da unserer Erfahrung nach der Studienerfolg in den Naturwissenschaften erheblich davon abhängt, ob Studierende innerhalb ihrer Kohorte zusammenarbeiten und Teil von Lerngruppen werden.
%4
\item Fachbibliotheken repräsentieren gewachsene Fachkulturen, u.a. in der Auswahl der Bücher, deren Sortierung und der Schulung des Personals.
Zudem haben die verschiedenen Bibliotheken sehr verschiedene Lernkulturen, die wir als erhaltenswert erachten.
%5
\item Gleichzeitig sollten die Schnittstellen zwischen den verschiedenen Bibliotheken innerhalb einer Hochschule bzw. Stadt verbessert und z.B. ein einheitliches Ausleihwesen für alle Bücher mit Rückgabemöglichkeit in sämtlichen Bibliotheken etabliert werden.
\end{enumerate}

\textbf{Die ZaPF spricht sich somit gegen den Trend aus, das Bibliothekswesen zu zentralisieren.
Wir wollen, dass die Hochschulen auch den Studierenden der Zukunft ermöglichen, das Beste aus gewachsenen und neuen Strukturen für sich zu nutzen:
Freie Wahl zwischen Print- und digitalen Medien und eine lebendige Gestaltung von Räumen für verschiedene Lernkulturen.}
\vfill
    \begin{flushright}
        Verabschiedet am 3.11.2019 in Freiburg.
    \end{flushright}
\end{document}
