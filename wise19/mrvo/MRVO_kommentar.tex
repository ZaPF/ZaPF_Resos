%\documentclass[a4paper,draft]{scrartcl}
\documentclass[a4paper]{scrartcl}

%% Language and font encodings
\usepackage[ngerman]{babel}
\usepackage[utf8]{inputenc}
\usepackage[T1]{fontenc}
\usepackage[dvipsnames,svgnames,table]{xcolor} % Kenne SVG Farbnamen \
\definecolor{Bernd}{rgb}{0.13, 0.55, 0.13}
\colorlet{Bruno}{Rhodamine}
\colorlet{Brutus}{BrickRed}
\usepackage{csquotes}% Anführungszeichen
\usepackage{enumitem}% Buchstaben in enumerate
\usepackage[scaled=0.81]{helvet}     % Schriftart

%% Sets page size and margins

% \usepackage[a4paper,top=3cm,bottom=2cm,left=3cm,right=3cm,marginparwidth=1.75cm]{geometry}

% eliminiert die Einrückung bei einem neuen Absatz
\setlength{\parindent}{0pt}
\setkomafont{paragraph}{\normalfont}

%% Useful packages
\usepackage{amsmath}
\usepackage{graphicx}
\usepackage{hyperref}
\hypersetup{
	colorlinks=true,
	linkcolor=black,    % Farbe der internen Links (u.a. Table of Contents)
	urlcolor=black,    % Farbe der url-links
	citecolor=black} % Farbe der Literaturverzeichnis-Links

% Damit gibt es eine Möglichkeit Kommentare in \todo{} in einer "internen" Version anzuzeigen oder in einer "externen" Version _nicht_ anzuzeigen.
% Zum Ausblenden muss in Zeile 1 das draft entfernt werden.

% Dinge von Jörg, um das Aussehen des Inhaltsverzeichnis bzw. die Sektionüberschriften anzupassen :)
%\renewcommand{\thesection}{\hspace{-1em}Teil \arabic{section}}
\renewcommand{\thesection}{}
%\renewcommand{\thesubsection}{\S\,\arabic{subsection}}
\renewcommand{\thesubsection}{}
\renewcommand{\thesubsubsection}{\thesubsection\alph{subsubsection})}

%% Der abschliessende Punkt hinter den Nummern ist ueberfluessig, also wegdefinieren:
%Originaldefinition
% Hier enden die Dinge von Jörg ;-)

\bibliographystyle{alpha}

%%%% Mit-TeXen Kommandoset
\usepackage[normalem]{ulem}
\usepackage{xcolor}

\parindent 0pt                 % Absatzeinrücken verhindern
\parskip 12pt                 % Absätze durch Lücke trennen
\setlength{\textheight}{23cm}
\usepackage{fancyhdr}
\pagestyle{fancy}
\fancyhead{} % clear all header fields
\cfoot{}
\lfoot{Zusammenkunft aller Physik-Fachschaften}
\rfoot{www.zapfev.de\\stapf@zapf.in}
\renewcommand{\headrulewidth}{0pt}
\renewcommand{\footrulewidth}{0.1pt}
\newcommand{\gen}{*innen}
\addto{\captionsngerman}{\renewcommand{\refname}{Quellen}}

\begin{document}
\hspace{0.87\textwidth}
\begin{minipage}{120pt}
	\vspace{-1.8cm}
	\includegraphics[width=80pt]{../logo.pdf}
	\centering
	\small Zusammenkunft aller Physik-Fachschaften
\end{minipage}
\begin{center}
	\huge{Kommentierte Version der Musterrechtsverordnung (MRVO) nach der Fassung vom 7.12.2017}\vspace{.25\baselineskip}\\
	\normalsize
\end{center}
\vspace{1cm}
\begin{abstract}
\noindent
In diesem Dokument werden vergangene Beschlüsse der ZaPF, die den jeweiligen Stellen der MRVO entsprechen, \textcolor{Bernd}{in der Farbe Bernd} einsortiert.
Diese sind für die Arbeit der Gutachter*innen gedacht.\\
Kommentare für eine Überarbeitung dieses Dokuments sind \textcolor{Brutus}{in der Farbe Bärbel} gehalten und für das Nutzen in einem weiteren AK vermerkt.\\
Änderungsvorschläge und Hinweise für das Mitwirken an einer Überarbeitung der MRVO sind \textcolor{Bruno}{in der Farbe Bruno} geschrieben. Wenn ein Punkt sowohl Interessant für \textcolor{Bernd}{Gutachter*innen} als auch für einen möglichen \textcolor{Bruno}{ÄnderungsAK} sind, ist \textcolor{Bernd}{Bernd} die dominante Farbe.\\
Der schwarze Text ist ab dem Inhaltsverzeichnis der Wortlaut der MRVO.
Die Begründung der MRVO ist hier nicht enthalten.\\
Ergänzend zu diesem Dokument existiert eine Liste zur Schnellüberprüfung eines Verfahrens in Form der Akkreditierungsrichtlinien \cite{akkreditierungsrichtlinien}.\\
\textcolor{Brutus}{\emph{Hier ist das durch den Beschluss der WinterZaPF 2019 ersetze Papier verlinkt, da zum Zeitpunkt der letzten Bearbeitung die aktuelle Fassung noch nicht veröffentlicht war.}}\\
\textcolor{Brutus}{
Die Einträge der ersten \pageref{todo:BisHierhinLinksKontrolliert} (genauer: bis §12 \nameref{todo:BisHierhinLinksKontrolliert}) und letzten Seiten (ab \pageref{todo:AbHierLinksKontrolliert})  (genauer: ab \nameref{todo:AbHierLinksKontrolliert}) wurden auf Richtigkeit geprüft (keine Garantie :), auf Vollständigkeit (d.h. alle Inhalte der Resos sind eingetragen) ist bisher bis einschließlich SoSe 2008 geprüft worden und ebenso von 2010 bis 2019.
}

\begin{flushright}
	Verabschiedet am 2.11.2019 in Freiburg.
\end{flushright}
\end{abstract}

%Musterrechtsverordnung gemäß Artikel 4 Absätze 1 – 4
%Studienakkreditierungsstaatsvertrag
%(Beschluss der Kultusministerkonferenz vom 07.12.2017)


%\paragraph{TODOs}

%\begin{itemize}
%\item nochmal alle Quellen durchgehen, ob wirklich aus allen alles gezogen
%\item Fortgeschrittenenpraktika: noch nicht beschlossen
%\item AK Vorläufige Verträge: noch nicht beschlossen
%\item Kritik an MRVO aus POS: Sose2018-AkkRL und RESO: Sose2018-RV in \textcolor{Bruno}{Farbe Bruno}
%\item Aus \cite{RESO: SoSe2002-RL} Forderungnach sinnvoller Anerkennung bisheriger Studien- und Prüfungsleistungen einfügen.
% nach deutschsprachigen Pflicht-Bachelormodulen sowie
%\item POS Papier Wisskomm aus Würzburg einbauen
%\item \textcolor{Bernd}{Bernd} und \textcolor{Bruno}{Bruno}: Korrekte Zuordnung prüfen. (Ist der Vorschlag im Rahmen der MRVO anwendbar oder fordert er eine Änderung dieser?)
%\item \textcolor{Bernd}{Bernd} und \textcolor{Bruno}{Bruno} abstimmen lassen
%\item Layout überdenken und vereinheitlichen.
%\item Kommentare zu unseren Beschlüssen (\emph{kursiv}) mit Kürzel [ANM] versehen und in \textcolor{Brutus}{Brutus}

%\end{itemize}

\newpage
\tableofcontents

\clearpage
\section{Teil 1 Allgemeine Vorschriften}

\subsection{§ 1 Anwendungsbereich}
\paragraph{(1)} Diese Verordnung regelt auf Grund von Artikel 4 des Staatsvertrages über die Organisation eines gemeinsamen Akkreditierungssystems zur Qualitätssicherung in Studium und Lehre an deutschen Hochschulen (Studienakkreditierungsstaatsvertrag; GVBl.) das Nähere zu den formalen Kriterien nach Artikel 2 Absatz 2, zu den fachlich-inhaltlichen Kriterien nach Artikel 2 Absatz 3 sowie zum Verfahren nach Artikel 3 des Studienakkreditierungsstaatsvertrages.
\paragraph{(2)} Soweit in dieser Verordnung keine besonderen Bestimmungen getroffen werden, gelten die nachfolgenden Regelungen der Programmakkreditierung auch für Ausbildungsgänge an staatlichen und staatlich anerkannten Berufsakademien, die zu der Abschlussbezeichnung Bachelor führen. Ein auf der Grundlage dieser (Muster-) Rechtsverordnung akkreditierter Bachelorabschluss steht hochschulrechtlich dem Bachelorabschluss einer Hochschule gleich.
\subsection{§ 2 Formen der Akkreditierung}
Formen der Akkreditierung sind die Verfahren nach Artikel 3 Absatz 1 Nummer 1 Studienakkreditierungsstaatsvertrag (Systemakkreditierung), nach Artikel 3 Absatz 1 Nummer 2 (Programmakkreditierung) oder alternative Akkreditierungsverfahren nach Artikel 3 Absatz 1 Nummer 3.

\section{Teil 2 Formale Kriterien für Studiengänge}
\subsection{§ 3 Studienstruktur und Studiendauer}
\paragraph{(1)} Im System gestufter Studiengänge ist der Bachelorabschluss der erste berufsqualifizierende Regelabschluss eines Hochschulstudiums; der Masterabschluss stellt einen weiteren berufsqualifizierenden Hochschulabschluss dar. Grundständige Studiengänge, die unmittelbar zu einem Masterabschluss führen, sind mit Ausnahme der in Absatz 3 genannten Studiengänge ausgeschlossen.\\

\textcolor{Bernd}{\textbf{\cite{RESO: SoSe2002-RL}} Die ZaPF erachtet es als hartes Akkreditierungskriterium, dass der Bachelor-Abschluss tatsächlich eine solide physikalische Grundausbildung und nicht nur die Zugangsvoraussetzung für den Master, also ein abgeschnittenes Diplomstudium darstellt.}\\

\paragraph{(2)} Die Regelstudienzeiten für ein Vollzeitstudium betragen sechs, sieben oder acht Semester bei den Bachelorstudiengängen und vier, drei oder zwei Semester bei den Masterstudiengängen. Im Bachelorstudium beträgt die Regelstudienzeit im Vollzeitstudium mindestens drei Jahre. Bei konsekutiven Studiengängen beträgt die Gesamtregelstudienzeit im Vollzeitstudium fünf Jahre (zehn Semester). Wenn das Landesrecht dies vorsieht, sind kürzere und längere Regelstudienzeiten bei entsprechender studienorganisatorischer  Gestaltung ausnahmsweise möglich, um den Studierenden eine individuelle Lernbiografie, insbesondere durch Teilzeit-, Fern-, berufsbegleitendes oder duales Studium sowie berufspraktische Semester, zu ermöglichen.
Abweichend von Satz 3 können in den künstlerischen Kernfächern an Kunst und Musikhochschulen nach näherer Bestimmung des Landesrechts konsekutive Bachelor- und Masterstudiengänge auch mit einer Gesamtregelstudienzeit von sechs Jahren eingerichtet werden.\\

\textcolor{Bernd}{\textbf{\cite{RESO: SoSe2002-RL}} Die ZaPF erachtet es als hartes Akkreditierungskriterium, dass die Dauer eines Bachelor-Studiengangs inklusive Bachelor-Arbeit 6 Semester beträgt.}\\

\textcolor{Bernd}{\textbf{\cite{RESO: SoSe2002-RL}} Die ZaPF erachtet es als hartes Akkreditierungskriterium, dass die Dauer eines Master-Studiengangs inklusive Master-Arbeit 4 Semester beträgt.}\\

\emph{\textcolor{Brutus}{[ANM] Das muss so nicht mehr unbedingt stimmen. Es spricht nicht unbedingt etwas gegen Bachelor mit einem Konzept mit einem Umfang von 6, 7 oder 8 Semester Vollzeitstudium bei entsprechend angepasstem Master.}}

\paragraph{(3)} Theologische Studiengänge, die für das Pfarramt, das Priesteramt und den Beruf der Pastoralreferentin oder des Pastoralreferenten qualifizieren (\enquote{Theologisches Vollstudium}), müssen nicht gestuft sein und können eine Regelstudienzeit von zehn Semestern aufweisen.
\subsection{§ 4 Studiengangsprofile}
\paragraph{(1)} Masterstudiengänge können in \enquote{anwendungsorientierte} und \enquote{forschungsorientierte} unterschieden werden. Masterstudiengänge an Kunst und Musikhochschulen können ein besonderes künstlerisches Profil haben. Masterstudiengänge, in denen die Bildungsvoraussetzungen für ein Lehramt vermittelt werden, haben ein besonderes lehramtsbezogenes Profil. Das jeweilige Profil ist in der Akkreditierung festzustellen.

\paragraph{(2)} Bei der Einrichtung eines Masterstudiengangs ist festzulegen, ob er konsekutiv oder weiterbildend ist. Weiterbildende Masterstudiengänge entsprechen in den Vorgaben zur Regelstudienzeit und zur Abschlussarbeit den konsekutiven Masterstudiengängen und führen zu dem gleichen Qualifikationsniveau und zu denselben Berechtigungen.

\paragraph{(3)} Bachelor- und Masterstudiengänge sehen eine Abschlussarbeit vor, mit der die Fähigkeit nachgewiesen wird, innerhalb einer vorgegebenen Frist ein Problem aus dem jeweiligen Fach selbständig nach wissenschaftlichen bzw. künstlerischen Methoden zu bearbeiten.\\


\textcolor{Bernd}{\textbf{\cite{RESO: SoSe2002-RL}} Es gibt eine Bachelorthesis mit Umfang 2-6 Monate. Es gibt eine Masterthesis, mit mind. 6 Monaten als hartem Kriteriem bzw. mind. 9 Monaten als weichem Kriterium.}\\ % Redundant mit § 8

\textcolor{Brutus}{\emph{[ANM] Referenz DQR}}\\

\textcolor{Brutus}{\emph{[ANM] Sehen wir das noch so? Im AK in Bonn gab es da gemischte Meinungen}}


\subsection{§ 5 Zugangsvoraussetzungen und Übergänge zwischen Studienangeboten}
\paragraph{(1)} Zugangsvoraussetzung für einen Masterstudiengang ist ein erster berufsqualifizierender Hochschulabschluss.\\


\textcolor{Bernd}{\textbf{\cite{POS: WiSe2017-AkkRL}} früher stand hier \enquote{in der Regel}, damit konnte man auch mit einer entsprechenden Ausbildung in den Master Quereinsteigen.}\\

Bei weiterbildenden und künstlerischen Masterstudiengängen kann der berufsqualifizierende Hochschulabschluss durch eine Eingangsprüfung ersetzt werden, sofern Landesrecht dies vorsieht. Weiterbildende Masterstudiengänge setzen qualifizierte berufspraktische Erfahrung von in der Regel nicht unter einem Jahr voraus.

\paragraph{(2)} Als Zugangsvoraussetzung für künstlerische Masterstudiengänge ist die hierfür erforderliche besondere künstlerische Eignung nachzuweisen. Beim Zugang zu weiterbildenden künstlerischen Masterstudiengängen können auch berufspraktische Tätigkeiten, die während des Studiums abgeleistet werden, berücksichtigt werden, sofern Landesrecht dies ermöglicht. Das Erfordernis berufspraktischer Erfahrung gilt nicht an Kunsthochschulen für solche Studien, die einer Vertiefung freikünstlerischer Fähigkeiten dienen, sofern landesrechtliche Regelungen dies vorsehen.
\paragraph{(3)} Für den Zugang zu Masterstudiengängen können weitere
Voraussetzungen entsprechend Landesrecht vorgesehen werden.\\

\textcolor{Bernd}{\textbf{\cite{RESO: SoSe2008-MaZu}} Die ZaPF fordert, dass alle Abschlüsse B. Sc. in Physik aus akkreditierten Studiengängen Physik gleichwertig behandelt werden und keine Zulassungsprüfungen für einen konsekutiven Master gefordert werden.}\\

\textcolor{Bernd}{\textbf{\cite{RESO: WiSe2008-RL}} Pflichtvorlesungen in einem deutschsprachigen Bachelorstudiengang sollen in deutscher Sprache angeboten werden.}\\

\textcolor{Bernd}{\textbf{\cite{RESO: WiSe2008-RL}} Wenn Pflichtvorlesungen oder ein Großteil des Wahlbereichs Fremdsprachenkenntnisse erfordern, so gehören diese (mindestens als Hinweis) in die Zugangsvoraussetzungen.}\\

\textcolor{Bernd}{\textbf{\cite{PROT: WiSe2015-RL}} Die Zulassung zum Studiengang soll nicht restriktiv gehandhabt werden. Insbesondere sollen fakultative Veranstaltungen vor Studienbeginn (Brückenkurse o.Ä.) keine Zulassungsvoraussetzung sein oder Inhalte aus dem Studium ersetzen. Quereinsteiger sollen nicht benachteiligt werden.}\\


\textcolor{Bernd}{\textbf{\cite{RESO: SoSe2002-RL}} Es muss ein vernünftiges und faires Konzept zur Anrechnung bisheriger Studien- und Prüfungsleistungen geben.}\\


\subsection{§ 6 Abschlüsse und Abschlussbezeichnungen}
\paragraph{(1)} Nach einem erfolgreich abgeschlossenen Bachelor- oder
Masterstudiengang wird jeweils nur ein Grad, der Bachelor- oder Mastergrad,
verliehen, es sei denn, es handelt sich um einen Multiple-Degree-Abschluss.
Dabei findet keine Differenzierung der Abschlussgrade nach der Dauer der
Regelstudienzeit statt.
\paragraph{(2)}
Für Bachelor- und konsekutive Mastergrade sind folgende Bezeichnungen
zu verwenden:
\begin{enumerate}
\item Bachelor of Arts (B.A.) und Master of Arts (M.A.) in den Fächergruppen Sprach- und Kulturwissenschaften, Sport, Sportwissenschaft, Sozialwissenschaften, Kunstwissenschaft, Darstellende Kunst und bei entsprechender inhaltlicher Ausrichtung in der Fächergruppe Wirtschaftswissenschaften sowie in künstlerisch angewandten Studiengängen,
\item Bachelor of Science (B.Sc.) und Master of Science (M.Sc.) in den Fächergruppen Mathematik, Naturwissenschaften, Medizin, Agrar-, Forst- und Ernährungswissenschaften, in den Fächergruppen Ingenieurwissenschaften und Wirtschaftswissenschaften bei entsprechender inhaltlicher Ausrichtung,
\item Bachelor of Engineering (B.Eng.) und Master of Engineering (M.Eng.) in der Fächergruppe Ingenieurwissenschaften bei entsprechender inhaltlicher Ausrichtung,
\item Bachelor of Laws (LL.B.) und Master of Laws (LL.M.) in der Fächergruppe Rechtswissenschaften,
\item Bachelor of Fine Arts (B.F.A.) und Master of Fine Arts (M.F.A.) in der Fächergruppe Freie Kunst,
\item Bachelor of Music (B.Mus.) und Master of Music (M.Mus.) in der
Fächergruppe Musik,
\item Bachelor of Education (B.Ed.) und Master of Education (M.Ed.) für Studiengänge, in denen die Bildungsvoraussetzungen für ein Lehramt vermittelt werden.
\end{enumerate}
Für einen polyvalenten Studiengang kann entsprechend dem inhaltlichen Schwerpunkt des Studiengangs eine Bezeichnung nach den Nummern 1 bis 7 vorgesehen werden. Fachliche Zusätze zu den Abschlussbezeichnungen und gemischtsprachige Abschlussbezeichnungen sind ausgeschlossen. Bachelorgrade mit dem Zusatz „honours“ („B.A. hon.“) sind ausgeschlossen. Bei interdisziplinären und Kombinationsstudiengängen richtet sich die Abschlussbezeichnung nach demjenigen Fachgebiet, dessen Bedeutung im Studiengang überwiegt. Für  Weiterbildungsstudiengänge dürfen auch Mastergrade verwendet werden, die von den vorher genannten Bezeichnungen abweichen. Für theologische Studiengänge, die für das Pfarramt, das Priesteramt und den Beruf der Pastoralreferentin oder des Pastoralreferenten qualifizieren (\enquote{Theologisches Vollstudium}), können auch abweichende Bezeichnungen verwendet werden.
\paragraph{(3)} In den Abschlussdokumenten darf an geeigneter Stelle verdeutlicht werden, dass das Qualifikationsniveau des Bachelorabschlusses einem Diplomabschluss an Fachhochulen bzw. das Qualifikationsniveau eines Masterabschlusses einem Diplomabschluss an Universitäten oder gleichgestellten Hochschulen entspricht.
\paragraph{(4)} Auskunft über das dem Abschluss zugrunde liegende Studium im Einzelnen erteilt das Diploma Supplement, das Bestandteil jedes Abschlusszeugnisses ist.\\

\emph{\textcolor{Brutus}{[ANM]MRVO schließt nicht explizit Akkreditierung von Diplomstudiengängen aus.}}

\subsection{§ 7 Modularisierung}
\paragraph{(1)} Die Studiengänge sind in Studieneinheiten (Module) zu gliedern, die durch die Zusammenfassung von Studieninhalten thematisch und zeitlich abgegrenzt sind. Die Inhalte eines Moduls sind so zu bemessen, dass sie in der Regel innerhalb von maximal zwei aufeinander folgenden Semestern vermittelt werden können; in besonders begründeten Ausnahmefällen kann sich ein Modul auch über mehr als zwei Semester erstrecken. Für das künstlerische Kernfach im Bachelorstudium sind mindestens zwei Module verpflichtend, die etwa zwei Drittel der Arbeitszeit in Anspruch nehmen
können.\\

\textcolor{Bernd}{\textbf{\cite{RESO: SoSe2002-RL}} Modularisierung soll in Bachelor- und Masterstudiengängen sinnvoll angewandt werden.}\\

\emph{\textcolor{Brutus}{[ANM] Hierzu könnte in anderen Resolutionen, Positionspapieren und Protokollen von AKs mehr stehen.}}



\paragraph{(2)} Die Beschreibung eines Moduls soll mindestens enthalten:
\begin{enumerate}
\item Inhalte und Qualifikationsziele des Moduls,
\item Lehr- und Lernformen,
\item Voraussetzungen für die Teilnahme,
\item Verwendbarkeit des Moduls,
\item Voraussetzungen für die Vergabe von ECTS-Leistungspunkten
entsprechend dem European Credit Transfer System (ECTS-Leistungspunkte),
\item ECTS-Leistungspunkte und Benotung,
\item Häufigkeit des Angebots des Moduls,
\item Arbeitsaufwand und
\item Dauer des Moduls.
\end{enumerate}


\textcolor{Bernd}{\textbf{\cite{PROT: WiSe2015-RL}} \enquote{Sitzscheine} sollen vermieden werden, Anwesenheitspflicht nur in Ausnahmefällen.}\\

\textcolor{Bernd}{\textbf{\cite{RESO: WiSe2008-RL}} Die mit den ECTS-Punkten des jeweiligen Moduls gewichteten Modulabschlussnoten ergeben die Gesamtnote. Abweichungen von dieser Regelung möglich, wenn diese entsprechend begründet werden. Eine geringere Gewichtung der Module im ersten/zweiten Semester (Übergang Schule/Studium, unterschiedliches Niveau der Anfänger) sollte sich problemlos begründen lassen, ebenso eine stärkere Gewichtung der Abschlussarbeit}

\paragraph{(3)} Unter den Voraussetzungen für die Teilnahme sind die Kenntnisse, Fähigkeiten und Fertigkeiten für eine erfolgreiche Teilnahme und Hinweise für die geeignete Vorbereitung durch die Studierenden zu benennen. Im Rahmen der Verwendbarkeit des Moduls ist darzustellen, welcher Zusammenhang mit anderen Modulen desselben Studiengangs besteht und inwieweit es zum Einsatz in anderen Studiengängen geeignet ist. Bei den Voraussetzungen für die Vergabe von ECTS-Leistungspunkten ist anzugeben, wie ein Modul erfolgreich absolviert werden kann (Prüfungsart, -umfang, -dauer).\\

\textcolor{Bernd}{\textbf{\cite{PROT: WiSe2015-RL}} Zulassungsvoraussetzungen für Module sollen der Persönlichkeitsentwicklung und Schwerpunktsetzung der Studierenden nicht entgegenlaufen. (z.B. durch Vorgeben der Studienreihenfolge).}\\


\textcolor{Bernd}{\textbf{\cite{PROT: WiSe2015-RL}} Interne Voraussetzungen müssen  möglichst vorsichtig eingesetzt werden um die Flexibilität des Studienablaufs zu gewährleisten.}

\subsection{§ 8 Leistungspunktesystem}
\paragraph{(1)} Jedem Modul ist in Abhängigkeit vom Arbeitsaufwand für die Studierenden eine bestimmte Anzahl von ECTS-Leistungspunkten
zuzuordnen. Je Semester sind in der Regel 30 Leistungspunkte zu Grunde zu legen. Ein Leistungspunkt entspricht einer Gesamtarbeitsleistung der Studierenden im Präsenz- und Selbststudium von 25 bis höchstens 30 Zeitstunden. Für ein Modul werden ECTS-Leistungspunkte gewährt, wenn die in der Prüfungsordnung vorgesehenen Leistungen nachgewiesen werden. Die Vergabe von ECTS-Leistungspunkten setzt nicht zwingend eine Prüfung, sondern den erfolgreichen Abschluss des jeweiligen Moduls voraus.\\

\textcolor{Bernd}{\textbf{\cite{RESO: SoSe2002-RL}} Creditierung nach ECTS soll stattfinden.}\\

\textcolor{Bernd}{\textbf{\cite{RESO: SoSe2002-RL,PROT: WiSe2015-RL}} ECTS-Gewichtung und -Vergabe sollen realistisch sein und dem Arbeitsaufwand entsprechen.}\textcolor{Brutus}{\emph{Hier vermuten wir mehr Quellen.}}\\

\textcolor{Bernd}{Wenn möglich Gewichtung der ECTS zu Beginn des Studiums etwas geringer, am Ende etwas höher.}\textcolor{Brutus}{\emph{Hier vermuten wir richtige Quellen.}}\\

\textcolor{Bernd}{\textbf{\cite{RESO: WiSe2008-RL}} Die Verteilung von ECTS-Punkten soll regelmäßig durch Workload-Erhebungen überprüft werden, welche mit Konsequenzen verbunden sind.}\\

\textcolor{Bernd}{\textbf{\cite{RESO: SoSe2010-BaMa}} Es sollen wirksame Mechanismen zur Qualitätssicherung der Studiengänge und eine Instanz zur sinnvollen Zuordnung und zur Überprüfung des tatsächlichen Arbeitsaufwandes vorhanden sein.}

\paragraph{(2)}
Für den Bachelorabschluss sind nicht weniger als 180 ECTS-Leistungspunkte nachzuweisen. Für den Masterabschluss werden unter Einbeziehung des vorangehenden Studiums bis zum ersten berufsqualifizierenden Abschluss 300 ECTS-Leistungspunkte benötigt. Davon kann bei entsprechender Qualifikation der Studierenden im Einzelfall abgewichen werden, auch wenn nach Abschluss eines Masterstudiengangs 300 ECTS-Leistungspunkte nicht erreicht werden. Bei konsekutiven Bachelor und Masterstudiengängen in den künstlerischen Kernfächern an Kunst- und Musikhochschulen mit einer Gesamtregelstudienzeit von sechs Jahren wird das Masterniveau mit 360 ECTS-Leistungspunkten erreicht.\\

\textcolor{Bernd}{\textbf{\cite{RESO: SoSe2002-RL}} Der Bachelorstudiengang soll 180 CP und der Master 120 CP umfassen.}\\

\textcolor{Brutus}{\emph{[ANM] Bei 6 + 4 Semestern }}\\

\textcolor{Bernd}{\textbf{\cite{RESO: SoSe2010-BaMa}} Der Bachelor- und der Masterstudiengang sollen mit 180 respektive 120~CP abschließbar sein.}


\paragraph{(3)} Der Bearbeitungsumfang beträgt für die Bachelorarbeit 6 bis 12 ECTSLeistungspunkte und für die Masterarbeit 15 bis 30 ECTS-Leistungspunkte. In Studiengängen der Freien Kunst kann in begründeten Ausnahmefällen der Bearbeitungsumfang für die Bachelorarbeit bis zu 20 ECTS-Leistungspunkte und für die Masterarbeit bis zu 40 ECTS-Leistungspunkte betragen.\\

\textcolor{Bernd}{\textbf{\cite{RESO: SoSe2002-RL}} Es muss eine Bachelorarbeit (als Abschlussarbeit im Bachelor) existieren (hartes Kriterium) und diese sollte vom Umfang her 2-6 Monate betragen (weiches Kriterium).}\\ % Redundant mit § 4

\textcolor{Bernd}{\textbf{\cite{RESO: SoSe2002-RL}} Es muss eine Masterarbeit (als Abschlussarbeit im Master) existieren (hartes Kriterium) und diese sollte vom Umfang her mindestens 6 Monate betragen (hartes Kriterium), wenn möglich sogar mindestens 9 (weiches Kriterium).}\\ % Redundant mit § 4

\textcolor{Bernd}{\textbf{\cite{RESO: WiSe2008-RL}} Die Bachelorarbeit ist so in den Studienplan integriert, dass sie den Übergang in den Masterstudiengang (auch beim Hochschulwechsel) nicht unnötig erschwert. Problematisch sind hier vor allem Arbeiten, die erst spät im 6. Semester abgeschlossen werden können (Dauer von Korrekturen und Gutachen, Fristen für Master-Einschreibungen, ...). Es gibt eine Auswahlmöglichkeit an physikalischen Vertiefungs-/Spe\-zi\-a\-li\-si\-erung\-sveranstaltungen, welche auch mindestens im ECTS-Punkteumfang einer üblichen Veranstaltung angerechnet werden. Zudem sind ECTS-Punkte (wiederum mindestens im Rahmen einer üblichen Veranstaltung) verfügbar, in denen nicht physikalische Veranstaltungen angerechnet werden  können.  Diese  Anforderung  ist  recht  allgemein  gehalten,  da  die  Umsetzung  sehr  unterschiedlich  erfolgen kann. Denkbar ist z.B. ein „Wahlpflichtmodul“, in dem aus verschiedenen Vertiefungen ausgewählt werden kann in Kombination mit einem „Nebenfachmodul“ oder auch ein freier ECTS-Punktebereich, in dem beliebige Veranstaltungen angerechnet werden können.}\\

\paragraph{(4)} In begründeten Ausnahmefällen können für Studiengänge mit besonderen
studienorganisatorischen Maßnahmen bis zu 75 ECTS-Leistungspunkte pro Studienjahr zugrunde gelegt werden. Dabei ist die Arbeitsbelastung eines ECTS-Leistungspunktes mit 30 Stunden bemessen. Besondere studienorganisatorische Maßnahmen können insbesondere Lernumfeld und Betreuung, Studienstruktur, Studienplanung und Maßnahmen zur Sicherung des Lebensunterhalts betreffen.
\paragraph{(5)} Bei Lehramtsstudiengängen für Lehrämter der Grundschule oder Primarstufe, für übergreifende Lehrämter der Primarstufe und aller oder einzelner Schularten der Sekundarstufe, für Lehrämter für alle oder einzelne Schularten der Sekundarstufe I sowie für Sonderpädagogische Lehrämter I kann ein Masterabschluss vergeben werden, wenn nach mindestens 240 an der Hochschule erworbenen ECTS-Leistungspunkten unter Einbeziehung des Vorbereitungsdienstes insgesamt 300 ECTS-Leistungspunkte erreicht sind.
\paragraph{(6)} An Berufsakademien sind bei einer dreijährigen Ausbildungsdauer für den Bachelorabschluss in der Regel 180 ECTS-Leistungspunkte nachzuweisen. Der Umfang der theoriebasierten Ausbildungsanteile darf 120 ECTSLeistungspunkte, der Umfang der praxisbasierten Ausbildungsanteile 30 ECTS-Leistungspunkte nicht unterschreiten.
\subsection{§ 9 Besondere Kriterien für Kooperationen mit nichthochschulischen
Einrichtungen}
\paragraph{(1)} Umfang und Art bestehender Kooperationen mit Unternehmen und sonstigen Einrichtungen sind unter Einbezug nichthochschulischer Lernorte und Studienanteile sowie der Unterrichtssprache(n) vertraglich geregelt und auf der Internetseite der Hochschule beschrieben. Bei der Anwendung von Anrechnungsmodellen im Rahmen von studiengangsbezogenen Kooperationen ist die inhaltliche Gleichwertigkeit anzurechnender nichthochschulischer Qualifikationen und deren Äquivalenz gemäß dem angestrebten Qualifikationsniveau nachvollziehbar dargelegt.\\

\textcolor{Bernd}{\textbf{\cite{finde ich nicht; sollte SoSe18 oder früher sein}} Bei externen Abschlussarbeiten muss die Wissenschaftlichkeit durch Betreuung an der Hochschule gewährleistet werden.}\\

\textcolor{Brutus}{\emph{[ANM] Hier fehlt die Quelle, sollte SoSe18 oder früher sein}}\\

\textcolor{Bernd}{\textbf{\cite{PROT: WiSe2015-RL}} Anerkennung außerhalb der Hochschule erbrachter Leistungen}\\

\paragraph{(2)} Im Fall von studiengangsbezogenen Kooperationen mit nichthochschulischen Einrichtungen ist der Mehrwert für die künftigen Studierenden und die gradverleihende Hochschule nachvollziehbar dargelegt.
\subsection{§ 10 Sonderregelungen für Joint-Degree-Programme}
\paragraph{(1)} Ein Joint-Degree-Programm ist ein gestufter Studiengang, der von einer
inländischen Hochschule gemeinsam mit einer oder mehreren Hochschulen ausländischer Staaten aus dem Europäischen Hochschulraum koordiniert und angeboten wird, zu einem gemeinsamen Abschluss führt und folgende weitere Merkmale aufweist:
\begin{enumerate}
\item Integriertes Curriculum,
\item Studienanteil an einer oder mehreren ausländischen Hochschulen von in
der Regel mindestens 25 Prozent,
\item vertraglich geregelte Zusammenarbeit,
\item abgestimmtes Zugangs- und Prüfungswesen und
\item eine gemeinsame Qualitätssicherung.
\end{enumerate}
\paragraph{(2)} Qualifikationen und Studienzeiten werden in Übereinstimmung mit dem
Gesetz zu dem Übereinkommen vom 11. April 1997 über die Anerkennung von Qualifikationen im Hochschulbereich in der europäischen Region vom 16. Mai 2007 (BGBl. 2007 II S. 712, 713) (Lissabon-Konvention) anerkannt. Das ECTS wird entsprechend §§ 7 und 8 Absatz 1 angewendet und die Verteilung der Leistungspunkte ist geregelt. Für den Bachelorabschluss sind 180 bis 240 Leistungspunkte nachzuweisen und für den Masterabschluss nicht weniger als 60 Leistungspunkte. Die wesentlichen Studieninformationen sind veröffentlicht und für die Studierenden jederzeit zugänglich.\\

\textcolor{Bernd}{\textbf{\cite{RESO: SoSe2013-Anerkennung} davor \cite{RESO: WiSe2008-RL}} Die ZaPF fordert eine hohe Flexibilität bei der Anrechnung von Studienleistungen,
welche an anderen Universitäten, insbesondere im Ausland, erbracht wurden.}
\textcolor{Brutus}{\emph{Hier vermuten wir mehr Quellen.}}

\paragraph{(3)} Wird ein Joint Degree-Programm von einer inländischen Hochschule
gemeinsam mit einer oder mehreren Hochschulen ausländischer Staaten koordiniert und angeboten, die nicht dem Europäischen Hochschulraum angehören (außereuropäische Kooperationspartner), so finden auf Antrag der inländischen Hochschule die Absätze 1 und 2 entsprechende Anwendung, wenn sich die außereuropäischen Kooperationspartner in der
Kooperationsvereinbarung mit der inländischen Hochschule zu einer Akkreditierung unter Anwendung der in den Absätzen 1 und 2 sowie in den §§ 16 Absatz 1 und 33 Absatz 1 geregelten Kriterien und Verfahrensregeln verpflichtet.


\textcolor{Bernd}{\subsection{Zusätzliche Kriterien der ZaPF}
\paragraph{(1)}  \textbf{\cite{RESO: SoSe2010-BaMa}} Die Prüfungs- und Studienordnungen müssen transparent und eindeutig sein.}

\section{Teil 3 Fachlich-inhaltliche Kriterien für Studiengänge und
Qualitätsmanagementsysteme}
\subsection{§ 11 Qualifikationsziele und Abschlussniveau}
\paragraph{(1)} Die Qualifikationsziele und die angestrebten Lernergebnisse sind klar formuliert und tragen den in Artikel 2 Absatz 3 Nummer 1 Studienakkreditierungsstaatsvertrag genannten Zielen von Hochschulbildung nachvollziehbar Rechnung.\\

\textcolor{Bernd}{\textbf{\cite{PROT: SoSe2016-Komp}} Eine vollständige Liste aller Kompetenzen, welche ein Absolvent eines Bachelor/ Masterstudienganges Physik erfüllen soll, findet sich im AK Protokoll.}\\

\textcolor{Bernd}{\textbf{\cite{RESO: SoSe2002-RL}} Im Bachelorstudium werden Schlüsselqualifikationen angerechnet.}\\

Die Dimension Persönlichkeitsbildung umfasst auch die künftige zivilgesellschaftliche, politische und kulturelle Rolle der Absolventinnen und Absolventen.\\



Die Studierenden sollen nach ihrem Abschluss in der Lage sein, gesellschaftliche Prozesse kritisch, reflektiert sowie mit Verantwortungsbewusstsein und in demokratischem Gemeinsinn maßgeblich mitzugestalten.\\

\textcolor{Bernd}{\textbf{\cite{RESO: WiSe2017-Akkwesen}}
Die Befähigung zum zivilgesellschaftlichen Engagement muss als Studienziel
erhalten bleiben}\\

\textcolor{Bernd}{\textbf{\cite{RESO: WiSe2012-ZivEng}}
Eine Auseinandersetzung mit gesellschaftlich relevanten As\-pek\-ten der Physik soll in geeigneter Weise in der Lehre Berücksichtigung finden. Gesellschaftliches Engagement soll gefördert werden und Studierenden aus ihrem Engagement kein Nachteil entstehen.}\\


\textcolor{Bernd}{\textbf{\cite{POS: WiSe2016-Ethik}} Die ZaPF spricht sich dafür aus, Ethikinhalte in einem angemessen Umfang in das Physikstudium einzubinden, sodass die Möglichkeit geboten wird, sich auch im Rahmen des Studiums mit ethischen Fragenstellungen auseinanderzusetzen.}

\paragraph{(2)} Die fachlichen und wissenschaftlichen/künstlerischen Anforderungen umfassen die Aspekte Wissen und Verstehen (Wissensverbreiterung, Wissensvertiefung und Wissensverständnis), Einsatz, Anwendung und Erzeugung von Wissen/Kunst (Nutzung und Transfer, wissenschaftliche Innovation), Kommunikation und Kooperation sowie wissenschaftliches/künstlerisches Selbstverständnis / Professionalität und sind stimmig im Hinblick auf das vermittelte Abschlussniveau.\\

\textcolor{Bernd}{\textbf{\cite{RESO: SoSe2002-RL}} Die ZaPF erachtet es als hartes Akkreditierungskriterium, dass der Bachelor eine solide physikalische Grundausbildung darstellt.}\\

\textcolor{Bernd}{\textbf{\cite{PROT: SoSe2018-Akk}} Der Bachelor ist eine solide Physikausbildung und ermöglicht einen Übergang in die Wirtschaft.} % https://zapfev.de/reader/2018_SoSe_Heidelberg.pdf pp 107
\\

\textcolor{Bernd}{\textbf{\cite{POS: WiSe2018-WiKom}} Wissenschaftskommunikation sollte ein ele\-men\-tarer Bestand\-teil im Studium sein. Diese sollte mindestens als Wahlpflichtmodul vorkommen. Sinnvoll für die Umsetzung sind insbesondere ein Seminar und/oder eine Ringvorlesung.}\\


\paragraph{(3)} Bachelorstudiengänge dienen der Vermittlung wissenschaftlicher Grundlagen, Methodenkompetenz und berufsfeldbezogener Qualifikationen und stellen eine breite wissenschaftliche Qualifizierung sicher.\\

\textcolor{Bernd}{\textbf{\cite{PROT: WiSe2015-RL}} Studiengänge sollen nicht nur forschungsausgerichtet sein, sondern auch einen Übergang in die Wirtschaft ermöglichen.}\\

Konsekutive Masterstudiengänge sind als vertiefende, verbreiternde, fachübergreifende oder fachlich andere Studiengänge ausgestaltet.
Weiterbildende Masterstudiengänge setzen qualifizierte berufspraktische Erfahrung von in der Regel nicht unter einem Jahr voraus. Das Studiengangskonzept weiterbildender Masterstudiengänge berücksichtigt die beruflichen Erfahrungen und knüpft zur Erreichung der Qualifikationsziele an diese an. Bei der Konzeption legt die Hochschule den Zusammenhang von beruflicher Qualifikation und Studienangebot sowie die Gleichwertigkeit der Anforderungen zu konsekutiven Masterstudiengängen dar. Künstlerische Studiengänge fördern die Fähigkeit zur künstlerischen Gestaltung und entwickeln diese fort.

\textcolor{Bernd}{\textbf{\cite{RESO: SoSe2010-BaMa}} Das Masterstudium sollte mit einer einjährigen Forschungsphase abgeschlossen werden, die mit einem Umfang von 60 CP bemessen ist.}\\


\subsection{§ 12 Schlüssiges Studiengangskonzept und adäquate Umsetzung}
\paragraph{(1)} Das Curriculum ist unter Berücksichtigung der festgelegten Eingangsqualifikation und im Hinblick auf die Erreichbarkeit der Qualifikationsziele adäquat aufgebaut. Die Qualifikationsziele, die Studiengangsbezeichnung, Abschlussgrad und -bezeichnung und das Modulkonzept sind stimmig aufeinander bezogen. Das Studiengangskonzept umfasst vielfältige, an die jeweilige Fachkultur und das Studienformat angepasste Lehr- und Lernformen sowie gegebenenfalls Praxisanteile. \\

\textcolor{Bernd}{\textbf{\cite{RESO: SoSe2010-BaMa}} In der Experimentalphysik sollen im Bachelor mindestens folgende Inhalte vermittelt werden: Klassische Mechanik, Thermodynamik, Elektrodynamik, Optik, Quanten- / Atomphysik.\\
In der theoretischen Physik sollen im Bachelor mindestens die folgenden Inhalte vermittelt werden: Klassische Mechanik, Analytische Mechanik, Elektrodynamik, Spezielle Relativitätstheorie, Einführung in die Quantenmechanik, Thermodynamik.\\
Eine für die Bewältigung der Studieninhalte notwendige Vermittlung der entsprechenden Rechenmethoden soll rechtzeitig erfolgen und ggf. durch ein ergänzendes Modul gewährleistet werden.\\
Der Umfang der sollte insgesamt etwa 50-60 CP betragen, mit einer Gewichtung von 1:1 von Experiment und Theorie. Universitäten können selbst Schwerpunkte auf Theorie oder Experiment legen, wobei die Gewichtung nicht stärker als 2:1 sein sollte.\\
In der mathematischen Ausbildung sollten folgende Inhalte vermittelt werden: Analysis einer Veränderlichen, Analysis mehrerer Veränderlicher, zugeh\"orige Integrationstheorie, Lineare Algebra (elementare Matrixberechnungen bis Eigenwertprobleme), gewöhnliche Differentialgleichungen, Funktionentheorie, Operatorentheorie auf Hilberträumen.\\
Diese Inhalte sollten etwa 30 CP umfassen.\\
Weiterhin sollen grundlegende Kenntnisse im Experimentieren vermittelt werden. Der Bachelor sollte Versuche im Grundpraktikum von mindestens 12 CP und im Fortgeschrittenenpraktikum im Umfang von 6-8 CP enthalten. Ein Ziel der Praktika sollte das Erlernen eigenständigen Arbeitens sein. Dies kann z.B. realisiert werden durch die Integration eines Projektpraktikums, welches das Grundpraktikum zum Teil ersetzen könnte.\\
Die Inhalte von Festkörperphysik, Kern- und Elementarteilchenphysik, Atom- und Molekülphysik, H\"ohere Quantenmechanik und Statistische Physik sind wichtige Themen des Physikstudiums und es soll sichergestellt werden, dass diese Inhalte bis zum Masterabschluss gehört und eingebracht werden können.\\
Im Bachelor sollte es möglich sein, Qualifikationen im Umfang von etwa 10 CP wie z.B. Programmiersprachen, Elektronik oder wissenschaftliches Präsentieren zu erlernen und einzubringen. Außerdem sollte es Raum von 33-45 CP für einen physikalischen Wahlbereich geben, der ein breites Angebot an Seminaren und ersten Vertiefungsvorlesungen im Bachelor beinhaltet Weiterhin sollte Raum für ein verpflichtendes nichtphysikalisches Nebenfach geschaffen werden, welches einen Umfang von h\"ochstens 12 CP haben sollte. Für physiknahe Fächer können zusätzlich CP aus dem physikalischen Wahlbereich hinzugezogen werden.
}\\

\textcolor{Bernd}{\textbf{\cite{RESO: SoSe2010-BaMa}} Das Masterstudium sollte mit einer einjährigen Forschungsphase abgeschlossen werden, die mit einem Umfang von 60 CP bemessen ist.}\\

\textcolor{Bernd}{\textbf{\cite{RESO: WiSe2010-UebKon}}Die ZaPF sieht die folgenden Punkte als wichtige Elemente eines Übungsbetriebes an:
\begin{itemize}
\item Zu den Vorlesungen werden in der Regel Übungsaufgaben gestellt (insbesondere zu den Grundvorlesungen).
\item Sofern der Dozent die Aufgaben nicht selbst stellt, werden sie von ihm bestätigt.
\item Zu den Aufgaben werden Musterlösungen als Dokument zugänglich gemacht.
\item Jeder Studierende soll Möglichkeit haben seine bearbeiteten Aufgaben zur Korrektur abzugeben.
\item Die Teilnehmerzahl einer Übungsgruppe soll 15 nicht überschreiten.
\item Den Studierenden wird die Möglichkeit gegeben in den Übungsgruppen die gerechneten und neuen Aufgaben zu besprechen und Fragen zu klären.
\item Vorlesungsinhalte sollen in den Übungsgruppen wiederholt werden (z.B. durch Verständnisfragen, Kurzvorträge).
\item Die Dozenten, Aufgabensteller und Übungsguppenbetreuer einer Lehrveranstaltung treffen sich regelmäßig und halten Rücksprache. Die erste Anlaufstelle für inhaltliche Fragen eines Studierenden ist der Betreuer seiner Übungsgruppe.
\end{itemize}}

\textcolor{Bernd}{\textbf{\cite{POS: SoSe2016-Prog}} Die Zusammenkunft aller Physik-Fachschaften (ZaPF) empfiehlt den Hochschulen ein Kursangebot zur Vermittlung von Kompetenzen der wissenschaftlichen Programmierung. Es sollen folgende Kompetenzen den Studierenden, auch solchen ohne Vorkenntnissen, vermittelt werden:
\begin{itemize}
\item Benutzung grundlegender Werkzeuge zur Softwareentwicklung (Versionierung, Kompilierung, Editierung, u.ä.) - Programmierung in einer geeigneten Sprache
\item Abbildung Physikalischer Probleme auf den Computer
\item Verifizierung der eigenen Problemlösung durch geeignete Tests (z.B. Unit Tests)
\item Dokumentation fremder Tools und sinnvolle Nutzung von Bibliotheken, sowie nachvollziehbare Dokumentation eigener Projekte
\item Anwendung grundlegender Modelle und Prinzipien der Algorithmik (z.B. Einsatz von Rekursion im Gegensatz zu Schleifen)
\end{itemize}
}

\textcolor{Bernd}{\textbf{\cite{RESO: WiSe2017-BPrak}} Wir fordern die Fachbereiche auf, den Studierenden die Möglichkeit zu geben, nicht nur Forschungs- und Laborpraktika an der eigenen Universität belegen zu können, sondern auch wissenschaftsorientierte Praktika an anderen Universitäten, in Forschungseinrichtungen und insbesondere auch in der Industrie anrechnen lassen zu können. An vielen Universitäten bereitet das Physikstudium vorwiegend auf eine akademische Laufbahn vor. Hierbei haben Studierende jedoch kaum Gelegenheit sich während des regulären Studienverlaufes einen Einblick in mögliche Berufsfelder zu verschaffen.}

Es schafft geeignete Rahmenbedingungen zur Förderung der studentischen Mobilität, die den Studierenden einen Aufenthalt an anderen Hochschulen ohne Zeitverlust ermöglichen.\\

\textcolor{Bernd}{\textbf{\cite{RESO: SoSe2010-BaMa}} Auslandsaufenthalt im Bachelor soll unterstützt werden sein.\\ Um Auslandsaufenthalte zu unterstützen und Hochschulwechsel zu ermöglichen, sollen extern erbrachte Studienleistungen im Pflichtbereich des Bachelorstudiums im vollen Leistungspunktumfang auf inhaltlich ähnliche Module der eigenen Hochschule angerechnet und als Qualifikation für Folgemodule anerkannt werden. Bei einer Differenz in der Anzahl der Leistungspunkte wird ein kulantes Vorgehen befürwortet. Gibt es an der eigenen Hochschule kein äquivalentes Modul, so sollen die Leistungen in einem entsprechenden Wahlbereich angerechnet werden.}\\

\textcolor{Bernd}{\textbf{\cite{PROT: WiSe2015-RL}} Anerkennung außerhalb der Hochschule erbrachter Leistungen}\\

\textcolor{Bernd}{\textbf{\cite{RESO: SoSe2018-Mob}} Abbau von Mobilitätshürden in Zugangs- und Zulassungssatzungen. Aufmerksamkeit bei Konflikten durch Rückstufung und Leistungsbasierte Einstufung in Verbindung mit Studienhöchstdauern.}
\label{todo:BisHierhinLinksKontrolliert}

Es bezieht die Studierenden aktiv in die Gestaltung von Lehr- und Lernprozessen ein (studierendenzentriertes Lehren und Lernen) und eröffnet Freiräume für ein selbstgestaltetes Studium.\\

\textcolor{Bernd}{\textbf{\cite{RESO: SoSe2002-RL}} Die Schienprüfungen sollen nicht festgelegt sein. }\\ (\emph{\textcolor{Brutus}{[ANM] öhm, Scheine? So 2002... Das kann man mal aktuell formulieren. }})\\

\textcolor{Bernd}{\textbf{\cite{RESO: SoSe2002-RL}} Etwas Spezialisierung im Bachelor soll auch möglich sein (außerhalb der Thesis).}\\

\textcolor{Bernd}{Es soll eine Auswahlmöglichkeit an physikalischen Vertiefungen geben}\textcolor{Brutus}{\emph{Hier vermuten wir Quellen.}}\\

\textcolor{Bernd}{\textbf{\cite{RESO: SoSe2002-RL}} Ein nicht-physikalisches Nebenfach soll im Bachelor obglitatorisch sein.}\\

\textcolor{Bernd}{\textbf{\cite{RESO: SoSe2002-RL}} Eine Spezialisierung im Master in angemessener Tiefe soll außerhalb der Thesis soll möglich sein. Der Umfang dieser Spezialisierung soll 30~\% - 70~\% betragen.}\\

\textcolor{Bernd}{ Es kann eine Veranstaltung mit ECTS mit nicht-physikalischem Inhalt geben. Denkbar ist z.B. ein "Wahlpflichtmodul", in dem aus verschiedenen Vertiefungen ausgewählt werden kann in Kombination mit einem "Nebenfachmodul" oder auch ein freier ECTS-Punktebereich, in dem beliebige Veranstaltungen angerechnet werden können.}
\textcolor{Brutus}{\emph{Hier vermuten wir Quellen.}}\\

\textcolor{Bernd}{\textbf{\cite{PROT: WiSe2015-RL}} Es soll eine ausreichende Wahlfreiheit im Studium bestehen, es soll nicht zu verschult sein.}\\

\textcolor{Bernd}{\textbf{\cite{RESO: SoSe2010-BaMa}} Im Master sollte es einen Bereich von 60 CP geben, der sowohl vertiefende Spezialisierungsveranstaltungen als auch Veranstaltungen über bisher nicht behandelte physikalische Themen beinhaltet. Ein verpflichtender Anteil sollte ingesamt einen Umfang von 20 CP nicht übersteigen.}\\

\textcolor{Bernd}{\textbf{\cite{PROT: WiSe2015-RL}} Einbindung von Studierenden in die Studiengangsentwicklung}

\paragraph{(2)} Das Curriculum wird durch ausreichendes fachlich und methodischdidaktisch qualifiziertes Lehrpersonal umgesetzt. Die Verbindung von Forschung und Lehre wird entsprechend dem Profil der Hochschulart insbesondere durch hauptberuflich tätige Professorinnen und Professoren sowohl in grundständigen als auch weiterführenden Studiengängen gewährleistet. Die Hochschule ergreift geeignete Maßnahmen der Personalauswahl und -qualifizierung.\\

\textcolor{Bernd}{\textbf{\cite{PROT: WiSe2015-RL}} Es existieren Mechanismen zur Überholung/Wartung von Praktikumsversuchen und Qualifizierung von Tutoren, Weiterbildungsmöglichkeiten für Professoren.}

\paragraph{(3)} Der Studiengang verfügt darüber hinaus über eine angemessene Ressourcenausstattung (insbesondere nichtwissenschaftliches Personal, Raum- und Sachausstattung, einschließlich IT-Infrastruktur, Lehr- und Lernmittel).\\

\textcolor{Bernd}{\textbf{\cite{RESO: SoSe15-Eduroam}} Wir fordern die Einhaltung der eduroam Policy Service Definition, festgelegt
von der GÉANT Association, in der Version 2.8 vom Juli 2012}

\textcolor{Bernd}{\textbf{\cite{PROT: WiSe2015-RL}} Es existieren Mechanismen zur Überholung/Wartung von Praktikumsversuchen und Qualifizierung von Tutoren, Weiterbildungsmöglichkeiten für Professoren.}


\paragraph{(4)} Prüfungen und Prüfungsarten ermöglichen eine aussagekräftige Überprüfung der erreichten Lernergebnisse. Sie sind modulbezogen und kompetenzorientiert.\\


\textcolor{Bernd}{\textbf{\cite{RESO: SoSe2002-RL}} Prüfung sollten studienbegleitend gestaltet sein.}\\
\textcolor{Brutus}{\emph{[ANM] Zeitgemäßer wäre zu Schreiben, dass die Prüfungslast angemessen sein muss}}
\\

\textcolor{Bernd}{\textbf{\cite{PROT: WiSe2015-RL}} Die Prüfungsform soll dem Inhalt des Moduls angemessen sein.}\\

\textcolor{Bernd}{\textbf{\cite{RESO: SoSe2010-BaMa}} Die Prüfungs- und Studienordnungen müssen transparent und eindeutig sein.}\\

\textcolor{Bernd}{\textbf{\cite{RESO: SoSe2010-BaMa}} Schon frühzeitig im Bachelorstudium sollen abweichend von der Klausur als Prüfungsform auch andere Prüfungsformen angeboten werden. Insbesondere werden mündliche, möglicherweise modulübergreifende Prüfungen befürwortet, um vernetztes Lernen der Studierenden zu fördern.}\\

\textcolor{Bernd}{\textbf{\cite{RESO: WiSe2018-PAA}} Flexible Prüfungsabmeldungen erlauben}\\

\vspace{4em}

\paragraph{(5)} Die Studierbarkeit in der Regelstudienzeit ist gewährleistet. \\

\textcolor{Bernd}{\textbf{\cite{RESO: SoSe2002-RL}} Studierbarkeit verbindlich für Akkreditierungsverfahren}\\

Dies umfasst insbesondere
\begin{enumerate}
\item einen planbaren und verlässlichen Studienbetrieb,
\item die weitgehende Überschneidungsfreiheit von Lehrveranstaltungen und Prüfungen,\\

\textcolor{Bruno}{\textbf{\cite{POS: WiSe2018-RL}} Die \enquote{weitgehende} Überschneidungsfreiheit von Lehrveranstaltungen und Prüfungen ist äußerst kritikwürdig. Es ist absolut untragbar, Prüfungen, welche im selben Semester vorgesehen sind, nicht überschneidungsfrei anzubieten. Ein Studium in Regelstudienzeit ist in diesem Fall nicht möglich. Genauso erschweren Überschneidungen von Lehrveranstaltungen den Studienerfolg unverhältnismäßig; mindestens bei Pflichtveranstaltungen und häufig gewählten Kombinationen muss das Studium überschneidungsfrei möglich sein.}


\item einen plausiblen und der Prüfungsbelastung angemessenen
durchschnittlichen Arbeitsaufwand, wobei die Lernergebnisse eines
Moduls so zu bemessen sind, dass sie in der Regel innerhalb eines
Semesters oder eines Jahres erreicht werden können, was in
regelmäßigen Erhebungen validiert wird, und
\item eine adäquate und belastungsangemessene Prüfungsdichte und -organisation, wobei in der Regel für ein Modul nur eine Prüfung
vorgesehen wird und Module mindestens einen Umfang von fünf ECTS-Leistungspunkten aufweisen sollen.\\

\textcolor{Bernd}{\textbf{\cite{PROT: WiSe2015-RL}} Für mündliche Prüfungen soll es keinen Prüfungszeitraum, sondern persönliche Absprachen zwischen Prüfer und Prüfling geben.}\\

\textcolor{Bernd}{ Wiederholungsprüfungen sollen zeitnah nach der eigentlichen Prüfung angeboten werden.}\textcolor{Brutus}{\emph{Hier vermuten wir Quellen.}}\\

\textcolor{Bernd}{\textbf{\cite{RESO: SoSe2002-RL}} Regelungen zur Notenverbesserung von bereits bestandenen Prüfung (Freiversuch) sind wünschenswert.}\textcolor{Brutus}{\emph{Hier vermuten wir Weitere Quellen.}}\\

\textcolor{Brutus}{\emph{[ANM] stärkere Formulierung}}\\

\textcolor{Bernd}{\textbf{\cite{RESO: SoSe2002-RL}} Prüfungen sollen bei Nichtbestehen wiederholt werden dürfen.}

\end{enumerate}

\textcolor{Bernd}{\textbf{\cite{PROT: WiSe2015-RL}} Zulassungsvoraussetzungen für Module sollen der Persönlichkeitsentwicklung und Schwerpunktsetzung der Studierenden nicht entgegenlaufen. (z.B. durch Vorgeben der Studienreihenfolge).}\\

\textcolor{Bernd}{\textbf{\cite{RESO: SoSe2002-RL}} Defizite aus dem Vorstudium sollen im Master ausgeglichen werden.}\\

\textcolor{Bernd}{\textbf{\cite{PROT: SoSe2018-Akk}} Bachelor-Arbeiten sollen so konzipiert und in den Studienablauf integriert sein, dass man auf jeden Fall zum Master fristgerecht die Hochschule wechseln kann.}\\ % https://zapfev.de/reader/2018_SoSe_Heidelberg.pdf pp 108

%\textcolor{Bernd}{\textbf{\cite{RESO: WiSe2018-VorVer}} Abschlussarbeiten sollen zur Vermeidung von Ausbeutung Mindestkriterien erfüllen:
%\begin{itemize}
%\item klares Zeitfenster (unabhängig von Anzahl der Module, welche Arbeit umfasst), sofortiges Anmelden nach Themenvergabe
%\item klar definierte Prüfungsleistung [4]
%\item schriftlichen Projektplan [8]
%\item Es sollte präzise genug sein um ein Ziel zu haben, aber vage genug, um kleine Probleme ab zu fangen.
%Änderungen/Entscheidungen einvernehmlich (BackUp Plan)
%\item Keine Einarbeitungsphasen, die nicht Bestandteil des Studiengangs sind (Gleichbehandlung, Vergleichbarkeit)\\
%\item thematische Trennung zu SHK/HiWi
%\item Einbettung in Studienalltag
%\end{itemize}
%}


\paragraph{(6)} Studiengänge mit besonderem Profilanspruch weisen ein in sich geschlossenes Studiengangskonzept aus, das die besonderen Charakteristika des Profils angemessen darstellt.

\subsection{§ 13 Fachlich-Inhaltliche Gestaltung der Studiengänge}
\paragraph{(1)} Die Aktualität und Adäquanz der fachlichen und wissenschaftlichen Anforderungen ist gewährleistet. Die fachlich-inhaltliche Gestaltung und die methodisch-didaktischen Ansätze des Curriculums werden kontinuierlich überprüft und an fachliche und didaktische Weiterentwicklungen angepasst.
Dazu erfolgt eine systematische Berücksichtigung des fachlichen Diskurses auf nationaler und gegebenenfalls internationaler Ebene.\\

\textcolor{Bernd}{\textbf{\cite{PROT: SoSe2018-Akk}} Lehrevaluationen müssen Konsequenzenn haben, es soll sinnvolle Mechanismen zur Reaktion geben.}\\ % https://zapfev.de/reader/2018_SoSe_Heidelberg.pdf pp 108


\textcolor{Bernd}{\textbf{\cite{PROT: WiSe2015-RL}} Evaluation von Lehrveranstaltungen mit Rückkopplung an die Lehrenden}\\

\textcolor{Bernd}{\textbf{\cite{RESO: SoSe2013-SysAkk}} Weitere kriterien und Ablaufsbeispiel zur Lehrevaluation}


\paragraph{(2)} In Studiengängen, in denen die Bildungsvoraussetzungen für ein Lehramt vermittelt werden, sind Grundlage der Akkreditierung sowohl die Bewertung der Bildungswissenschaften und Fachwissenschaften sowie deren Didaktik
nach ländergemeinsamen und länderspezifischen fachlichen Anforderungen als auch die ländergemeinsamen und länderspezifischen strukturellen Vorgaben für die Lehrerausbildung.
\paragraph{(3)} Im Rahmen der Akkreditierung von Lehramtsstudiengängen ist insbesondere zu prüfen, ob
\begin{enumerate}
\item ein integratives Studium an Universitäten oder gleichgestellten
Hochschulen von mindestens zwei Fachwissenschaften und von Bildungswissenschaften in der Bachelorphase sowie in der Masterphase (Ausnahmen sind bei den Fächern Kunst und Musik zulässig),
\item schulpraktische Studien bereits während des Bachelorstudiums und
\item eine Differenzierung des Studiums und der Abschlüsse nach Lehrämtern erfolgt sind. Ausnahmen beim Lehramt für die beruflichen Schulen sind zulässig.
\end{enumerate}

\textcolor{Bernd}{\textbf{\cite{RESO: WiSe2008-Lehramt}} Die Lehramtsstudierenden im Fach Physik sollen in allen Bereichen auf sie zugeschnittene Veranstaltungen erhalten.Physikvorlesungen für Fachfremde sind hierfür kein Ersatz! Die Bereiche umfassen in der theoretischen Physik für die Sekundarstufen mindestens die klassische Mechanik, die Elektrodynamik und die Quantenmechanik. Die Vermittlung der grundlegenden mathematischen Fertigkeiten ist für alle zu gewährleisten.}

\subsection{§ 14 Studienerfolg}
Der Studiengang unterliegt unter Beteiligung von Studierenden und Absolventinnen und Absolventen einem kontinuierlichen Monitoring. Auf dieser Grundlage werden Maßnahmen zur Sicherung des Studienerfolgs abgeleitet. Diese werden fortlaufend überprüft und die Ergebnisse für die Weiterentwicklung des Studiengangs genutzt. Die Beteiligten werden über die Ergebnisse und die ergriffenen Maßnahmen unter Beachtung datenschutzrechtlicher Belange informiert.

\textcolor{Bernd}{\textbf{\cite{RESO: SoSe2010-BaMa} davor \cite{RESO: WiSe2008}} Es sollen wirksame Mechanismen zur Qualitätssicherung der Studiengänge und eine Instanz zur sinnvollen Zuordnung und zur Überprüfung des tatsächlichen Arbeitsaufwandes vorhanden sein.}\textcolor{Brutus}{\emph{Es Existiere eine Resolution aus dem Wintersemester 2008}}\\

\textcolor{Bernd}{\textbf{\cite{PROT: WiSe2015-RL}} Der Verbleib der von Absolventen soll in Erfahrung gebracht werden und bei der Weiterentwicklung des Studiengangs berücksichtigt werden.}\\

\textcolor{Bernd}{Studierende sollen auch bei der Ausarbeitung der Konsquenzen beteiligt sein, s. Stellungnahme Pool}
\textcolor{Brutus}{\emph{hier fehlt die Referenz zur Poolstellungnahme}}

\subsection{§ 15 Geschlechtergerechtigkeit und Nachteilsausgleich}
Die Hochschule verfügt über Konzepte zur Geschlechtergerechtigkeit und zur Förderung der Chancengleichheit von Studierenden in besonderen Lebenslagen, die auf der Ebene des Studiengangs umgesetzt werden.\\

\textcolor{Bernd}{\textbf{\cite{PROT: WiSe2015-RL}} Benennung von Studierendenberatenden}\\

\textcolor{Bernd}{\textbf{\cite{PROT: WiSe2015-RL}} Praktikumslabore sollen möglichst barrierefrei sein, ggf. müssen Ersatzversuche angegeben werden.}\\

\subsection{§ 16 Sonderregelungen für Joint-Degree-Programme}
\paragraph{(1)} Für Joint-Degree-Programme finden die Regelungen in § 11 Absätze 1
und 2, sowie § 12 Absatz 1 Sätze 1 bis 3, Absatz 2 Satz 1, Absätze 3 und 4
sowie § 14 entsprechend Anwendung. Daneben gilt:
\begin{enumerate}
\item Die Zugangsanforderungen und Auswahlverfahren sind der Niveaustufe und der Fachdisziplin, in der der Studiengang angesiedelt ist, angemessen.
\item Es kann nachgewiesen werden, dass mit dem Studiengang die angestrebten Lernergebnisse erreicht werden.
\item Soweit einschlägig, sind die Vorgaben der Richtlinie 2005/36/EG vom 07.09.2005 (ABl. L 255 vom 30.9.2005, S. 22-142) über die Anerkennung von Berufsqualifikationen, zuletzt geändert durch die Richtlinie 2013/55/EU vom 17.01.2014 (ABl. L 354 vom 28.12.2013, S. 132-170) berücksichtigt.
\item Bei der Betreuung, der Gestaltung des Studiengangs und den angewendeten Lehr- und Lernformen werden die Vielfalt der Studierenden und ihrer Bedürfnisse respektiert und die spezifischen Anforderungen mobiler Studierender berücksichtigt.
\item  Das Qualitätsmanagementsystem der Hochschule gewährleistet die
Umsetzung der vorstehenden und der in § 17 genannten Maßgaben.
\end{enumerate}
\paragraph{(2)} Wird ein Joint Degree-Programm von einer inländischen Hochschule gemeinsam mit einer oder mehreren Hochschulen ausländischer Staaten koordiniert und angeboten, die nicht dem Europäischen Hochschulraum angehören (außereuropäische Kooperationspartner), so findet auf Antrag der inländischen Hochschule Absatz 1 entsprechende Anwendung, wenn sich die außereuropäischen Kooperationspartner in der Kooperationsvereinbarung mit der inländischen Hochschule zu einer Akkreditierung unter Anwendung der in Absatz 1, sowie der in den §§ 10 Absätze 1 und 2 und 33 Absatz 1 geregelten Kriterien und Verfahrensregeln verpflichtet.
\subsection{§ 17 Konzept des Qualitätsmanagementsystems (Ziele, Prozesse,
Instrumente)}
\paragraph{(1)} Die Hochschule verfügt über ein Leitbild für die Lehre, das sich in den Curricula ihrer Studiengänge widerspiegelt. Das Qualitätsmanagementsystem folgt den Werten und Normen des Leitbildes für die Lehre und zielt darauf ab, die Studienqualität kontinuierlich zu verbessern. Es gewährleistet die systematische Umsetzung der in Teil 2 und 3 genannten Maßgaben. Die Hochschule hat Entscheidungsprozesse, Zuständigkeiten und Verantwortlichkeiten für die Einrichtung, Überprüfung, Weiterentwicklung und Einstellung von Studiengängen und die hochschuleigenen Verfahren zur Akkreditierung von Studiengängen im Rahmen ihres Qualitätsmanagementsystems festgelegt und hochschulweit veröffentlicht.
\paragraph{2} Das Qualitätsmanagementsystem wurde unter Beteiligung der Mitgliedsgruppen der Hochschule und unter Einbeziehung externen Sachverstands erstellt. Es stellt die Unabhängigkeit von Qualitätsbewertungen sicher und enthält Verfahren zum Umgang mit
hochschulinternen Konflikten sowie ein internes Beschwerdesystem. Es beruht auf geschlossenen Regelkreisen, umfasst alle Leistungsbereiche der Hochschule, die für Studium und Lehre unmittelbar relevant sind und verfügt über eine angemessene und nachhaltige Ressourcenausstattung.Funktionsfähigkeit und Wirksamkeit mit Bezug auf die Studienqualität werden von der Hochschule regelmäßig überprüft und kontinuierlich weiterentwickelt.
\textcolor{Bernd}{\textbf{\cite{RESO: SoSe2013-SysAkk}}
Das Qualitätsmanagementsystem wird durch eine zentrale hochschulweite Kommission sowie mehrere fachnahe Kommissionen gesteuert. Dabei muss die zentrale fakultätenübergreifende Kommission in ihrer Entscheidungshoheit uneingeschränkt sein. Insbesondere darf kein Abhängigkeitsverhältnis zur Hochschulleitung bestehen.
Es existiert ein hochschulinternes Beschwerdemanagement.}



\subsection{§ 18 Maßnahmen zur Umsetzung des Qualitätsmanagementkonzepts}
\paragraph{(1)} Das Qualitätsmanagementsystem beinhaltet regelmäßige Bewertungen der Studiengänge und der für Lehre und Studium relevanten Leistungsbereiche durch interne und externe Studierende, hochschulexterne wissenschaftliche Expertinnen und Experten, Vertreterinnen und Vertreter der Berufspraxis, Absolventinnen und Absolventen.
Zeigt sich dabei Handlungsbedarf, werden die erforderlichen Maßnahmen ergriffen und umgesetzt.\\

\textcolor{Bernd}{\textbf{\cite{RESO: SoSe2013-SysAkk}}Um eine Qualitätskontrolle sicher zu stellen sind die Programme regelmäßig zu evaluieren.}\\

\textcolor{Bernd}{\textbf{\cite{RESO: SoSe2010-BaMa}} Es sollen wirksame Mechanismen zur Qualitätssicherung der Studiengänge und eine Instanz zur sinnvollen Zuordnung und zur Überprüfung des tatsächlichen Arbeitsaufwandes vorhanden sein.}

\paragraph{(2)} Sofern auf der Grundlage des Qualitätsmanagementsystems der Hochschule auch Bewertungen von Lehramtsstudiengängen, Lehramtsstudiengängen mit dem Kombinationsfach Evangelische oder Katholische Theologie/Religion, evangelisch-theologischen Studiengängen, die für das Pfarramt qualifizieren, und anderen Bachelor- und Masterstudiengängen mit dem Kombinationsfach Evangelische oder Katholische Theologie vorgenommen werden, gelten die Mitwirkungs- und Zustimmungserfordernisse gemäß § 25 Absatz 1 Sätze 3 bis 5 entsprechend.
\paragraph{(3)} Die für die Umsetzung des Qualitätsmanagmentsystems erforderlichen Daten werden hochschulweit und regelmäßig erhoben.
\paragraph{(4)} Die Hochschule dokumentiert die Bewertung der Studiengänge des hochschulinternen Qualitätsmanagementsystems unter Einschluss der Voten der externen Beteiligten und informiert Hochschulmitglieder, Öffentlichkeit, Träger und Sitzland regelmäßig über die ergriffenen Maßnahmen. Sie informiert die Öffentlichkeit über die auf der Grundlage des hochschulinternen Verfahrens erfolgten  Akkreditierungsentscheidungen und stellt dem Akkreditierungsrat die zur Veröffentlichung nach § 29 erforderlichen Informationen zur Verfügung.
\subsection{§ 19 Kooperationen mit nichthochschulischen Einrichtungen}
Führt eine Hochschule einen Studiengang in Kooperation mit einer nichthochschulischen Einrichtung durch, ist die Hochschule für die Einhaltung der Maßgaben gemäß der Teile 2 und 3 verantwortlich. Die gradverleihende Hochschule darf Entscheidungen über Inhalt und Organisation des Curriculums, über Zulassung, Anerkennung und Anrechnung, über die Aufgabenstellung und Bewertung von Prüfungsleistungen, über die Verwaltung von Prüfungs- und Studierendendaten, über die Verfahren der Qualitätssicherung sowie über Kriterien und Verfahren der Auswahl des Lehrpersonals nicht delegieren.
\subsection{§ 20 Hochschulische Kooperationen}
\label{todo:AbHierLinksKontrolliert}
\paragraph{(1)} Führt eine Hochschule eine studiengangsbezogene Kooperation mit einer anderen Hochschule durch, gewährleistet die gradverleihende Hochschule bzw. gewährleisten die gradverleihenden Hochschulen die Umsetzung und die Qualität des Studiengangskonzeptes. Art und Umfang der Kooperation sind beschrieben und die der Kooperation zu Grunde liegenden Vereinbarungen dokumentiert.
\paragraph{(2)} Führt eine systemakkreditierte Hochschule eine studiengangsbezogene Kooperation mit einer anderen Hochschule durch, kann die systemakkreditierte Hochschule dem Studiengang das Siegel des Akkreditierungsrates gemäß § 22 Absatz 4 Satz 2 verleihen, sofern sie selbst gradverleihend ist und die Umsetzung und die Qualität des
Studiengangskonzeptes gewährleistet. Abs. 1 Satz 2 gilt entsprechend.
\paragraph{(3)} Im Fall der Kooperation von Hochschulen auf der Ebene ihrer Qualitätsmanagementsysteme ist eine Systemakkreditierung jeder der beteiligten Hochschulen erforderlich. Auf Antrag der kooperierenden Hochschulen ist ein gemeinsames Verfahren der Systemakkreditierung zulässig.
\subsection{§ 21 Besondere Kriterien für Bachelorausbildungsgänge an
Berufsakademien}
\paragraph{(1)} Die hauptberuflichen Lehrkräfte an Berufsakademien müssen die Einstellungsvoraussetzungen für Professorinnen und Professoren an Fachhochschulen gemäß § 44 Hochschulrahmengesetz in der Fassung der Bekanntmachung vom 19. Januar 1999 (BGBl. I S. 18), das zuletzt durch Artikel 6 Absatz 2 des Gesetzes vom 23. Mai 2017 (BGBl. I S. 1228) geändert worden ist, erfüllen. Soweit Lehrangebote überwiegend der Vermittlung praktischer Fertigkeiten und Kenntnisse dienen, für die nicht die Einstellungsvoraussetzungen für Professorinnen oder Professoren an Fachhochschulen erforderlich sind, können diese entsprechend § 56 Hochschulrahmengesetz und einschlägigem Landesrecht hauptberuflich tätigen Lehrkräften für besondere Aufgaben übertragen werden. Der Anteil der Lehre, der von hauptberuflichen Lehrkräften erbracht wird, soll 40 Prozent nicht unterschreiten. Im Ausnahmefall gehören dazu auch Professorinnen oder Professoren an Fachhochschulen oder Universitäten, die in Nebentätigkeit an einer Berufsakademie lehren, wenn auch durch sie die Kontinuität im Lehrangebot und die Konsistenz der Gesamtausbildung sowie verpflichtend die Betreuung und Beratung der Studierenden gewährleistet sind; das Vorliegen dieser Voraussetzungen ist im Rahmen der Akkreditierung des einzelnen Studiengangs gesondert festzustellen.
\paragraph{(2)} Absatz 1 Satz 1 gilt entsprechend für nebenberufliche Lehrkräfte, die
theoriebasierte, zu ECTS-Leistungspunkten führende Lehrveranstaltungen anbieten oder die als Prüferinnen oder Prüfer an der Ausgabe und Bewertung der Bachelorarbeit mitwirken.
Lehrveranstaltungen nach Satz 1 können ausnahmsweise auch von nebenberuflichen Lehrkräften angeboten werden, die über einen fachlich einschlägigen Hochschulabschluss oder einen gleichwertigen Abschluss sowie über eine fachwissenschaftliche und didaktische Befähigung und über eine mehrjährige fachlich einschlägige Berufserfahrung entsprechend den Anforderungen an die Lehrveranstaltung verfügen.
\paragraph{(3)} Im Rahmen der Akkreditierung ist auch zu überprüfen:
\begin{enumerate}
\item das Zusammenwirken der unterschiedlichen Lernorte (Studienakademie und Betrieb),
\item die Sicherung von Qualität und Kontinuität im Lehrangebot und in der Betreuung und Beratung der Studierenden vor dem Hintergrund der
besonderen Personalstruktur an Berufsakademien und
\item das Bestehen eines nachhaltigen Qualitätsmanagementsystems, das die unterschiedlichen Lernorte umfasst.
\end{enumerate}





\section{Teil 4 Verfahrensregeln für die Programm- und Systemakkreditierung}


\subsection{§ 22 Entscheidung des Akkreditierungsrates; Verleihung des Siegels}
\paragraph{(1)} Der Akkreditierungsrat entscheidet auf Antrag der Hochschule über die Akkreditierung durch die Feststellung der Einhaltung der formalen Kriterien und der fachlich-inhaltlichen Kriterien gemäß Artikel 3 Absatz 5 Satz 1 Studienakkreditierungsstaatsvertrag in Verbindung mit Teil 2 und Teil 3 dieser Rechtsverordnung. Grundlage für die Entscheidung über die formalen Kriterien ist ein Prüfbericht gemäß Artikel 4 Absatz 3 Satz 1 Nummer 2 Buchstabe b Studienakkreditierungsstaatsvertrag. Grundlage für die Entscheidung über die fachlich-inhaltlichen Kriterien ist ein Gutachten gemäß Artikel 3 Absatz 2 Satz 1 Nummer 4 Studienakkreditierungsstaatsvertrag.
\paragraph{(2)} Die Entscheidung ergeht durch schriftlichen Bescheid. Sie ist zu begründen.
\paragraph{(3)} Die Hochschule erhält vor der Entscheidung des Akkreditierungsrates Gelegenheit zur Stellungnahme, wenn er von der Empfehlung der Gutachterinnen und Gutachter in erheblichem Umfang abzuweichen beabsichtigt. Die Frist zur Stellungnahme beträgt einen Monat.
\paragraph{(4)} Mit der Akkreditierung verleiht der Akkreditierungsrat dem Studiengang oder dem Qualitätsmanagementsystem sein Siegel. Bei einer Systemakkreditierung erhält die Hochschule das Recht, das Siegel des Akkreditierungsrates für die von ihr geprüften Studiengänge selbst zu verleihen.
\paragraph{(5)} Die Akkreditierung von katholisch-theologischen Studiengängen, die für das Priesteramt und den Beruf der Pastoralreferentin oder des Pastoralreferenten qualifizieren (\enquote{Theologisches Vollstudium}), erfolgt ausschließlich in Form der Programmakkreditierung. Die Entscheidung des Akkreditierungsrates bedarf in volltheologischen und teiltheologischen Studiengängen der Zustimmung der zuständigen kirchlichen Stellen.
\subsection{§ 23 Vorzulegende Unterlagen}
\paragraph{(1)} Dem Antrag sind folgende Unterlagen beizufügen:
\begin{enumerate}
\item Selbstbericht der Hochschule,
\item ein Akkreditierungsbericht einer beim Akkreditierungsrat zugelassenen Agentur, der aus einem Prüfbericht und einem Gutachten besteht; im Fall der Systemakkreditierung bezieht sich der Prüfbericht auf die Nachweise gemäß Nummern 3 und 4,
\item bei Antrag auf Systemakkreditierung zusätzlich der Nachweis, dass mindestens ein Studiengang das Qualitätsmanagementsystem
durchlaufen hat,
\item bei Antrag auf Systemreakkreditierung der Nachweis, dass grundsätzlich alle Bachelor- und Masterstudiengänge das Qualitätsmanagementsystem mindestens einmal durchlaufen haben.
\end{enumerate}
\paragraph{(2)} Von den Unterlagen nach Absatz 1 Nummer 2 sind, soweit sie nicht in deutscher Sprache verfasst sind, Übersetzungen in deutscher Sprache vorzulegen.
\paragraph{(3)} Sobald der Akkreditierungsrat ein elektronisches Datenverarbeitungssystem zur Verfügung stellt, ist dieses zu nutzen.
\subsection{§ 24 Beauftragung einer Agentur; Akkreditierungsgutachten; Begehung}
\paragraph{(1)} Die Hochschule beauftragt eine beim Akkreditierungsrat gemäß Artikel 5 Absatz 3 Satz 1 Nummer 5 Studienakkreditierungsstaatsvertrag zugelassene Agentur mit der Begutachtung der formalen und fachlich-inhaltlichen Kriterien und der Erstellung eines Akkreditierungsberichts. Für katholisch-theologische Studiengänge, die für das Priesteramt und den Beruf der Pastoralreferentin oder des Pastoralreferenten qualifizieren (\enquote{Theologisches Vollstudium}), erfolgt die Begutachtung durch die Agentur für Qualitätssicherung und Akkreditierung kanonischer Studiengänge in Deutschland, die durch den Akkreditierungsrat zugelassen ist.
\paragraph{(2)} Die Hochschule stellt der Agentur einen Selbstbericht zur Verfügung, der mindestens Angaben zu den Qualitätszielen der Hochschule und zu den formalen und fachlich-inhaltlichen Kriterien nach Teil 2 und Teil 3 enthält. Der Selbstbericht der Hochschule, an dessen Erstellung die Studierendenvertretung zu beteiligen ist, soll für die Programmakkreditierung 20 Seiten und für die System- und Bündelakkreditierung 50 Seiten nicht überschreiten.
\paragraph{(3)} Der Prüfbericht wird von der Agentur erstellt; bei Studiengängen nach § 25 Absatz 1 Sätze 3 und 4 bedarf der Prüfbericht der Zustimmung der dort jeweils benannten Personen. Maßgebliche Standards für den Prüfbericht sind die formalen Kriterien nach Teil 2. Er enthält einen Vorschlag zur Feststellung
der Einhaltung der formalen Kriterien. Der Prüfbericht ist in dem durch den Akkreditierungsrat vorzugebenden Raster abzufassen. Über die Nichterfüllung eines formalen Kriteriums ist die Hochschule unverzüglich zu informieren.
\paragraph{(4)} Das Gutachten wird vom Gutachtergremium nach § 25 abgegeben. Das Gutachtergremium erhält den Prüfbericht nach Absatz 3.Maßgebliche Standards für das Gutachten sind die fachlich-inhaltlichen Kriterien nach Teil 3. Es enthält einen Vorschlag zur Feststellung der Einhaltung der fachlich-inhaltlichen Kriterien. Das Gutachten ist in dem durch den Akkreditierungsrat vorzugebenden Raster abzufassen und soll für die Programmakkreditierung 20 Seiten und für die System- und Bündelakkreditierung 100 Seiten nicht überschreiten.
\paragraph{(5)} Im Rahmen der Begutachtung der fachlich-inhaltlichen Kriterien findet eine Begehung durch das Gutachtergremium statt. \\

\textcolor{Bernd}{\textbf{\cite{PROT: WiSe2015-RL}} Tutor*innen, insbesondere in Praktika, sollen bei Begehungen im Gespräch mit den Lehrenden dabei sein.}\\

Bei der Akkreditierung eines Studiengangs, der zum Zeitpunkt der Beauftragung der Agentur noch nicht angeboten wird (Konzeptakkreditierung), kann das Gutachtergremium einvernehmlich auf eine Begehung verzichten. Gleiches gilt bei der Reakkreditierung eines Studiengangs.
\subsection{§ 25 Zusammensetzung des Gutachtergremiums;
Anforderungen an die Gutachterinnen und Gutachter}
\paragraph{(1)} Dem Gutachtergremium der Agenturen gehören bei einer Programmakkreditierung mindestens vier Personen an. Es setzt sich wie folgt zusammen:
\begin{enumerate}
\item mindestens zwei fachlich nahestehende Hochschullehrerinnen oder
Hochschullehrer,
\item eine fachlich nahestehende Vertreterin oder ein fachlich nahestehender
Vertreter aus der beruflichen Praxis,
\item eine fachlich nahestehende Studierende oder ein fachlich nahestehender Studierender.
\end{enumerate}

Bei der Akkreditierung von Studiengängen, die die Befähigung für die
Aufnahme in den Vorbereitungsdienst für ein Lehramt vermitteln, tritt eine Vertreterin oder ein Vertreter der für das Schulwesen zuständigen Obersten Landesbehörde an die Stelle der Person nach Nummer 2; bei Lehramtsstudiengängen mit dem Kombinationsfach Evangelische oder Katholische Theologie/Religion tritt zusätzlich eine Vertreterin oder ein Vertreter der örtlich zuständigen Diözese oder Landeskirche hinzu. Bei der Akkreditierung von theologischen Studiengängen, die für das Pfarramt, das Priesteramt und den Beruf der Pastoralreferentin oder des Pastoralreferenten qualifizieren (\enquote{Theologisches Vollstudium}) und in allen anderen Bachelor- und Masterstudiengängen mit dem Kombinationsfach Evangelische oder Katholische Theologie/Religion tritt an die Stelle der Person nach Nummer 2 eine Vertreterin oder ein Vertreter der zuständigen kirchlichen Stelle. Für die in den Sätzen 3 und 4 genannten Studiengänge bedarf die Abgabe des Gutachtens gemäß § 24 Absatz 4 Satz 1 der Zustimmung der jeweils genannten Personen; ohne diese Zustimmung erfolgt keine Vorlage des Gutachtens an den Akkreditierungsrat.\\

\textcolor{Bernd}{\textbf{\cite{RESO: SoSe2018-RV}} Bei Akkreditierungen von Lehramtsstudiengängen darf die Vertretung der Berufspraxis in der Gutachtergruppe nicht durch einen Vertreter oder eine Vertreterin der obersten Landesbehörde ersetzt werden, sondern soll um diese ergänzt werden. (Zuvor bereits aufgegriffen in \textbf{\cite{POS: WiSe2017-AkkRL}})}

\paragraph{(2)} Dem Gutachtergremium der Agenturen gehören bei einer Systemakkreditierung mindestens fünf Personen an. Es setzt sich wie folgt zusammen:
\begin{enumerate}
\item mindestens drei Hochschullehrerinnen oder Hochschullehrer mit
einschlägiger Erfahrung in der Qualitätssicherung im Bereich Lehre,
\item eine Vertreterin oder ein Vertreter aus der beruflichen Praxis,
\item eine Studierende oder ein Studierender.
\end{enumerate}
\paragraph{(3)} Die Hochschullehrerinnen und Hochschullehrer verfügen über die Mehrheit der Stimmen. In dem jeweiligen Gutachtergremium muss die Mehrzahl der Gutachterinnen oder Gutachter über Erfahrungen mit Akkreditierungen verfügen. Bei einer Systemakkreditierung muss die Mehrzahl der Gutachterinnen und Gutachter über Erfahrungen mit Systemakkreditierungen verfügen.\\

\textcolor{Bernd}{\textbf{\cite{RESO: SoSe2018-RV}} Alle Gutachter*innen sollen im Bereich Akkreditierung geschult sein -- entweder durch ihre Erfahrung oder durch entsprechende Fortbildungsmaßnahmen.}

\paragraph{(4)} Die Gutachterinnen und Gutachter werden von der mit der Erstellung des Akkreditierungsberichts beauftragten Agentur benannt.Die Agentur ist bei der Bestellung an das von der Hochschulrektorenkonferenz zu entwickelnde Verfahren gemäß Artikel 3  Absatz 3 Satz 3 Studienakkreditierungsstaatsvertrag gebunden.
\paragraph{(5)} Als Gutachter ist ausgeschlossen, wer
\begin{enumerate}
\item an der Hochschule, die den Antrag auf Akkreditierung stellt, tätig oder eingeschrieben ist,
\item bei Kooperationsstudiengängen oder Joint-Degree-Programmen an einer der an dem Studiengang beteiligten Hochschulen tätig oder eingeschrieben ist oder
\item nach in der Wissenschaft üblichen Regeln als befangen gilt.
\end{enumerate}
\paragraph{(6)} Die Agentur teilt der Hochschule vor der Benennung der Gutachterinnen und Gutachter die personelle Zusammensetzung des Gutachtergremiums mit. Die Hochschule hat ein Recht zur Stellungnahme innerhalb einer Frist von zwei Wochen.
\subsection{§ 26 Geltungszeitraum der Akkreditierung; Verlängerung}
\paragraph{(1)} Die erstmalige Akkreditierung ist für den Zeitraum von acht Jahren ab Beginn des Semesters oder Trimesters gültig, in dem die Akkreditierungsentscheidung bekanntgegeben wird. Ist bei einer Programmakkreditierung der Studiengang noch nicht eröffnet, ist die Akkreditierung ab dem Beginn des Semesters oder Trimesters, in dem der Studiengang erstmalig angeboten wird, spätestens aber mit Beginn des zweiten auf die Bekanntgabe der Akkreditierungsentscheidung folgenden Semesters oder Trimesters wirksam.\\

\textcolor{Bernd}{\textbf{\cite{RESO: SoSe2018-RV}}
Eine Akkreditierungsfrist von 8 Jahren für eine Erstakkreditierung ist zu lang. Für neueingerichtete Studiengänge fordert die ZaPF eine erstmalige Reakkreditierung ein Jahr nach Ablauf der Regelstudienzeit, spätestens nach 5 Jahren.}

\paragraph{(2)} Vor Ablauf des Geltungszeitraums der Akkreditierung ist eine unmittelbar anschließende Akkreditierung (Reakkreditierung) einzuleiten. Reakkreditierungen sind für den Zeitraum von acht Jahren gültig.
\paragraph{(3)} Wird ein akkreditierter Studiengang nicht fortgeführt, kann die Akkreditierung für bei Ablauf des Geltungszeitraums der Akkreditierung noch eingeschriebene Studierende verlängert werden. Die Akkreditierung eines Studiengangs kann für einen Zeitraum von bis zu zwei Jahren verlängert werden, wenn die Hochschule einen Antrag auf eine Bündel- oder Systemakkreditierung vorbereitet, in die der jeweilige Studiengang einbezogen ist. Bei Antragstellung auf eine Bündel- oder Systemakkreditierung kann die Akkreditierung von Studiengängen, deren Akkreditierung während des Verfahrens endet, für die Dauer des Verfahrens zuzüglich eines Jahres vorläufig verlängert werden.
\subsection{§ 27 Auflagen}
\paragraph{(1)} Für die Erfüllung einer Auflage ist eine Frist von in der Regel zwölf Monaten zu setzen.
\paragraph{(2)} In begründeten Ausnahmefällen kann die Frist auf Antrag der Hochschule verlängert werden.
\paragraph{(3)} Die Erfüllung der Auflage ist gegenüber dem Akkreditierungsrat nachzuweisen.
\subsection{§ 28 Anzeigepflicht bei Änderungen}
\paragraph{(1) }Die Hochschule ist verpflichtet, dem Akkreditierungsrat unverzüglich jede wesentliche Änderung am Akkreditierungsgegenstand während des Geltungszeitraums der Akkreditierung anzuzeigen.
\paragraph{(2)} Der Akkreditierungsrat entscheidet, ob die wesentliche Änderung von der bestehenden Akkreditierung umfasst ist.
\subsection{§ 29 Veröffentlichung}
Die Entscheidung des Akkreditierungsrates und der Akkreditierungsbericht werden vom Akkreditierungsrat auf seiner Internetseite veröffentlicht. Bei der Veröffentlichung dürfen personenbezogene Daten nicht offenbart werden, es sei denn, die betroffene Person hat eingewilligt oder die Einholung der Einwilligung der betroffenen Person ist nicht oder nur mit unverhältnismäßigem Aufwand möglich und es ist offensichtlich, dass die Offenbarung im Interesse der betroffenen Person liegt. Sätze 1 und 2 gelten für interne Akkreditierungsentscheidungen systemakkreditierter Hochschulen entsprechend.
\subsection{§ 30 Bündelakkreditierung; Teil-Systemakkreditierung}
\paragraph{(1)} Das Gutachten des Gutachtergremiums nach § 24 Absatz 4 kann mehrere Studiengänge umfassen, wenn diese eine hohe fachliche Nähe aufweisen, die über die bloße Zugehörigkeit zu einer Fächerkultur (Geistes- und Kulturwissenschaften, Sozialwissenschaften oder Naturwissenschaften) hinaus geht (Bündelakkreditierung). Die fachlich-inhaltlichen Kriterien nach Teil 3 sind für jeden Studiengang gesondert zu prüfen. Ein Bündel soll sich aus nicht mehr als zehn Studiengängen zusammensetzen.
\\

\textcolor{Bernd}{\textbf{\cite{POS: WiSe2017-AkkRL}} Gebündelte Akkreditierungen von bis zu 10 Studiengängen in einem Verfahren sind möglich, ohne dass sich die Größe oder Zusammensetzung der Gutachter*innengruppe oder die Länge des Verfahrens ändert und unterliegt mangels klarer Definitionen kaum Einschränkungen.}

\paragraph{(2)} Auf Antrag der Hochschule kann der Akkreditierungsrat die konkrete Zusammensetzung des Bündels vor Einreichung des Antrags nach § 23 genehmigen.
\paragraph{(3)} Im Ausnahmefall kann eine studienorganisatorische Teileinheit der Hochschule Gegenstand der Systemakkreditierung sein. Dies kann insbesondere der Fall sein, wenn
\begin{enumerate}
\item die Akkreditierung des Qualitätsmanagementsystems für die gesamte
Hochschule noch nicht sinnvoll oder nicht praktikabel ist,
\item das Qualitätsmanagementsystem der Teileinheit in die Hochschule integriert ist und
\item mindestens ein Studiengang der Teileinheit dieses System bereits durchlaufen hat.
\end{enumerate}
\subsection{§ 31 Stichproben}
\paragraph{(1)} Bei der Systemakkreditierung und Teil-Systemakkreditierung wird vom Gutachtergremium nach § 25 Absatz 2 eine Stichprobe durchgeführt. In der Stichprobe wird geprüft, ob die im zu begutachtenden Qualitätsmanagementsystem angestrebten Wirkungen auf der Ebene des Studiengangs eintreten.
\paragraph{(2)} Gegenstand der Stichprobe ist
\begin{enumerate}
\item die Berücksichtigung aller Kriterien gemäß Teil 2 und Teil 3 innerhalb eines Studiengangs, der das Qualitätsmanagementsystem der Hochschule durchlaufen hat und
\item die Berücksichtigung formaler und fachlich-inhaltlicher Kriterien gemäß Teil 2 und Teil 3 nach Maßgabe des Gutachtergremiums
\end{enumerate}
Bei der Auswahl der Stichprobe berücksichtigt das Gutachtergremium das Fächerspektrum der Hochschule in der Lehre.
\paragraph{(3)} Bietet die Hochschule Studiengänge an, die auch auf einen reglementierten Beruf vorbereiten, ist hiervon zusätzlich einer unter Berücksichtigung der Kriterien nach Teil 2 und 3, die sich auf Studiengänge beziehen, in die Stichproben einzubeziehen; Gleiches gilt für den Fall von Lehramtsstudiengängen für jeweils einen Studiengang von jedem angebotenen Lehramtstyp sowie für Studiengänge mit Evangelischer oder Katholischer Theologie/Religion. An der Stichprobe wirkt jeweils ein von der für den jeweiligen reglementierten Beruf zuständigen Stelle benannter Vertreter oder eine von der für den jeweiligen reglementierten Beruf zuständigen Stelle benannte Vertreterin oder ein Vertreter oder eine Vertreterin der für das Schulwesen zuständigen Obersten Landesbehörde oder der jeweiligen kirchlichen Stelle mit.
\section{Teil 5 Verfahrensregeln für besondere Studiengangsformen}
\subsection{§ 32 Kombinationsstudiengänge}
\paragraph{(1)}  Wählen die Studierenden aus einer größeren Zahl zulässiger Fächer für das Studium einzelne Fächer aus, ist jedes dieser Fächer ein Teilstudiengang als Teil eines Kombinationsstudiengangs.
\paragraph{(2)} Akkreditierungsgegenstand ist der Kombinationsstudiengang. Die Hochschulen stellen durch ihr jeweiliges Qualitätsmanagement sicher, dass die Studierbarkeit in allen möglichen Fächerkombinationen gegeben ist.
\paragraph{(3)} Die Akkreditierung eines Kombinationsstudiengangs kann durch die Aufnahme weiterer wählbarer Teilstudiengänge oder Studienfächer ergänzt werden. Die  Akkreditierungsfrist für den Kombinationsstudiengang ändert sich dadurch nicht.\\

\textcolor{Bernd}{\textbf{\cite{POS: WiSe2017-AkkRL}} Bei akkreditierten Kombinationsstudiengängen können weitere Teilstudiengänge hinzugefügt werden, ohne, dass diese neu begutachtet werden müssen. Insbesondere muss so auf die Studierbarkeit der neuen Teilstudiengänge in Verbindung mit den alten Teilstudiengängen keine gesonderte Rücksicht genommen werden. }

\paragraph{(4)} Auf der Akkreditierungsurkunde werden alle in die Akkreditierung einbezogenen Teilstudiengänge oder Studienfächer aufgeführt. Im Falle der Ergänzung der Akkreditierung nach Absatz 3 ist eine neue Akkreditierungsurkunde auszustellen.
\paragraph{(5)} Die Regelungen von Teil 4 bleiben im Übrigen unberührt.\\


\subsection{§ 33 Joint-Degree-Programme}
\paragraph{(1)} Für Joint-Degree-Programme, an denen eine inländische Hochschule und weitere Hochschulen aus dem Europäischen Hochschulraum beteiligt sind, kann die Akkreditierungsentscheidung in Abweichung von § 22 Absatz 1 durch Anerkennung der Bewertung durch eine in dem European Quality Assurance Register for Higher Education (EQAR) gelistete Agentur getroffen werden. Der Akkreditierungsrat erkennt diese Bewertung auf Antrag der Hochschule an und verleiht sein Siegel, wenn die Einhaltung der formalen und fachlichinhaltlichen Kriterien für Joint-Degree-Programme gemäß Teil 2 und Teil 3 dieser Verordnung nachgewiesen ist und das Begutachtungsverfahren folgenden Anforderungen genügt hat:
\begin{enumerate}
\item die Durchführung des Verfahrens wurde dem Akkreditierungsrat vor Beginn des Verfahrens angezeigt,
\item die Akkreditierungsentscheidung beruht auf einem Selbstbericht der kooperierenden Hochschulen, der insbesondere Informationen zu den jeweiligen nationalen Rahmenbedingungen enthält und der die besonderen Merkmale des Joint-Degree-Programms hervorhebt,
\item es hat eine Begehung an mindestens einem Standort des Studiengangs unter Mitwirkung von Vertreterinnen und Vertretern aller kooperierenden Hochschulen sowie anderen Beteiligten stattgefunden,
\item die Bewertung beruht auf einem Gutachten, das die Maßgaben von JointDegree-Programmen in Teil 2 und Teil 3 beachtet,
\item die Begutachtung ist durch eine mindestens vierköpfige Gutachtergruppe erfolgt, die sich mindestens wie folgt zusammengesetzt hat:
\begin{enumerate}[label=\alph*]
\item Mitglieder aus mindestens zwei der am Joint-Degree-Programm
beteiligten Länder,
\item mindestens ein studentischer Vertreter oder eine studentische
Vertreterin,
\item die Gutachtergruppe repräsentiert Expertise in den entsprechenden Fächern und Fachdisziplinen einschließlich des Arbeitsmarktes/der Arbeitswelt in den entsprechenden Bereichen und Expertise auf dem Gebiet der Qualitätssicherung im Hochschulbereich und verfügt über Kenntnisse der Hochschulsysteme der beteiligten Hochschulen sowie der verwendeten Unterrichtssprachen und
\item die Maßgaben gemäß § 25 Absatz 3 Satz 1, Absätze 5 und 6 wurden
eingehalten,
\end{enumerate}
\item die Bewertung benennt folgende Merkmale: Begründung, Bestandskraft und gegebenenfalls nachgewiesene Erfüllung von Auflagen und
\item die Agentur hat das Gutachten und die Bewertung auf ihrer Homepage in deutscher und englischer Sprache veröffentlicht.
\end{enumerate}
Die § 22 Absätze 2, 3, und 4 Satz 1, 26 Absatz 1 Satz 1 und Absatz 2 Satz 1, §§ 28 und 29 gelten entsprechend. 4Die Akkreditierungsfrist beträgt in Abweichung von § 26 Absatz. 1 Satz 1 und Absatz 2 Satz 2 sechs Jahre. Bei der Veröffentlichung wird die Entscheidung als Akkreditierungsentscheidung auf Basis des gesonderten Verfahrens für Joint-Degree-Programme kenntlich gemacht. Die Hochschule hat dies in den Studienabschlussdokumenten deutlich zu machen.
\paragraph{(2)} Wird ein Joint Degree-Programm von einer inländischen Hochschule gemeinsam mit einer oder mehreren Hochschulen ausländischer Staaten koordiniert und angeboten, die nicht dem Europäischen Hochschulraum angehören (außereuropäische Kooperationspartner), so findet auf Antrag der inländischen Hochschule Absatz 1 entsprechende Anwendung, wenn sich die außereuropäischen Kooperationspartner in der Kooperationsvereinbarung mit der inländischen Hochschule zu einer Akkreditierung unter Anwendung der in Absatz 1, sowie der in den §§ 10 Absätze 1 und 2 und 16 Absatz 1 geregelten Kriterien verpflichtet.
\section{Teil 6 Alternative Akkreditierungsverfahren nach Artikel 3 Absatz 1 Nummer 3 Studienakkreditierungsstaatsvertrag}
\subsection{§ 34 Alternative Akkreditierungsverfahren}
\paragraph{(1)} Neben die beiden in Teil 4 geregelten Verfahren können gemäß Artikel 3 Absatz 1 Nummer 3 Studienakkreditierungsstaatsvertrag auch alternative Verfahren zur Sicherung und Entwicklung der Qualität in Studium und Lehre treten.
\paragraph{(2)} In alternativen Verfahren sind die Kriterien nach Teil 2 und Teil 3 dieser Verordnung einzuhalten. Die in Artikel 3 Absatz 2 Satz 1 Studienakkreditierungsstaatsvertrag sowie die im Studienakkreditierungsstaatsvertrag und in dieser Verordnung geltenden Grundsätze für die angemessene Beteiligung der Wissenschaft gelten entsprechend; ebenso gelten die Mitwirkungs- und Zustimmungserfordernisse gemäß § 18 Absatz 2 entsprechend.
\paragraph{(3)} Die Durchführung von alternativen Verfahren bedarf vorab der Zustimmung des Akkreditierungsrates und der zuständigen Wissenschaftsbehörde des jeweiligen Landes; der Akkreditierungsrat kann eine externe Begutachtung veranlassen. Der Antrag ist über die zuständige Wissenschaftsbehörde dem Akkreditierungsrat vorzulegen. Der Akkreditierungsrat kann im Rahmen der Abstimmung mit dem Land seine Zustimmung nur verweigern, wenn das alternative Verfahren den Maßgaben des Artikel 2 und den Bestimmungen des Artikel 3 Absatz 2 Satz 1 Studienakkreditierungsstaatsvertrag sowie den im Studienakkreditierungsstaatsvertrag und in dieser Verordnung festgelegten Grundsätzen für die angemessene Beteiligung der Wissenschaft nicht entspricht. Das alternative Verfahren soll geeignet sein, grundsätzliche Erkenntnisse zu alternativen Ansätzen externer Qualitätssicherung jenseits der in Artikel 3 Absatz 1 Nummern 1 und 2
Studienakkreditierungsstaatsvertrag genannten Verfahren zu gewinnen.
\paragraph{(4)} Der Akkreditierungsrat entwickelt eine Verfahrensordnung, die insbesondere die Antragsvoraussetzungen regelt.
\paragraph{(5)} Das alternativen Verfahren wird auf maximal acht Jahre befristet.§ 22 Absatz 4 Satz 2 und § 26 Absatz 3 Satz 3 gelten entsprechend.3 Es wird durch den Akkreditierungsrat begleitet und ist in der Regel zwei Jahre vor Ablauf der Projektzeit von einer unabhängigen, wissenschaftsnahen Einrichtung zu evaluieren.
\section{Teil 7 Sonstiges}
\subsection{§ 35 Verbindung mit Verfahren, die die berufszulassungsrechtliche Eignung eines Studiengangs zum Gegenstand haben }
\paragraph{(1)} Akkreditierungsverfahren gemäß Artikel 3 Absatz 1 Nummer 1 und Artikel 3 Absatz 1 Nummer 2 Studienakkreditierungsstaatsvertrag können auf Antrag der Hochschule mit Verfahren, die über die berufszulassungsrechtliche Eignung eines Studiengangs entscheiden, organisatorisch verbunden werden.
\paragraph{(2)} Die Beteiligung von zusätzlich zu den anderen Vertretern oder den Vertreterinnen der Berufspraxis zu berufenden externen Experten oder Expertinnen mit beratender Funktion in den Gutachtergremien gemäß § 25 Absatz 1 und Absatz 2 erfolgt durch Benennung der für den reglementierten Beruf jeweils zuständigen staatlichen Stelle.
\subsection{§ 36 Evaluation}
\paragraph{(1)} Nach Ablauf von drei Jahren nach Inkrafttreten dieser Verordnung werden ihre Anwendungen und Auswirkungen überprüft.
\paragraph{(2)} Über das Ergebnis ist der Ständigen Konferenz der Kultusminister der Länder in der Bundesrepublik Deutschland zu berichten.
\subsection{§ 37 Inkrafttreten}


\pagebreak
%\appendix %auskommentiert, da für eine Liste nicht sinnvoll
\section{Liste der Beschlüsse und Positionen der ZaPF}

\renewcommand\refname{\,}
\begin{thebibliography}{999}
	% \textsc{Name}, Vorname: \textit{Titel. Untertitel.} Auflage.
	%Verlagsort:
	%Verlag, Jahreszahl (= Reihe).

	% Name der Website,\\ \url{URL}(Aufrufdatum, ~Zeitpunkt)

\bibitem[RESO: WiSe2002-RL]{RESO: SoSe2002-RL} Weiterführende Richtlinien der ZaPF zu den Akkreditierungskriterien SoSe 2002\\
\url{https://zapf.wiki/Sammlung_aller_Resolutionen_und_Positionspapiere#WiSe_2002:_Akkreditierungskriterien_f.C3.BCr_Bachelorstudieng.C3.A4nge}

\bibitem[RESO: SoSe2008-MaZu]{RESO: SoSe2008-MaZu}Resolutionen zur Masterzulassung\\
\url{https://zapf.wiki/SoSe08_Beschl\%C3\%BCsse}

\bibitem[RESO: WiSe2008-RL]{RESO: WiSe2008-RL} Weiterführende Richtlinien der ZaPF zu den Akkreditierungskriterien WiSe 2008\\
\url{https://zapf.wiki/Sammlung_aller_Resolutionen_und_Positionspapiere#Resolution_zu_den_Akkreditierungsrichtlinien_der_ZaPF}

\bibitem[RESO: WiSe2008-Lehramt]{RESO: WiSe2008-Lehramt} Resoution aus der AK Lehramt\\
Reader Aachen 2008 \url{https://zapfev.de/reader/2008_WiSe_Aachen.pdf}\\

\bibitem[RESO: SoSe2010-BaMa]{RESO: SoSe2010-BaMa} Empfehlungen der ZaPF zur Ausgestaltung der Bachelor- und Master-Studiengänge im Fach Physik SoSe 2010\\
\url{https://zapf.wiki/Sammlung_aller_Resolutionen_und_Positionspapiere#Empfehlungen_zur_Ausgestaltung_der_Bachelor-_und_Master-Studieng.C3.A4nge_im_Fach_Physik}\\

\bibitem[RESO: WiSe2010-UebKon]{RESO: WiSe2010-UebKon} Resolution zu Übungskonzepten im Physikstudium WiSe 2010\\
\url{https://zapf.wiki/Sammlung_aller_Resolutionen_und_Positionspapiere#.C3.9Cbungskonzepte_2}

\bibitem[RESO: SoSe2013-Akksys]{RESO: SoSe2013-Akksys} Positionspapier zum deutschen Akkreditierungssystem SoSe 2013\\
\url{https://zapf.wiki/images/b/b2/Reso_SoSe13_Systemakkreditierung.pdf}\\

\bibitem[PROT: WiSe2015-RL]{PROT: WiSe2015-RL} Protokoll des AK Überarbeitung der Weiterführenden Richtlinien der ZaPF zu den Akkreditierungskriterien WiSe 2015\\
\url{https://zapf.wiki/WiSe15_AK_Überarbeitung_Akkreditierungsrichtlinien}

\bibitem[PROT: WiSe2018-Akk]{PROT: SoSe2018-Akk} Protokoll des AKs Akkreditierung I und II\\
\url{https://zapf.wiki/SoSe18_AK_Akkreditierung}\\
\url{https://zapf.wiki/SoSe18_AK_Akkreditierung_II}

\bibitem[POS: SoSe2016-Akksys]{POS: SoSe2016-Akksys} Positionspapier zum deutschen Akkreditierungssystem SoSe 2016\\
\url{https://zapf.wiki/Sammlung_aller_Resolutionen_und_Positionspapiere#Positionspapier_zum_deutschen_Akkreditierungssystem}

\bibitem[POS: SoSe2016-Prog]{POS: SoSe2016-Prog} Positionspapier zur Programmierfähigkeiten im Physikstudium\\
\url{https://zapf.wiki/Sammlung_aller_Resolutionen_und_Positionspapiere#Vermittlung_von_Programmierkompetenzen_im_Physikstudium}

\bibitem[PROT: SoSe2016-Komp]{PROT: SoSe2016-Komp} Positionspapier der ZaPF zu kompetenzorientierten Studiengängen SoSe 2016\\
\url{https://zapf.wiki/SoSe16_AK_Kompetenzorientierter_Physikstudiengang}

\bibitem[POS: WiSe2016-Ethik]{POS: WiSe2016-Ethik} Positionspapier der ZaPF zu Ethikinhalten im Physikstudium\\
\url{https://zapf.wiki/Sammlung_aller_Resolutionen_und_Positionspapiere#Positionspapier_zu_Ethikinhalten_im_Physikstudium}

\bibitem[POS: SoSe2017-Prak]{POS: SoSe2017-Prak} Positionspapier zur Ausgestaltung von Grund-/Anfängerpraktika\\
\url{https://zapf.wiki/Sammlung_aller_Resolutionen_und_Positionspapiere#Positionspapier_zu_Lernzielen_f.C3.BCr_Grund-_oder_Anf.C3.A4ngerpraktika_der_Physik}

\bibitem[RESO: WiSe2017-BPrak]{RESO: WiSe2017-BPrak} Resolution zu Berufsorientierenden Praktika\\
\url{https://zapfev.de/resolutionen/wise17/BerufsorientierendePraktika/Reso_BerufsorientierendePraktika.pdf}

\bibitem[POS: WiSe2017-AkkRL]{POS: WiSe2017-AkkRL} Positionspapier der ZaPF zur Überarbeitung der Akkreditierungsrichtlinien WiSe 2017\\
\url{https://zapf.wiki/Sammlung_aller_Resolutionen_und_Positionspapiere#Positionspapier_zu_Akkreditierung}

\bibitem[POS: WiSe2018-WiKom]{POS: WiSe2018-WiKom} Positionspapier der ZaPF zur Förderung der Wissenschaftskommunikation WiSe 2018\\
\url{https://zapf.wiki/Sammlung_aller_Resolutionen_und_Positionspapiere#Positionspapier_zur_F.C3.B6rderung_der_Wissenschaftskommunikation_in_der_akademischen_Ausbildung}

\bibitem[RESO: SoSe2018-Mob]{RESO: SoSe2018-Mob} Resolution zur Mobilität/Uniwechsel SoSe 2018\\
\url{https://zapf.wiki/Sammlung_aller_Resolutionen_und_Positionspapiere#Resolution_zur_Studierendenmobilit.C3.A4t}

\bibitem[RESO: SoSe2018-RV]{RESO: SoSe2018-RV} Resolution zu länderspezifischen Rechtsverordnungen als Spezifizierung der MRVO\\
\url{https://zapf.wiki/Sammlung_aller_Resolutionen_und_Positionspapiere#Resolution_zur_l.C3.A4nderspezifischen_Rechtsverordnungen_als_Spezifizierung_der_MRVO}

\bibitem[POS: SoSe2018-AkkRL]{POS: SoSe2018-AkkRL} Positionspapier der ZaPF zur Musterrechtsverordnung WiSe 2017\\
\url{https://zapf.wiki/Sammlung_aller_Resolutionen_und_Positionspapiere#Positionspapier_zum_g.C3.BCltigen_Studienakkreditierungsstaatsvertrag_und_der_dazugeh.C3.B6rigen_Musterrrechtsverordnung}

\bibitem[POS: SoSe2018-Prak]{POS: SoSe2018-Prak} Positionspapier zur Ausgestaltung von Fortgeschrittenenpraktika (Momentan noch nicht beschlossen, siehe Protokoll SoSe2018 gab es in Würzburg FortsetzungsAK zu)\\
\url{https://zapf.wiki/SoSe18_AK_Fortgeschrittenenpraktikums}

\bibitem[RESO: WiSe2018-PAA]{RESO: WiSe2018-PAA} Resolution zur flexiblen Prüfungsan-/abmeldung \\
\url{https://zapf.wiki/Sammlung_aller_Resolutionen_und_Positionspapiere#Resolution_f.C3.BCr_einen_flexibleren_Umgang_mit_Pr.C3.BCfungsan-_und_abmeldungen}

\bibitem[RESO: WiSe2018-VorVer]{RESO: WiSe2018-VorVer} Resolution der ZaPF zu vorläufigen Verträgen von Abschlussarbeiten (Momentan noch nicht beschlossen, siehe Protokoll SoSe2018 und WiSe2018)\\
\url{https://zapf.wiki/WiSe18_AK_Vorläufige_Verträge_für_Abschlussarbeiten}


\bibitem[Akk-Pool-POS: WiSe2018]{POS: WiSe2018-RL} Stellungsnahme des studentischen Akkreditierungspools zur MRVO \url{https://www.studentischer-pool.de/wp-content/uploads/2017/11/MuRVO_stud_Stellungnahme_28.11.pdf}

\bibitem[RESO: SoSe2018-Fr]{RESO: SoSe2018-Fr} Resolution zu Reakkreditierungsfristen und Gutachter*innenzusammensetzung \url{https://zapfev.de/resolutionen/sose18/Akkreditierung/reso_laender_akkr.pdf}


\bibitem[RESO: WiSe2017-Akkwesen]{RESO: WiSe2017-Akkwesen} Resolution zu Änderungen im Akrreditierungswesen \url{https://zapfev.de/resolutionen/sose18/Akkreditierung/reso_laender_akkr.pdf}

\bibitem[RESO: SoSe15-Eduroam]{RESO: SoSe15-Eduroam} Resolution zur qualitativen Umsetzung von eduroam und anderen
hochschulöffentlichen Netzwerken an allen Hochschulen \url{https://zapfev.de/resolutionen/sose15/Netzneutralitaet_in_Universitaetsnetzen/Resolution_SoSe15_Netzneutralit\%C3\%A4t_in_Universit\%C3\%A4tsnetzen.pdf}

\bibitem[RESO: SoSe2013-Anerkennung]{RESO: SoSe2013-Anerkennung}
Anerkennung von Studienleistungen \url{https://zapfev.de/resolutionen/sose15/Netzneutralitaet_in_Universitaetsnetzen/Resolution_SoSe15_Netzneutralit\%C3\%A4t_in_Universit\%C3\%A4tsnetzen.pdf}

\bibitem[RESO: SoSe2013-SysAkk]{RESO: SoSe2013-SysAkk} Systemakkreditierung \url{https://zapfev.de/resolutionen/sose13/Reso_SoSe13_Systemakkreditierung.pdf}

\bibitem[RESO: WiSe2012-ZivEng]{RESO: WiSe2012-ZivEng} Zivilgesellschaftliches Engagement \url{https://zapfev.de/resolutionen/wise12/Reso_WiSe12_ZivilgesellschaftlichesEngagement.pdf}

\end{thebibliography}

\renewcommand\refname{Liste von ZaPF externen Quellen}
\begin{thebibliography}{999}
		% \textsc{Name}, Vorname: \textit{Titel. Untertitel.} Auflage.
		%Verlagsort:
		%Verlag, Jahreszahl (= Reihe).

		% Name der Website,\\ \url{URL}(Aufrufdatum, ~Zeitpunkt)

\bibitem[ESG]{ESG} ESG-Kriterien\\
\url{https://www.hrk.de/uploads/media/ESG_German_and_English_2015.pdf}

\bibitem[MRVO-Raster]{MRVO-Raster} Raster der Musterrechtsverordnung\\
\url{http://www.akkreditierungsrat.de/fileadmin/Seiteninhalte/AR/Beschluesse/Neues_System/Raster_Akkreditierungsbericht_Programm_Fassung_01.pdf}

\bibitem[akkreditierungsrichtlinien]{akkreditierungsrichtlinien} SoSe19 Resolution zu Akkreditierungsrichtlinien\\
\url{https://zapfev.de/resolutionen/sose19/Akkreditierungsrichtlinien_der_ZaPF/Akkreditierungsrichtlinien_der_ZaPF.pdf}

RESO: WiSe2012-ZivEng





\end{thebibliography}




\end{document}
