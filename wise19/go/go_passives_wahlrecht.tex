\documentclass[draft,10pt,oneside]{scrartcl}

% Sprache und Encodings
\usepackage[ngerman]{babel}
\usepackage[T1]{fontenc}
\usepackage[utf8]{inputenc}

% Typographisch interessante Pakete
\usepackage{microtype} % Randkorrektur und andere Anpassungen

% References to Internet and within the document
\usepackage[pdftex,colorlinks=false,
pdftitle={Antrag zur Änderung der Geschäftsordnung für Plenen der ZaPF},
pdfauthor={Jörg Behrmann (FUB), Björn Guth (RWTH)},
pdfcreator={pdflatex},
pdfdisplaydoctitle=true]{hyperref}

% Absaetze nicht Einruecken
\setlength{\parindent}{0pt}
\setlength{\parskip}{2pt}

% Formatierung auf A4 anpassen
\usepackage{geometry}
\geometry{paper=a4paper,left=15mm,right=15mm,top=10mm,bottom=10mm}

\begin{document}

\section*{Antrag zur Änderung der Geschäftsordnung für Plenen der ZaPF}

\textbf{Antragsteller:} Jörg Behrmann (FUB), Björn Guth (RWTH)

\subsection*{Antrag}

Hiermit beantragen wir die Geschäftsordnung für Plenen der ZaPF wie folgend zu
ändern:

In 4.2 ersetze:
\begin{quote}
    \begin{enumerate}
        \item Das passive Wahlrecht für Personenwahlen haben alle teilnehmenden
            Personen der ZaPF. Von dieser Regel wird abgesehen, falls die
            Personenwahl eine Wiederwahl oder Bestätigung im Amt ist, so dass
            in diesem Fall auch nicht teilnehmende Personen gewählt werden
            können.
    \end{enumerate}
\end{quote}
durch
\begin{quote}
    \begin{enumerate}
        \item Das \textit{passive Wahlrecht} für Personenwahlen haben alle
            natürlichen Personen.
    \end{enumerate}
\end{quote}
Zusätzlich füge folgendes als Anhang an:
\begin{quote}
    \subsection*{Passives Wahlrecht}

    Das Plenum soll jede Person wählen können, der die teilnehmenden Personen
    die Ausübung des Amtes zutrauen. Dies ist ein breites Recht und bringt die
    Pflicht mit, sorgfältig auszuwählen, wen es wählt. Die teilnehmenden
    Personen sollen sich mit den kandidierenden Personen bekannt machen und die
    ZaPF nutzen, diese kennenzulernen und sich eine Meinung über sie zu bilden.
    Kandidierende Personen sollen sich dem Plenum in geeigneter Form
    vorstellen.

    Es ist immer eine Option Menschen nicht zu wählen und Ämter vakant zu
    lassen, da es besser sein kann sich länger mit kandidierenden Personen
    vertraut zu machen und sie im Zweifel später zu wählen. Die Wahl in Ämter
    ist keine Voraussetzung um sich aktiv in Gremien einzubringen.
\end{quote}

\end{document}
