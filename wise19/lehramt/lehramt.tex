\documentclass[DIV=calc]{scrartcl}
\usepackage[utf8]{inputenc}
\usepackage[T1]{fontenc}
\usepackage[ngerman]{babel}
\usepackage{graphicx}
\usepackage{csquotes}

\usepackage{paralist}
\usepackage{fixltx2e}
%\usepackage{ellipsis}
\usepackage[tracking=true]{microtype}

\usepackage{lmodern}              % Ersatz fuer Computer Modern-Schriften
%\usepackage{hfoldsty}

%\usepackage{fourier}             % Schriftart
\usepackage[scaled=0.81]{helvet}     % Schriftart

\usepackage{url}
\usepackage{tocloft}             % Paket für Table of Contents

\usepackage{xcolor}
\definecolor{urlred}{HTML}{660000}

\usepackage{hyperref}
\hypersetup{colorlinks=false}

\usepackage{mdwlist}     % Änderung der Zeilenabstände bei itemize und enumerate

\parindent 0pt                 % Absatzeinrücken verhindern
\parskip 12pt                 % Absätze durch Lücke trennen

\setlength{\textheight}{23cm}
\usepackage{fancyhdr}
\pagestyle{fancy}
\fancyhead{} % clear all header fields
\cfoot{}
\lfoot{Zusammenkunft aller Physik-Fachschaften}
\rfoot{www.zapfev.de\\stapf@zapf.in}
\renewcommand{\headrulewidth}{0pt}
\renewcommand{\footrulewidth}{0.1pt}
\newcommand{\gen}{*innen}
\addto{\captionsngerman}{\renewcommand{\refname}{Quellen}}

%%%% Mit-TeXen Kommandoset
\usepackage[normalem]{ulem}
\usepackage{xcolor}

\begin{document}

	\hspace{0.87\textwidth}
	\begin{minipage}{120pt}
		\vspace{-1.8cm}
		\includegraphics[width=80pt]{../logo.pdf}
		\centering
		\small Zusammenkunft aller Physik-Fachschaften
	\end{minipage}
	\begin{center}
		\huge{Positionspapier zum Lehramtsstudium}\vspace{.25\baselineskip}\\
		\normalsize
	\end{center}
	\vspace{1cm}

	%%%% Text des Antrages zur veröffentlichung %%%%

	\section*{Flexibilisierung der Fächerkombinationen im Lehramt}


	Mathematik gilt als gängiges Fach in der Kombination mit dem Fach Physik.
	Als Folge dessen setzen einige Physikvorlesungen implizit mathematische Grundlagen voraus.
	Diese sind in einem Lehramtsbachelor ohne das Fach Mathematik, als eines der beiden Fächer nicht im Studienverlaufsplan vorgesehen.
	Dadurch wird die Kombinierbarkeit von Physik mit anderen Fächern stark eingeschränkt.
	Die ZaPF fordert die Öffnung der Kombinationsmöglichkeiten mit anderen Fächern und ruft die Hochschulen auf, die Studierbarkeit in verschiedenen Kombinationen zu ermöglichen.
	\newline
	\\


\vfill
\begin{flushright}
	Verabschiedet am 3.11.2019 in Freiburg.
\end{flushright}


\end{document}
