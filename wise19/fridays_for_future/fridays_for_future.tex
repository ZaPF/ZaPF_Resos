\documentclass[draft,10pt,oneside]{scrartcl}

% Sprache und Encodings
\usepackage[ngerman]{babel}

% Typographisch interessante Pakete
\usepackage{microtype} % Randkorrektur und andere Anpassungen

% References to Internet and within the document
\usepackage[pdftex,colorlinks=false, pdftitle={Antrag zur Änderung der
Geschäftsordnung für Plenen der ZaPF}, pdfauthor={Jörg Behrmann (FUB), Björn
Guth (RWTH)}, pdfcreator={pdflatex}, pdfdisplaydoctitle=true]{hyperref}

% Absaetze nicht Einruecken
\setlength{\parindent}{0pt} \setlength{\parskip}{2pt}

% Formatierung auf A4 anpassen
\usepackage{geometry}
\geometry{paper=a4paper,left=15mm,right=15mm,top=10mm,bottom=10mm}

\begin{document}

\section*{Solidarisierung der ZaPF mit Fridays for Future}

\textbf{Antragstellende:} Björn (RWTH), Jörg (FUB), Stephie (HUB), Joshua
(Köln), Fabs (TUB), Hannah (HUB)

\textbf{zu Adressierende:} alle deutschsprachigen Hochschulen, HRK,
Kultusministerien, Schulleitungsverbände der Länder, Landesschulbehörden, KMK,
deutschsprachige Physik-Fachschaften, LAKs, MeTaFa

\subsection*{Antrag}

Die ZaPF möge beschließen:

\begin{quote}
    Die ZaPF solidarisiert sich mit der Bewegung \glqq{}Fridays For
    Future\grqq{}.

    Wir fordern von den (Hoch)Schulen, die nötigen Freiräume zu schaffen, um
    Kindern, Jugendlichen und jungen Erwachsenen die Teilnahme an Protesten zu
    ermöglichen.  Weiter verurteilen wir Repressionen gegen die an den
    Protesten Teilnehmenden. Dies betrifft sowohl die Androhung, als auch die
    konkrete Anwendung solcher Maßnahmen.

    Außerdem rufen wir alle Fachschaften dazu auf, die Proteste im Rahmen ihrer
    Möglichkeiten zu unterstützen und sich dafür einzusetzen, Studierenden die
    oben erwähnten Freiräume zu gewähren.
\end{quote}

\end{document}
