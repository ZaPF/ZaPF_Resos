\documentclass[draft,10pt,oneside]{scrartcl}

% Sprache und Encodings
\usepackage[ngerman]{babel}

% Typographisch interessante Pakete
\usepackage{microtype} % Randkorrektur und andere Anpassungen

% References to Internet and within the document
\usepackage[pdftex,colorlinks=false, pdftitle={Antrag zur Änderung der
Geschäftsordnung für Plenen der ZaPF}, pdfauthor={Jörg Behrmann (FUB), Björn
Guth (RWTH)}, pdfcreator={pdflatex}, pdfdisplaydoctitle=true]{hyperref}

% Absaetze nicht Einruecken
\setlength{\parindent}{0pt} \setlength{\parskip}{2pt}

% Formatierung auf A4 anpassen
\usepackage{geometry}
\geometry{paper=a4paper,left=15mm,right=15mm,top=10mm,bottom=10mm}

\begin{document}

\section*{Solidarisierung der ZaPF mit Fridays for Future}

\textbf{Antragstellende:} Björn (RWTH), Jörg (FUB), Steph (HUB), Joshua
(Köln), Fabs (TUB), Hannah (HUB)

\textbf{zu Adressierende:} alle deutschsprachigen Hochschulen, HRK,
Kultusministerien, Schulleitungsverbände der Länder, Landesschulbehörden, KMK,
deutschsprachige Physik-Fachschaften, LAKs, MeTaFa

\subsection*{Antrag}

Die ZaPF möge beschließen:

\begin{quote}
    Die ZaPF solidarisiert sich mit der Bewegung \glqq{}Fridays For
    Future\grqq{} und ruft dazu auf, ihre Forderungen umzusetzen.

    Wir fordern von allen (Hoch)Schulen, die nötigen Freiräume zu schaffen, um
    Kindern, Jugendlichen und jungen Erwachsenen die Teilnahme an Protesten zu
    ermöglichen.  Weiter verurteilen wir alle Repressionen gegen die an den
    Protesten Teilnehmenden. Dies betrifft sowohl die Androhung, als auch die
    konkrete Anwendung solcher Maßnahmen. Als Beispiel für eine verträgliche
    Regelung möchten wir hier die Einbeziehung der Personensorgeberechtigten
    bei der Freistellung vom Unterricht zur Teilnahme an Demonstrationen
    erwähnen, wie sie am Gymnasium Markranstädt praktiziert wird.

    Außerdem rufen wir alle Hochschulangehörigen dazu auf, die Proteste zu
    unterstützen, zum Beispiel im Rahmen der Public Climate
    School\footnote{https://studentsforfuture.info/public-climate-school/}, und
    sich dafür einzusetzen, Studierenden die oben erwähnten Freiräume zu
    gewähren.
\end{quote}

\end{document}
