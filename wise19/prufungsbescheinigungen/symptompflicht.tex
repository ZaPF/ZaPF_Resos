\documentclass[DIV=calc]{scrartcl}
\usepackage[utf8]{inputenc}
\usepackage[T1]{fontenc}
\usepackage[ngerman]{babel}
\usepackage{graphicx}

\usepackage{fixltx2e}
\usepackage{ellipsis}
\usepackage[tracking=true]{microtype}

\usepackage{lmodern}                        % Ersatz fuer Computer Modern-Schriften
\usepackage{hfoldsty}

\usepackage[scaled=0.81]{helvet} 	% Schriftart

\usepackage{url}
\usepackage{tocloft} 			% Paket für Table of Contents

\usepackage{xcolor}
\definecolor{urlred}{HTML}{660000}

\usepackage{hyperref}
\hypersetup{
  colorlinks=true,
  linkcolor=black,	% Farbe der internen Links (u.a. Table of Contents)
  urlcolor=black,	% Farbe der url-links
  citecolor=black} % Farbe der Literaturverzeichnis-Links

\usepackage{mdwlist} 	% Änderung der Zeilenabstände bei itemize und enumerate

\parindent 0pt 				% Absatzeinrücken verhindern
\parskip 12pt 				% Absätze durch Lücke trennen

\usepackage{titlesec}	% Abstand nach Überschriften neu definieren
\titlespacing{\subsection}{0ex}{3ex}{-1ex}
\titlespacing{\subsubsection}{0ex}{3ex}{-1ex}

\parindent 0pt                 % Absatzeinrücken verhindern
\parskip 12pt                 % Absätze durch Lücke trennen
\setlength{\textheight}{23cm}
\usepackage{fancyhdr}
\pagestyle{fancy}
\fancyhead{} % clear all header fields
\cfoot{}
\lfoot{Zusammenkunft aller Physik-Fachschaften}
\rfoot{www.zapfev.de\\stapf@zapf.in}
\renewcommand{\headrulewidth}{0pt}
\renewcommand{\footrulewidth}{0.1pt}
\newcommand{\gen}{*innen}
\addto{\captionsngerman}{\renewcommand{\refname}{Quellen}}

%%%% Mit-TeXen Kommandoset
\usepackage[normalem]{ulem}
\usepackage{xcolor}


\begin{document}
\hspace{0.87\textwidth}
\begin{minipage}{120pt}
	\vspace{-1.8cm}
	\includegraphics[width=80pt]{../logo.pdf}
	\centering
	\small Zusammenkunft aller Physik-Fachschaften
\end{minipage}

\begin{center}
\huge{Resolution zu Prüfungsunfähigkeitsbescheinigungen} \\
\normalsize
\end{center}



\section*{}
  Wir fordern, dass zum Nachweis der krankheitsbedingten Prüfungsunfähigkeit eine ärztliche Bescheinigung über die Prüfungsunfähigkeit akzeptiert wird. Eine Arbeitsunfähigkeitsbescheinigung ist dabei einer ärztlichen Prüfungsunfähigkeitsbescheinigung gleichzusetzen.

  Sollte das ursprünglich ausgestellte Attest nicht den Ansprüchen des Prüfungsausschusses genügen, obliegt es dem Prüfungsausschuss der jeweiligen Hochschule einen Amtsarzt/eine Amtsärztin hinzuzuziehen. Jedoch sollte auch deren Urteil unter ärztliche Schweigepflicht   gestellt sein und nur hinsichtlich der Leistungsminderung und Prüfungsempfehlung ein Urteil abgegeben werden. Die Kosten für den Amtsarzt/die Amtsärztin sind dabei von der Hochschule zu übernehmen, um eine Benachteiligung finanziell schwächer gestellter Studierender zu verhindern.

  In keinem Fall dürfen Studierende dazu gezwungen werden, Diagnosen oder Symptome  gegenüber der Hochschule offenzulegen und somit medizinisches Fachpersonal von der Schweigepflicht zu entbinden. Dies stellt einen  hoch unangemessenen sowie unnötigen Eingriff in die Intimsphäre der betroffenen Person dar.

\vfill
\begin{flushright}
Verabschiedet am 3.11.2019 in Freiburg.
\end{flushright}

\end{document}
