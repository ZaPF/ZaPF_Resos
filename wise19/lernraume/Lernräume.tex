\documentclass[a4paper]{scrartcl}
\usepackage[utf8]{inputenc}
\usepackage[T1]{fontenc}
\usepackage[ngerman]{babel}
\usepackage{graphicx}
\usepackage{csquotes}

\usepackage{ulem}
%\usepackage[dvipsnames]{xcolor}
\usepackage{paralist}
\usepackage{fixltx2e}
%\usepackage{ellipsis}
\usepackage[tracking=true]{microtype}

\usepackage{lmodern}              % Ersatz fuer Computer Modern-Schriften
%\usepackage{hfoldsty}

\usepackage[scaled=0.81]{helvet}     % Schriftart

\usepackage{url}
\usepackage{tocloft}             % Paket für Table of Contents

\usepackage{xcolor}
\definecolor{urlred}{HTML}{660000}

\usepackage{hyperref}
\hypersetup{
    colorlinks=true,
    linkcolor=black,    % Farbe der internen Links (u.a. Table of Contents)
    urlcolor=black,    % Farbe der url-links
    citecolor=black} % Farbe der Literaturverzeichnis-Links

\usepackage{mdwlist}     % Änderung der Zeilenabstände bei itemize und enumerate

\parindent 0pt                 % Absatzeinrücken verhindern
\parskip 12pt                 % Absätze durch Lücke trennen

\setlength{\textheight}{23cm}
\usepackage{fancyhdr}
\pagestyle{fancy}
\fancyhead{} % clear all header fields
\cfoot{}
\lfoot{Zusammenkunft aller Physik-Fachschaften}
\rfoot{www.zapfev.de\\stapf@zapf.in}
\renewcommand{\headrulewidth}{0pt}
\renewcommand{\footrulewidth}{0.1pt}
\newcommand{\gen}{*innen}
\addto{\captionsngerman}{\renewcommand{\refname}{Quellen}}


\begin{document}
    \hspace{0.87\textwidth}
    \begin{minipage}{120pt}
        \vspace{-1.8cm}
        \includegraphics[width=80pt]{../logo.pdf}
        \centering
        \small Zusammenkunft aller Physik-Fachschaften
    \end{minipage}
    \begin{center}
        \huge{Resolution zu Lern- und Arbeitsräumen}\vspace{.25\baselineskip}\\
        \normalsize
    \end{center}
    \vspace{0.5cm}

\subsection*{Lern- und Arbeitsräume}

Die ZaPF fordert, dass den Studierenden ausreichende Räumlichkeiten zum Selbststudium zur Verfügung gestellt werden. Mit großer Sorge betrachten wir, dass im Rahmen von Umstrukturierungen immer mehr solcher Räume verschwinden.


Eine der grundlegenden Kompetenzen, die im Physikstudium vermittelt werden, ist das selbstorganisierte Arbeiten, sowohl alleine, als auch in Gruppen. Bestehende Räume sind dafür oftmals nicht ausreichend ausgestattet und bieten nicht die notwendigen Rahmenbedingungen.

\begin{itemize}
	\item Sowohl Einzel-, insbesondere aber auch Gruppenarbeitsplätze sollten in ausreichender Anzahl zur Verfügung stehen.
	\item Neben ausreichender Beleuchtung sind Internetanschluss, Stromanschluss und eine ruhige Umgebung maßgeblich.
	\item In Gruppenlernräumen sollte eine Tafel oder ein Whiteboard und Gebrauchsmaterial dafür vorhanden sein.
	\item Zum Selbststudium gehört auch die selbständige Einteilung der Lernzeiten. Lern\-räume sollten daher möglichst nicht durch Schließzeiten eingeschränkt werden.
	\item Arbeitsräume, die zum Zwecke des eigenständigen Lernens zur Verfügung gestellt sind, sollten von der Verwaltung als genutzt angesehen werden.
	\item Seminarräume sollten für die Studierenden zum eigenständigen Lernen zur Ver\-fügung stehen, wenn sie nicht anderweitig genutzt werden.
\end{itemize}

Neben reinen Lernräumen sind auch Aufenthaltsräume für Studierende notwendig. Sie fördern den semesterübergreifenden Austausch und tragen so zu einem effektiven Arbeitsklima bei.

Sofern keine baulichen Maßnahmen notwendig sind, ist die Umsetzung obiger Forderungen nach Ansicht der ZaPF für die zuständigen Instanzen mit geringem finanziellen Aufwand verbunden. Mit den Vertretungen der Studierenden sollte dabei Rücksprache gehalten werden.

\vspace*{\fill}
\begin{flushright}
	Verabschiedet am 2.11.2019 in Freiburg
\end{flushright}

\end{document}
