\documentclass[DIV=calc]{scrartcl}
\usepackage[utf8]{inputenc}
\usepackage[T1]{fontenc}
\usepackage[ngerman]{babel}
\usepackage{graphicx}
\usepackage{csquotes}

%\usepackage[dvipsnames]{xcolor}
\usepackage{paralist}
\usepackage{fixltx2e}
%\usepackage{ellipsis}
\usepackage[tracking=true]{microtype}

\usepackage{lmodern}              % Ersatz fuer Computer Modern-Schriften
%\usepackage{hfoldsty}

%\usepackage{fourier}             % Schriftart
\usepackage[scaled=0.81]{helvet}     % Schriftart

\usepackage{url}
\usepackage{tocloft}             % Paket für Table of Contents

\usepackage{xcolor}
\definecolor{urlred}{HTML}{660000}

\usepackage{hyperref}
\hypersetup{
    colorlinks=true,
    linkcolor=black,    % Farbe der internen Links (u.a. Table of Contents)
    urlcolor=black,    % Farbe der url-links
    citecolor=black} % Farbe der Literaturverzeichnis-Links

\usepackage{mdwlist}     % Änderung der Zeilenabstände bei itemize und enumerate

\parindent 0pt                 % Absatzeinrücken verhindern
\parskip 12pt                 % Absätze durch Lücke trennen

\setlength{\textheight}{23cm}
\usepackage{fancyhdr}
\pagestyle{fancy}
\fancyhead{} % clear all header fields
\cfoot{}
\lfoot{Zusammenkunft aller Physik-Fachschaften}
\rfoot{www.zapfev.de\\stapf@zapf.in}
\renewcommand{\headrulewidth}{0pt}
\renewcommand{\footrulewidth}{0.1pt}
\newcommand{\gen}{*innen}
\addto{\captionsngerman}{\renewcommand{\refname}{Quellen}}

%%%% Mit-TeXen Kommandoset
\usepackage[normalem]{ulem}
\usepackage{xcolor}


\begin{document}
    \hspace{0.87\textwidth}
    \begin{minipage}{120pt}
        \vspace{-1.8cm}
        \includegraphics[width=80pt]{../logo.pdf}
        \centering
        \small Zusammenkunft aller Physik-Fachschaften
    \end{minipage}
    \begin{center}
      \vspace{1cm}
      \huge{Selbstverplichtung für Vertrauenspersonen}\vspace{.25\baselineskip}\\
      \normalsize
    \end{center}
    % \vspace{1cm}


Die ZaPF bekräftigt (und konkretisiert) die Selbstverpflichtung aus Wien 2013\footnote{\url{https://zapf.wiki/WiSe13_AK_Anti_Harassment_Policy}}:
\begin{quote}
	Die angesprochene Vertrauensperson ist zur Diskretion gegenüber den Hilfesuchenden verpflichtet.
	Der Opfer- und Täterschutz ist den Umständen entsprechend zu gewährleisten.
	Die Vertrauenspersonen sind in letzter Instanz ihrem Gewissen verpflichtet.
\end{quote}
Um die Arbeit der Vertrauenspersonen zu verbessern, ist der Austausch der Vertrauenspersonen untereinander gestattet.
Dabei wird die Anonymität der Hilfesuchenden stets gewährleistet.
In jedem Einzelfall muss mit der hilfesuchenden Person abgeklärt werden, ob und in welchem Umfang das Themengebiet abstrahiert in den Austausch getragen wird.
Dazu ist zum Beispiel geplant, dass sich Vertrauenspersonen zum Ende einer ZaPF treffen.
Mit einem Austausch soll sicher gestellt werden, dass Probleme erkannt und strukturiert angegangen werden können.
Ziel ist es beispielsweise Fortbildungen zu Schwerpunktthemen arrangieren zu können.
Mögliche Maßnahmen gegen strukturelle Probleme sollen mit zukünftigen Orgas besprochen werden.

\vfill
    \begin{flushright}
        Verabschiedet am 2.11.2019 in Freiburg.
    \end{flushright}
\end{document}
