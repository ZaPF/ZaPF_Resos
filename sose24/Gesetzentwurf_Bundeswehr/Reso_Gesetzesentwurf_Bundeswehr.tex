\documentclass[DIV=calc]{scrartcl}
\usepackage[utf8]{inputenc}
\usepackage[T1]{fontenc}
\usepackage[ngerman]{babel}
\usepackage{graphicx}
\usepackage[draft, markup=underlined]{changes}
\usepackage{csquotes}
\usepackage{eurosym}

\usepackage{ulem}
%\usepackage[dvipsnames]{xcolor}
\usepackage{paralist}
%\usepackage{fixltx2e}
%\usepackage{ellipsis}
\usepackage[tracking=true]{microtype}

\usepackage{lmodern}              % Ersatz fuer Computer Modern-Schriften
%\usepackage{hfoldsty}

%\usepackage{fourier}             % Schriftart
\usepackage[scaled=0.81]{helvet}     % Schriftart

\usepackage{url}
%\usepackage{tocloft}             % Paket für Table of Contents
\def\UrlBreaks{\do\a\do\b\do\c\do\d\do\e\do\f\do\g\do\h\do\i\do\j\do\k\do\l%
\do\m\do\n\do\o\do\p\do\q\do\r\do\s\do\t\do\u\do\v\do\w\do\x\do\y\do\z\do\0%
\do\1\do\2\do\3\do\4\do\5\do\6\do\7\do\8\do\9\do\-}%

\usepackage{xcolor}
\definecolor{urlred}{HTML}{660000}

\usepackage{hyperref}
\hypersetup{colorlinks=false}

%\usepackage{mdwlist}     % Änderung der Zeilenabstände bei itemize und enumerate
% \usepackage[scale=0.8,colorspec=0.9]{draftwatermark} % Wasserzeichen ``Entwurf''
% \SetWatermarkText{Vorbehaltlich\\redaktioneller\\Änderungen}

\parindent 0pt                 % Absatzeinrücken verhindern
\parskip 12pt                 % Absätze durch Lücke trennen

\setlength{\textheight}{23cm}
\usepackage{fancyhdr}
\pagestyle{fancy}
\fancyhead{} % clear all header fields
\cfoot{}
\lfoot{Zusammenkunft aller Physik-Fachschaften}
\rfoot{www.zapfev.de\\stapf@zapf.in}
\renewcommand{\headrulewidth}{0pt}
\renewcommand{\footrulewidth}{0.1pt}
\newcommand{\gen}{*innen}
\addto{\captionsngerman}{\renewcommand{\refname}{Quellen}}

%%%% Mit-TeXen Kommandoset
\usepackage[normalem]{ulem}
\usepackage{xcolor}
\usepackage{xspace} 

\newcommand{\replace}[2]{
    \sout{\textcolor{blue}{#1}}~\textcolor{blue}{#2}}
\newcommand{\delete}[1]{
    \sout{\textcolor{red}{#1}}}
\newcommand{\add}[1]{
    \textcolor{blue}{#1}}

\newif\ifcomments
\commentsfalse
%\commentstrue

\newcommand{\red}[1]{{\ifcomments\color{red} {#1}\else{#1}\fi}\xspace}
\newcommand{\blue}[1]{{\ifcomments\color{blue} {#1}\else{#1}\fi}\xspace}
\newcommand{\green}[1]{{\ifcomments\color{green} {#1}\else{#1}\fi}\xspace}

\newcommand{\repl}[2]{{\ifcomments{\color{red} \sout{#1}}{\color{blue} {\xspace #2}}\else{#2}\fi}}
%\newcommand{\repl}[2]{{\color{red} \sout{#1}\xspace{\color{blue} {#2}}\else{#2}\fi}\xspace}

\newcommand{\initcomment}[2]{%
	\expandafter\newcommand\csname#1\endcsname{%
		\def\thiscommentname{#1}%
		\definecolor{col}{rgb}{#2}%
		\def\thiscommentcolor{col}%
}}

% initcomment Name RGB-color
\initcomment{Philipp}{0, 0.5, 0}

%\renewcommand{\comment}[1]{{\ifcomments{\color{red} {#1}}{}\fi}\xspace}

\renewcommand{\comment}[2][\nobody]{
	\ifdefined#1
	{\ifcomments{#1 \expandafter\color{\thiscommentcolor}{\thiscommentname: #2}}{}\fi}\xspace
	\else
	{\ifcomments{\color{red} {#2}}{}\fi}\xspace
	\fi
}

\newcommand{\zapf}{ZaPF\xspace}

\let\oldgrqq=\grqq
\def\grqq{\oldgrqq\xspace}

\setlength{\parskip}{.6em}
\setlength{\parindent}{0mm}

%\usepackage{geometry}
%\geometry{left=2.5cm, right=2.5cm, top=2.5cm, bottom=3.5cm}

% \renewcommand{\familydefault}{\sfdefault}




\begin{document}

\hspace{0.87\textwidth}
\begin{minipage}{120pt}
	\vspace{-1.8cm}
	\includegraphics[width=80pt]{../logos/logo.pdf}
	\centering
	\small Zusammenkunft aller Physik-Fachschaften
\end{minipage}

\begin{center}
  \huge{Resolution zum Bayerischen Gesetzentwurf zur Förderung der Bundeswehr}\vspace{.25\baselineskip}\\
  \normalsize
\end{center}
\vspace{1cm}

%%%% Metadaten %%%%

%\paragraph{Adressierte:} [VERMUTLICH, DER ANTRAG IST KAPUTT] Bayrische Landesregierung (poststelle@stk.bayern.de), Bayrisches Kultusministerium (poststelle@stmuk.bayern.de), Bayrisches Wissenschaftsministerium (…), GEW (Bayern) (info@gew-bayern.de), Bildungspolitische Sprecher*innen der Landtagsfraktionen (inkl. AfD?) (info@gruene-fraktion-bayern.de, info@bayernspd-landtag.de, geschaeftsstelle@afdbayern.de), Bildungsministerium (bundesweit) (bmbf@bmbf.bund.de), alle Hochschulpräsidien in Bayern 

%\paragraph{Antragstellende:} xxx

%%%% Text des Antrages zur veröffentlichung %%%%

%\section*{Antragstext}

Die ZaPF lehnt den Gesetzentwurf zur Förderung der Bundeswehr in Bayern ab und fordert eine klare Trennung zwischen Bildungseinrichtungen, Wissenschaft und dem Militär.

\paragraph{Zur Änderung des Bayerischen Hochschulinnovationsgesetzes} \mbox{}\\
Wie die ZaPF in der Vergangenheit bereits gefordert hat\footnote{\url{https://zapf.wiki/images/0/0a/Verantwortung_SoSe17.pdf}}, ist es notwendig, dass die Hochschulen einen Beitrag zu einer gerechten, nachhaltigen, friedlichen und demokratischen Welt leisten. Dies steht direkt im Widerspruch zu §1 Abs. 2 des Gesetzesentwurfs zur Förderung der Bundeswehr in Bayern. Wettrüsten, das als \glqq Sicherheitspolitik\grqq getarnt wird, führt unweigerlich zu zunehmenden globalen Spannungen. Im Kontrast dazu fordern wir daher, dass sich bayerische Hochschulen in ihrer Grundordnung eine Zivilklausel geben, um damit ihrer gesellschaftlichen Verantwortung nachzukommen. Weiterhin wird es der Bundeswehr durch §1 Abs. 1 des Gesetzesentwurf ermöglicht, eine Kooperation mit Hochschulen zu erzwingen, ohne dass die Hochschulen dazu in der Lage sind, dies abzulehnen. Eine erzwungene Kooperation mit der Bundeswehr wird einen kritischen Diskurs an Hochschulen über die Bundeswehr einschränken. Die Hochschulen sind aktuell bereits viel zu stark auf Drittmittel angewiesen und die Fokussierung öffentlicher Mittel auf militärische Zwecke wird das Problem dieser Abhängigkeit nur verstärken. Die ZaPF sieht hierin einen Eingriff in die Freiheit der Forschung und Lehre an Hochschulen, welche in Art. 5 Abs. 3 des Grundgesetz verankert ist.

\paragraph{Zur Änderung des Bayerischen Gesetzes über das Erziehungs- und Unterrichtswesen} \mbox{}\\
Eine zentrale Aufgabe der (politischen) Bildung in Schulen ist es, Schüler*innen im Sinne einer friedlichen und nachhaltigen Welt zu erziehen, sie zu ermündigen, sich eine eigene Meinung zu bilden und dafür einzustehen. Die Bundeswehr ist in diesem Prozess keinesfalls neutral, weshalb eine Verpflichtung der Schulen, dabei mit der Bundeswehr zusammenzuarbeiten, diesem Ziel entgegen steht. Die bevorzugte Stellung der Bundeswehr gegenüber anderen Organisationen/Ansichten, wie sie durch den Gesetzesentwurf herbeigeführt wird, führt damit zu einer unausgeglichenen Darstellung.
Der erste Satz der vorgeschlagenen Gesetzesänderung des §2 BayEUG verpflichtet Schulen und Lehrkräfte zur Zusammenarbeit mit Jugendoffizier*innen. Dies sehen wir als schwerwiegenden Eingriff in die Unterrichtsgestaltung der bayerischen Lehrkräfte. Wir halten die Lehrkräfte durch ihre fundierte Ausbildung befähigt, selbstständig zu entscheiden, ob das Einladen von Gästen an die Schule im Sinne ihres Unterrichts ist. Wir teilen die Auffassung der GEW Bayern, dass Bundeswehr-Jugendoffizier*innen pädagogisch nicht für einen sicherheitspolitischen Exkurs im Sozialkundeunterricht oder für Klassenfahrten von Physik-Kursen qualifizert sind und das Prinzip der Schüler*innenorientierung in der politischen Bildung durch die parteiische Darstellung ohne sichtbare, zivile Gegenposition gefährdet wird.\footnote{\url{https://www.gew-bayern.de/aktuelles/detailseite/zum-gesetzentwurf-zur-foerderung-der-bundeswehr-in-bayern}}

Die ZaPF vertritt auch die Meinung, dass Schulen kein Ort der Rekrutierung Minderjähriger sind, und dem weder durch ein verpflichtendes Einladen von Jugendoffizier*innen gezeichnetes Selbstbild der Bundeswehr noch durch offene Werbung durch Karriereberater*innen der Bundeswehr in den Klassenzimmern Vorschub geleistet werden sollte.
 
%\section*{Begründung}
%yyyy

%\vspace{1cm} 
%
\vfill
\begin{flushright}
	Verabschiedet am 21. Mai 2024 \\
	auf der ZaPF in Kiel.
\end{flushright}

\end{document}
