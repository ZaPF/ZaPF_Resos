\documentclass[DIV=calc]{scrartcl}
\usepackage[utf8]{inputenc}
\usepackage[T1]{fontenc}
\usepackage[ngerman]{babel}
\usepackage{graphicx}
\usepackage[draft, markup=underlined]{changes}
\usepackage{csquotes}
\usepackage{eurosym}

\usepackage{ulem}
%\usepackage[dvipsnames]{xcolor}
\usepackage{paralist}
%\usepackage{fixltx2e}
%\usepackage{ellipsis}
\usepackage[tracking=true]{microtype}

\usepackage{lmodern}              % Ersatz fuer Computer Modern-Schriften
%\usepackage{hfoldsty}

%\usepackage{fourier}             % Schriftart
\usepackage[scaled=0.81]{helvet}     % Schriftart

\usepackage{url}
%\usepackage{tocloft}             % Paket für Table of Contents
\def\UrlBreaks{\do\a\do\b\do\c\do\d\do\e\do\f\do\g\do\h\do\i\do\j\do\k\do\l%
\do\m\do\n\do\o\do\p\do\q\do\r\do\s\do\t\do\u\do\v\do\w\do\x\do\y\do\z\do\0%
\do\1\do\2\do\3\do\4\do\5\do\6\do\7\do\8\do\9\do\-}%

\usepackage{xcolor}
\definecolor{urlred}{HTML}{660000}

\usepackage{hyperref}
\hypersetup{colorlinks=false}

%\usepackage{mdwlist}     % Änderung der Zeilenabstände bei itemize und enumerate
% \usepackage[scale=0.8,colorspec=0.9]{draftwatermark} % Wasserzeichen ``Entwurf''
% \SetWatermarkText{Vorbehaltlich\\redaktioneller\\Änderungen}

\parindent 0pt                 % Absatzeinrücken verhindern
\parskip 12pt                 % Absätze durch Lücke trennen

\setlength{\textheight}{23cm}
\usepackage{fancyhdr}
\pagestyle{fancy}
\fancyhead{} % clear all header fields
\cfoot{}
\lfoot{Zusammenkunft aller Physik-Fachschaften}
\rfoot{www.zapfev.de\\stapf@zapf.in}
\renewcommand{\headrulewidth}{0pt}
\renewcommand{\footrulewidth}{0.1pt}
\newcommand{\gen}{*innen}
\addto{\captionsngerman}{\renewcommand{\refname}{Quellen}}

%%%% Mit-TeXen Kommandoset
\usepackage[normalem]{ulem}
\usepackage{xcolor}
\usepackage{xspace} 

\newcommand{\replace}[2]{
    \sout{\textcolor{blue}{#1}}~\textcolor{blue}{#2}}
\newcommand{\delete}[1]{
    \sout{\textcolor{red}{#1}}}
\newcommand{\add}[1]{
    \textcolor{blue}{#1}}

\newif\ifcomments
\commentsfalse
%\commentstrue

\newcommand{\red}[1]{{\ifcomments\color{red} {#1}\else{#1}\fi}\xspace}
\newcommand{\blue}[1]{{\ifcomments\color{blue} {#1}\else{#1}\fi}\xspace}
\newcommand{\green}[1]{{\ifcomments\color{green} {#1}\else{#1}\fi}\xspace}

\newcommand{\repl}[2]{{\ifcomments{\color{red} \sout{#1}}{\color{blue} {\xspace #2}}\else{#2}\fi}}
%\newcommand{\repl}[2]{{\color{red} \sout{#1}\xspace{\color{blue} {#2}}\else{#2}\fi}\xspace}

\newcommand{\initcomment}[2]{%
	\expandafter\newcommand\csname#1\endcsname{%
		\def\thiscommentname{#1}%
		\definecolor{col}{rgb}{#2}%
		\def\thiscommentcolor{col}%
}}

% initcomment Name RGB-color
\initcomment{Philipp}{0, 0.5, 0}

%\renewcommand{\comment}[1]{{\ifcomments{\color{red} {#1}}{}\fi}\xspace}

\renewcommand{\comment}[2][\nobody]{
	\ifdefined#1
	{\ifcomments{#1 \expandafter\color{\thiscommentcolor}{\thiscommentname: #2}}{}\fi}\xspace
	\else
	{\ifcomments{\color{red} {#2}}{}\fi}\xspace
	\fi
}

\newcommand{\zapf}{ZaPF\xspace}

\let\oldgrqq=\grqq
\def\grqq{\oldgrqq\xspace}

\setlength{\parskip}{.6em}
\setlength{\parindent}{0mm}

%\usepackage{geometry}
%\geometry{left=2.5cm, right=2.5cm, top=2.5cm, bottom=3.5cm}

% \renewcommand{\familydefault}{\sfdefault}




\begin{document}

\hspace{0.87\textwidth}
\begin{minipage}{120pt}
	\vspace{-1.8cm}
	\includegraphics[width=80pt]{../logos/logo.pdf}
	\centering
	\small Zusammenkunft aller Physik-Fachschaften
\end{minipage}

\begin{center}
  \huge{Resolution zu gendergerechter Sprache}\vspace{.25\baselineskip}\\
  \normalsize
\end{center}
\vspace{1cm}

%%%% Metadaten %%%%

%\paragraph{Adressierte:} Alle Landtagsfraktionen der anderen Bundesländer (außer AFD), Alle Fraktionen des bayrischen Landtags (außer AFD), BMBF

%Zur Info an: BLLV (bayrischer Lehrer- und Lehrerinnen-Verband), GEW, Metafa, Bufata Chemie


%\paragraph{Antragstellende:} xxx

%%%% Text des Antrages zur veröffentlichung %%%%

%\section*{Antragstext}

Die ZaPF kritisiert die neuesten Einschränkungen der gendergerechten Sprache im
bayerischen Landesgesetz sowie ähnliche Ansätze in anderen Bundesländern aufs
Schärfste. Wir sehen dies als Teil einer problematischen Entwicklung in Deutschland
und werten es als einen aufgezwungenen Eingriff in die Sprache. Wir
verweisen hierbei auch auf das
Kurzgutachten der Antidiskriminierungsstelle des Bundes\footnote{\url{https://www.antidiskriminierungsstelle.de/SharedDocs/aktuelles/DE/2024/20240513_gutachten_genderverbote.html}} und die Positionierung
der BuFaTa Chemie\footnote{\url{https://bufata-chemie.org/stellungnahme-der-bufatachemie-zum-genderverbot-in-bayern/}}.

Die Sensibilisierung junger Menschen für eine vielfältige Gesellschaft wird von uns
als unerlässlich erachtet und deswegen ist eine Sprache, die alle Geschlechter mit
einbezieht, sehr wichtig. Daher sprechen wir uns für eine gendergerechte Sprache aus,
die alle Menschen inkludiert. Die Berücksichtigung von Menschen mit Sprachbarrieren,
Lernschwierigkeiten und Behinderungen ist dabei essenziell. Verbote halten wir jedoch
für den vollkommen falschen Weg.

Zum 01.04.2024 wurde in Bayern eine Regelung zur gendergerechten Sprache erlassen,
die mehrgeschlechtliche Schreibweisen durch Wortbinnenzeichen als unzulässig
deklariert. Davon sind neben den Behörden auch die staatlichen Schulen und
Hochschulen betroffen. Diese populistische Maßnahme ist nicht nur ein Eingriff in die
Wahl der eigenen Sprache, sondern mit ihr wird aktiv der Sichtbarkeit der Vielfalt
von Geschlechtern entgegengewirkt. Die Rücksicht auf seheingeschränkte und -behinderte Menschen bezüglich der Nutzung von Wortbinnenzeichen begrüßen wir, um eine
uneingeschränkte Teilhabe an der Gesellschaft zu ermöglichen. Eine Einbeziehung
zulasten anderer Minderheiten und ein damit verbundenen Ausspielen dieser
gegeneinander ist jedoch keinesfalls anzustreben. Oberstes Ziel sollte eine
Beteiligung aller Minderheiten sein und eine gemeinsame Bemühung um Barrierefreiheit.

Wir wehren uns gegen jede Form der Diskriminierung, insbesondere auch der von trans*,
inter* und nichtbinären Personen. Verbote, die diese Gruppen unsichtbar machen
möchten, kritisieren wir daher klar und deutlich. Anordnungen, die zurück zu einer
binären Gesellschaft führen, sind zudem ein Nährboden für rechte und rechtsextreme
Tendenzen und als solche in einer wehrhaften Demokratie nicht tolerierbar.

%\section*{Begründung}
%yyyy

%\vspace{1cm} 
%
\vfill
\begin{flushright}
	Verabschiedet am 21. Mai 2024 \\
	auf der ZaPF in Kiel.
\end{flushright}

\end{document}
