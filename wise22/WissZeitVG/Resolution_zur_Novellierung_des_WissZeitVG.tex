\documentclass[DIV=calc]{scrartcl}
\usepackage[utf8]{inputenc}
\usepackage[T1]{fontenc}
\usepackage[ngerman]{babel}
\usepackage{graphicx}
\usepackage[draft, markup=underlined]{changes}
\usepackage{csquotes}
\usepackage{eurosym}

\usepackage{ulem}
%\usepackage[dvipsnames]{xcolor}
\usepackage{paralist}
%\usepackage{fixltx2e}
%\usepackage{ellipsis}
\usepackage[tracking=true]{microtype}

\usepackage{lmodern}              % Ersatz fuer Computer Modern-Schriften
%\usepackage{hfoldsty}

%\usepackage{fourier}             % Schriftart
\usepackage[scaled=0.81]{helvet}     % Schriftart

\usepackage{url}
%\usepackage{tocloft}             % Paket für Table of Contents
\def\UrlBreaks{\do\a\do\b\do\c\do\d\do\e\do\f\do\g\do\h\do\i\do\j\do\k\do\l%
\do\m\do\n\do\o\do\p\do\q\do\r\do\s\do\t\do\u\do\v\do\w\do\x\do\y\do\z\do\0%
\do\1\do\2\do\3\do\4\do\5\do\6\do\7\do\8\do\9\do\-}%

\usepackage{xcolor}
\definecolor{urlred}{HTML}{660000}

\usepackage{hyperref}
\hypersetup{colorlinks=false}

%\usepackage{mdwlist}     % Änderung der Zeilenabstände bei itemize und enumerate
% \usepackage{draftwatermark} % Wasserzeichen ``Entwurf''
% \SetWatermarkText{Antrag}

\parindent 0pt                 % Absatzeinrücken verhindern
\parskip 12pt                 % Absätze durch Lücke trennen

\setlength{\textheight}{23cm}
\usepackage{fancyhdr}
\pagestyle{fancy}
\fancyhead{} % clear all header fields
\cfoot{}
\lfoot{Zusammenkunft aller Physik-Fachschaften}
\rfoot{www.zapfev.de\\stapf@zapf.in}
\renewcommand{\headrulewidth}{0pt}
\renewcommand{\footrulewidth}{0.1pt}
\newcommand{\gen}{*innen}
\addto{\captionsngerman}{\renewcommand{\refname}{Quellen}}

%%%% Mit-TeXen Kommandoset
\usepackage[normalem]{ulem}
\usepackage{xcolor}
\usepackage{xspace} 

\newcommand{\replace}[2]{
    \sout{\textcolor{blue}{#1}}~\textcolor{blue}{#2}}
\newcommand{\delete}[1]{
    \sout{\textcolor{red}{#1}}}
\newcommand{\add}[1]{
    \textcolor{blue}{#1}}

\newif\ifcomments
\commentsfalse
%\commentstrue

\newcommand{\red}[1]{{\ifcomments\color{red} {#1}\else{#1}\fi}\xspace}
\newcommand{\blue}[1]{{\ifcomments\color{blue} {#1}\else{#1}\fi}\xspace}
\newcommand{\green}[1]{{\ifcomments\color{green} {#1}\else{#1}\fi}\xspace}

\newcommand{\repl}[2]{{\ifcomments{\color{red} \sout{#1}}{\color{blue} {\xspace #2}}\else{#2}\fi}}
%\newcommand{\repl}[2]{{\color{red} \sout{#1}\xspace{\color{blue} {#2}}\else{#2}\fi}\xspace}

\newcommand{\initcomment}[2]{%
	\expandafter\newcommand\csname#1\endcsname{%
		\def\thiscommentname{#1}%
		\definecolor{col}{rgb}{#2}%
		\def\thiscommentcolor{col}%
}}

% initcomment Name RGB-color
\initcomment{Philipp}{0, 0.5, 0}

%\renewcommand{\comment}[1]{{\ifcomments{\color{red} {#1}}{}\fi}\xspace}

\renewcommand{\comment}[2][\nobody]{
	\ifdefined#1
	{\ifcomments{#1 \expandafter\color{\thiscommentcolor}{\thiscommentname: #2}}{}\fi}\xspace
	\else
	{\ifcomments{\color{red} {#2}}{}\fi}\xspace
	\fi
}

\newcommand{\zapf}{ZaPF\xspace}

\let\oldgrqq=\grqq
\def\grqq{\oldgrqq\xspace}

\setlength{\parskip}{.6em}
\setlength{\parindent}{0mm}

%\usepackage{geometry}
%\geometry{left=2.5cm, right=2.5cm, top=2.5cm, bottom=3.5cm}

% \renewcommand{\familydefault}{\sfdefault}


%\usepackage{draftwatermark}
%\SetWatermarkText{ENTWURF}
%\SetWatermarkScale{1}

\begin{document}

\hspace{0.87\textwidth}
\begin{minipage}{120pt}
	\vspace{-1.8cm}
	\includegraphics[width=80pt]{logo.pdf}
	\centering
	\small Zusammenkunft aller Physik-Fachschaften
\end{minipage}

\begin{center}
  \huge{Resolution zur Novellierung des WissZeitVG}\vspace{.25\baselineskip}\\
  \normalsize
\end{center}
\vspace{1cm}

%%%% Metadaten %%%%

%\paragraph{Adressierte:} 

%\paragraph{Antragstellende:} Vorname, Nachname (Uni)
%%%% Text des Antrages zur veröffentlichung %%%%

%\section*{Antragstext}

Aktuell sind im deutschen Wissenschaftssystem 81\% der sozialversicherungspflichtig Beschäftigten befristet in Arbeitsverhältnisses angestellt, die geprägt sind von Kettenbefristungen, fehlender Planungssicherheit, zu hoher Arbeitsbelastung und Abhängigkeiten. Berufswege in der Wissenschaft sind also durch prekäre Beschäftigungsverhältnisse gekennzeichnet, was u.a. zu Abwanderungen aus dem deutschen Wissenschaftsbetrieb führt. Damit kann keine Kontinuität in Lehre und Forschung an den wissenschaftlichen Einrichtungen gewährleistet werden, was dem Innovationsstandort Deutschland langfristig schadet. Hier hat auch die letzte Novelle des Wissenschaftszeitvertragsgesetzes (WissZeitVG) keine signifikanten Verbesserungen gebracht.\footnote{Evaluation des novellierten Wissenschaftszeitvertragsgesetzes 2022}\\

Deswegen hat die ZaPF folgende Forderungen zur Novellierung des Wissenschaftszeitvertragsgesetzes:

\section{Studentische Beschäftigte}
Studentische Beschäftigte machen an den Hochschulen einen signifikanten Anteil der wissenschaftlichen Mitarbeitenden aus. Die Hilfstätigkeiten, die sie gemäß WissZeitVG in Lehre und Forschung erbringen, sind oftmals für die dauerhafte Aufrechterhaltung des Wissenschafts- und Lehrbetriebes unentbehrlich. Gerade Studierende, die oftmals auf eine Beschäftigung während des Studiums angewiesen sind, muss Planungssicherheit und eine Perspektive für die Bestreitung ihres Lebensunterhalts geboten werden.\\
Die ZaPF fordert daher, dass studentischen Beschäftigten eine Mindestvertragslaufzeit von zwei Jahren angeboten werden muss. Außerdem muss die bisherigen Höchstbefristungsdauer von sechs Jahren abgeschafft werden.

\section{Absicherung der Promotionsphase}
Die Zeit der Promotion ist derzeit oftmals von Kettenbefristungen geprägt, deren Vertragslaufzeiten nicht im Verhältnis zur Dauer einer Promotion stehen. Genau jene Sollregelung in § 2 Abs. 1 Satz 3 WissZeitVG entfaltet hier keinerlei Wirkung. Die ZaPF fordert daher, dass der Gesetzgeber die Beschäftigungsdauer wirksam an die durchschnittliche Promotionsdauer anpassen muss.\\
Aus unserer Sicht ergeben sich zwei Möglichkeiten, dies umzusetzen:
\begin{itemize}
\item Zum einen kann durch eine Zweckbefristung die Vertragslaufzeit an das Erreichen des Qualifikationsziels Promotion gekoppelt werden.
\item Alternativ könnte zu Beginn der Promotion eine Mindestvertragslaufzeit von vier Jahren festgeschrieben werden mit der Möglichkeit der Verlängerung um zunächst zwei Jahre. Wir legen dieser Forderung die durchschnittliche Promotionsdauer in Deutschland von 5,7 Jahren zugrunde.\footnote{Beiträge zur Hochschulforschung, Heft 1, 24. Jahrgang, 2002}\footnote{Bundesbericht Wissenschaftlicher Nachwuchs 2021}
\end{itemize}
Grundsätzlich soll Promovierenden ein Arbeitsvertrag angeboten werden, in dem ein Qualifikationsziel vereinbart werden muss. Für die Promotionsphase ist aus unserer Sicht allein der Abschluss der Promotion das geeignete Qualifikationsziel. In Anlehnung an den europäischen Rechtsrahmen soll die Promotion die höchste erreichbare wissenschaftliche Qualifikation sein. In diesem Zusammenhang fordern wir den Gesetzgeber auf, den Begriff der Qualifikation beziehungsweise des Qualifikationsziels in diesem Sinne legal zu definieren.

\section{Dauerstellen für Daueraufgaben nach der Promotionen}
Wissenschaftliche Tätigkeiten sind im Regelfall Daueraufgaben in Lehre und Forschung. Die ZaPF fordert daher, dass Beschäftigte insbesondere nach Abschluss der Promotion unbefristet eingestellt werden sollen.\\
Die Hochschulen sollen dafür ihre Daueraufgaben identifizieren. Folglich müssen die Hochschulen Personalstrukturpläne aufstellen. Daraus sollen Personalentwicklungspläne für Beschäftigte entstehen, die zur langfristigen Beschäftigungs- und Karriereentwicklung genutzt werden.

\section{Familienpolitische Komponente}
Bereits im Sommersemester 2022 hat die ZaPF in ihrer Resolution ''Gleichstellung von durch Drittmittel finanzierten Stellen mit Qualifizierungsstellen'' gefordert, dass die Möglichkeiten des WissZeitVG zur Verlängerung der Höchstbefristungsdauer auch auf überwiegend aus Drittmitteln Beschäftigte angewendet werden.\footnote{\url{https://zapfev.de/resolutionen/sose22/WissZeitVG/WissZeitVG.pdf}}

\section{Tarifsperre}
Es ist aus unserer Sicht unverständlich, warum potentielle Tarifpartner*innen nicht einvernehmlich Rahmenbedingungen für Beschäftigungsverhältnisse in der Wissenschaft aushandeln dürfen.\\
Die ZaPF fordert daher, dass die sogenannte Tarifsperre in § 1 Abs. 1 Satz 2 WissZeitVG entfällt. Perspektivisch soll auch für studentische Beschäftigte ein Tarifvertrag angestrebt werden.


%%%% Text der Begründung für das Plenum %%%%
%\newpage
%\section*{Begründung}




\vspace{1cm} 

\vfill
\begin{flushright}
	Verabschiedet am 13. November 2022 \\
	auf der ZaPF in Hamburg.
\end{flushright}

\end{document}

