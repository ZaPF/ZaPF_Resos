\documentclass[DIV=calc]{scrartcl}
\usepackage[utf8]{inputenc}
\usepackage[T1]{fontenc}
\usepackage[ngerman]{babel}
\usepackage{graphicx}
\usepackage[draft, markup=underlined]{changes}
\usepackage{csquotes}
\usepackage{eurosym}

\usepackage{ulem}
%\usepackage[dvipsnames]{xcolor}
\usepackage{paralist}
\usepackage{fixltx2e}
%\usepackage{ellipsis}
\usepackage[tracking=true]{microtype}

\usepackage{lmodern}              % Ersatz fuer Computer Modern-Schriften
%\usepackage{hfoldsty}

%\usepackage{fourier}             % Schriftart
\usepackage[scaled=0.81]{helvet}     % Schriftart

\usepackage{url}
%\usepackage{tocloft}             % Paket für Table of Contents
\def\UrlBreaks{\do\a\do\b\do\c\do\d\do\e\do\f\do\g\do\h\do\i\do\j\do\k\do\l%
\do\m\do\n\do\o\do\p\do\q\do\r\do\s\do\t\do\u\do\v\do\w\do\x\do\y\do\z\do\0%
\do\1\do\2\do\3\do\4\do\5\do\6\do\7\do\8\do\9\do\-}%

\usepackage{xcolor}
\definecolor{urlred}{HTML}{660000}

\usepackage{hyperref}
\hypersetup{colorlinks=false}

%\usepackage{mdwlist}     % Änderung der Zeilenabstände bei itemize und enumerate
% \usepackage{draftwatermark} % Wasserzeichen ``Entwurf''
% \SetWatermarkText{Antrag}

\parindent 0pt                 % Absatzeinrücken verhindern
\parskip 12pt                 % Absätze durch Lücke trennen

\setlength{\textheight}{23cm}
\usepackage{fancyhdr}
\pagestyle{fancy}
\fancyhead{} % clear all header fields
\cfoot{}
\lfoot{Zusammenkunft aller Physik-Fachschaften}
\rfoot{www.zapfev.de\\stapf@zapf.in}
\renewcommand{\headrulewidth}{0pt}
\renewcommand{\footrulewidth}{0.1pt}
\newcommand{\gen}{*innen}
\addto{\captionsngerman}{\renewcommand{\refname}{Quellen}}

%%%% Mit-TeXen Kommandoset
\usepackage[normalem]{ulem}
\usepackage{xcolor}
\usepackage{xspace} 

\newcommand{\replace}[2]{
    \sout{\textcolor{blue}{#1}}~\textcolor{blue}{#2}}
\newcommand{\delete}[1]{
    \sout{\textcolor{red}{#1}}}
\newcommand{\add}[1]{
    \textcolor{blue}{#1}}

\newif\ifcomments
\commentsfalse
%\commentstrue

\newcommand{\red}[1]{{\ifcomments\color{red} {#1}\else{#1}\fi}\xspace}
\newcommand{\blue}[1]{{\ifcomments\color{blue} {#1}\else{#1}\fi}\xspace}
\newcommand{\green}[1]{{\ifcomments\color{green} {#1}\else{#1}\fi}\xspace}

\newcommand{\repl}[2]{{\ifcomments{\color{red} \sout{#1}}{\color{blue} {\xspace #2}}\else{#2}\fi}}
%\newcommand{\repl}[2]{{\color{red} \sout{#1}\xspace{\color{blue} {#2}}\else{#2}\fi}\xspace}

\newcommand{\initcomment}[2]{%
	\expandafter\newcommand\csname#1\endcsname{%
		\def\thiscommentname{#1}%
		\definecolor{col}{rgb}{#2}%
		\def\thiscommentcolor{col}%
}}

% initcomment Name RGB-color
\initcomment{Philipp}{0, 0.5, 0}

%\renewcommand{\comment}[1]{{\ifcomments{\color{red} {#1}}{}\fi}\xspace}

\renewcommand{\comment}[2][\nobody]{
	\ifdefined#1
	{\ifcomments{#1 \expandafter\color{\thiscommentcolor}{\thiscommentname: #2}}{}\fi}\xspace
	\else
	{\ifcomments{\color{red} {#2}}{}\fi}\xspace
	\fi
}

\newcommand{\zapf}{ZaPF\xspace}

\let\oldgrqq=\grqq
\def\grqq{\oldgrqq\xspace}

\setlength{\parskip}{.6em}
\setlength{\parindent}{0mm}

%\usepackage{geometry}
%\geometry{left=2.5cm, right=2.5cm, top=2.5cm, bottom=3.5cm}

% \renewcommand{\familydefault}{\sfdefault}




\begin{document}


\hspace{0.87\textwidth}
\begin{minipage}{120pt}
	\vspace{-1.8cm}
	\includegraphics[width=80pt]{logo.pdf}
	\centering
	\small Zusammenkunft aller Physik-Fachschaften
\end{minipage}

\begin{center}
  \huge{Resolution zur finanziellen Unterstützung der
Studierendenwerke}\vspace{.25\baselineskip}\\
  \normalsize
\end{center}
\vspace{1cm}

%%%% Text des Antrages zur veröffentlichung %%%%
Die ZaPF fordert die Studierendenwerke auf, Studierende in der aktuell andauernden schwierigen ökonomischen Lage zu entlasten und in dieser Zeit multipler Krisen konsequent zu unterstützen. Hierbei muss insbesondere die Landesregierung die Studierendenwerke mit ausreichenden Mitteln versorgen, damit die Studierendenwerke weiterhin ihre Aufgaben auch in den Bereichen Wohnen und Verpflegung erfüllen können.

Auf Grund der durch die aktuellen Krisen gestiegenen Heiz- und Nebenkosten bangen vielen Studierende um ihren Wohnheimsplatz. Daher fordern wir ein Kündigungsmoratorium für Mietverträge in Studierendenwohnheimen sowie Härtefallregelungen für Studierende, die aufgrund der Krisen in finanzielle Not geraten sind.

Zur gesetzlichen Versorgungspflicht der Studierendenwerke gehört auch der Betrieb der Mensen und Cafeterien, auf die viele Studierenden angewiesen sind, um Zugang zu einem warmen Mittagessen zu haben.
Aus diesem Grund fordern wir die Studierendenwerke auf, sicherzustellen, dass
\begin{itemize}
    \item die Mensapreise nicht steigen,
    \item die Lebensmittelqualität gesichert bleibt,
    \item die Portionsgrößen angemessen sind,
    \item eine ausgewogene Ernährung gewährleistet wird
    \item und, wie vielerorts üblich, mindestens eine vegane Option für jede Mahlzeit zur Verfügung gestellt wird.
\end{itemize}
Für die Studierenden ist ein bezahlbares, warmes Mittagessen wichtig und eine ausgewogene Ernährung sollte nicht unter der aktuellen Lage leiden.

Wir haben Verständnis dafür, dass auch die Studierendenwerke in finanziellen Nöten sind und fordern, wie auch in der letzten \glqq Resolution zu Studierendenwerken\grqq eine bessere Ausfinanzierung der Studierendenwerke durch die Länder\footnote{https://zapfev.de/resolutionen/sose22/Studwerk/Studwerk.pdf}. Die ZaPF fordert Studierendenvertretungen und Studierendenwerke auf, gemeinsam öffentlich für diese Ausfinanzierung einzutreten.

Die Kosten der Krisen dürfen keinesfalls auf Studierende abgewälzt werden!
    
\vspace{1cm} 

\vfill
\begin{flushright}
	Verabschiedet am 13. November 2022 \\
	auf der ZaPF in Hamburg.
\end{flushright}

\end{document}
