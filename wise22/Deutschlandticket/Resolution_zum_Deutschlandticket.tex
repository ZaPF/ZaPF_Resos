\documentclass[DIV=calc]{scrartcl}
\usepackage[utf8]{inputenc}
\usepackage[T1]{fontenc}
\usepackage[ngerman]{babel}
\usepackage{graphicx}
\usepackage[draft, markup=underlined]{changes}
\usepackage{csquotes}
\usepackage{eurosym}

\usepackage{ulem}
%\usepackage[dvipsnames]{xcolor}
\usepackage{paralist}
\usepackage{fixltx2e}
%\usepackage{ellipsis}
\usepackage[tracking=true]{microtype}

\usepackage{lmodern}              % Ersatz fuer Computer Modern-Schriften
%\usepackage{hfoldsty}

%\usepackage{fourier}             % Schriftart
\usepackage[scaled=0.81]{helvet}     % Schriftart

\usepackage{url}
%\usepackage{tocloft}             % Paket für Table of Contents
\def\UrlBreaks{\do\a\do\b\do\c\do\d\do\e\do\f\do\g\do\h\do\i\do\j\do\k\do\l%
\do\m\do\n\do\o\do\p\do\q\do\r\do\s\do\t\do\u\do\v\do\w\do\x\do\y\do\z\do\0%
\do\1\do\2\do\3\do\4\do\5\do\6\do\7\do\8\do\9\do\-}%

\usepackage{xcolor}
\definecolor{urlred}{HTML}{660000}

\usepackage{hyperref}
\hypersetup{colorlinks=false}

%\usepackage{mdwlist}     % Änderung der Zeilenabstände bei itemize und enumerate
% \usepackage{draftwatermark} % Wasserzeichen ``Entwurf''
% \SetWatermarkText{Antrag}

\parindent 0pt                 % Absatzeinrücken verhindern
\parskip 12pt                 % Absätze durch Lücke trennen

\setlength{\textheight}{23cm}
\usepackage{fancyhdr}
\pagestyle{fancy}
\fancyhead{} % clear all header fields
\cfoot{}
\lfoot{Zusammenkunft aller Physik-Fachschaften}
\rfoot{www.zapfev.de\\stapf@zapf.in}
\renewcommand{\headrulewidth}{0pt}
\renewcommand{\footrulewidth}{0.1pt}
\newcommand{\gen}{*innen}
\addto{\captionsngerman}{\renewcommand{\refname}{Quellen}}

%%%% Mit-TeXen Kommandoset
\usepackage[normalem]{ulem}
\usepackage{xcolor}
\usepackage{xspace} 

\newcommand{\replace}[2]{
    \sout{\textcolor{blue}{#1}}~\textcolor{blue}{#2}}
\newcommand{\delete}[1]{
    \sout{\textcolor{red}{#1}}}
\newcommand{\add}[1]{
    \textcolor{blue}{#1}}

\newif\ifcomments
\commentsfalse
%\commentstrue

\newcommand{\red}[1]{{\ifcomments\color{red} {#1}\else{#1}\fi}\xspace}
\newcommand{\blue}[1]{{\ifcomments\color{blue} {#1}\else{#1}\fi}\xspace}
\newcommand{\green}[1]{{\ifcomments\color{green} {#1}\else{#1}\fi}\xspace}

\newcommand{\repl}[2]{{\ifcomments{\color{red} \sout{#1}}{\color{blue} {\xspace #2}}\else{#2}\fi}}
%\newcommand{\repl}[2]{{\color{red} \sout{#1}\xspace{\color{blue} {#2}}\else{#2}\fi}\xspace}

\newcommand{\initcomment}[2]{%
	\expandafter\newcommand\csname#1\endcsname{%
		\def\thiscommentname{#1}%
		\definecolor{col}{rgb}{#2}%
		\def\thiscommentcolor{col}%
}}

% initcomment Name RGB-color
\initcomment{Philipp}{0, 0.5, 0}

%\renewcommand{\comment}[1]{{\ifcomments{\color{red} {#1}}{}\fi}\xspace}

\renewcommand{\comment}[2][\nobody]{
	\ifdefined#1
	{\ifcomments{#1 \expandafter\color{\thiscommentcolor}{\thiscommentname: #2}}{}\fi}\xspace
	\else
	{\ifcomments{\color{red} {#2}}{}\fi}\xspace
	\fi
}

\newcommand{\zapf}{ZaPF\xspace}

\let\oldgrqq=\grqq
\def\grqq{\oldgrqq\xspace}

\setlength{\parskip}{.6em}
\setlength{\parindent}{0mm}

%\usepackage{geometry}
%\geometry{left=2.5cm, right=2.5cm, top=2.5cm, bottom=3.5cm}

% \renewcommand{\familydefault}{\sfdefault}




\begin{document}


\hspace{0.87\textwidth}
\begin{minipage}{120pt}
	\vspace{-1.8cm}
	\includegraphics[width=80pt]{logo.pdf}
	\centering
	\small Zusammenkunft aller Physik-Fachschaften
\end{minipage}

\begin{center}
  \huge{Resolution zum Deutschlandticket}\vspace{.25\baselineskip}\\
  \normalsize
\end{center}
\vspace{1cm}

%%%% Text des Antrages zur veröffentlichung %%%%
Die ZaPF fordert die Bundesregierung und die Landesregierungen auf, für Schüler*innen, Auszubildende und Studierende ein von ihnen bezahlbares Bildungsticket in Form eines ermäßigten Deutschlandtickets einzuführen.\\\\
Wir begrüßen die Vereinheitlichung und Vereinfachung des ÖPNV durch das Deutschlandticket, sehen allerdings für die oben genannten Gruppen die Kosten von \EUR{49} pro Monat weit über dem finanziell tragbaren Rahmen.\\\\
Gerade für sie besteht aber eine starke Abhängigkeit von der Nutzung des ÖPNV. Die Wohnungssituation erlaubt oft keine hinreichende Nähe zwischen Wohn- und Bildungsort, um die Alltagswege zu Fuß oder mit dem Fahrrad zu bestreiten. Oft pendeln sie sogar über Landes- und Verkehrsverbundsgrenzen hinaus.\\\\
Das Deutschlandticket erlaubt es ressourcenschonend weit entfernte Heimatorte zu erreichen und Bildungsexkursionen zu unternehmen. Außerdem ermöglicht es soziale Teilhabe und dient der Vernetzung.\\\\
Damit auch Schüler*innen, Auszubildende und Studierende von den Vorteilen des kürzlich beschlossenen Tickets profitieren können, muss das vorgeschlagene Bildungsticket wesentlich günstiger als \EUR{49} pro Monat sein.\\\\
Für eine faire Lösung halten wir einen Preis von maximal \EUR{15} pro Monat ($\approx$~\EUR{0,50}~pro~Tag).\\\\
Auf Schüler*innen, Studierende und Auszubildende mit Kind muss durch Mitnahmeregelungen Rücksicht genommen werden.\\\\
Für Studierende, deren finanzielle Lage auch den Erwerb des Bildungstickets nicht zulässt, so etwa BAföG-Beziehende, muss es die Möglichkeit für eine Erstattung der Kosten geben.\\\\
Wir verweisen auch auf die Pressemitteilung des fzs\footnote{Pressemitteilung des fzs vom 01.11.2022: \href{https://www.fzs.de/2022/11/01/nicht-den-anschluss-verpassen-studierende-fordern-bundesweites-29e-bildungsticket-jetzt/}{fzs.de/2022/11/01/nicht-den-anschluss-verpassen-studierende-fordern-bundesweites-29e-bildungsticket-jetzt/}}, dessen inhaltlichen Kern wir unterstützen.\\\\
Auch andere Gruppen unserer Gesellschaft in finanziell prekären Verhältnissen dürfen nicht vergessen werden. Auch sie müssen sich die Vorteile des Deutschlandtickets leisten können.\\\\
Langfristiges Ziel muss ein kostenfreier ÖPNV sowie dessen großflächiger Ausbau sein.
    
\vspace{1cm} 

\vfill
\begin{flushright}
	Verabschiedet am 13. November 2022 \\
	auf der ZaPF in Hamburg.
\end{flushright}

\end{document}
