\documentclass[DIV=calc]{scrartcl}
\usepackage[utf8]{inputenc}
\usepackage[T1]{fontenc}
\usepackage[ngerman]{babel}
\usepackage{graphicx}
\usepackage[draft, markup=underlined]{changes}
\usepackage{csquotes}
\usepackage{eurosym}

\usepackage{ulem}
%\usepackage[dvipsnames]{xcolor}
\usepackage{paralist}
\usepackage{fixltx2e}
%\usepackage{ellipsis}
\usepackage[tracking=true]{microtype}

\usepackage{lmodern}              % Ersatz fuer Computer Modern-Schriften
%\usepackage{hfoldsty}

%\usepackage{fourier}             % Schriftart
\usepackage[scaled=0.81]{helvet}     % Schriftart

\usepackage{url}
%\usepackage{tocloft}             % Paket für Table of Contents
\def\UrlBreaks{\do\a\do\b\do\c\do\d\do\e\do\f\do\g\do\h\do\i\do\j\do\k\do\l%
\do\m\do\n\do\o\do\p\do\q\do\r\do\s\do\t\do\u\do\v\do\w\do\x\do\y\do\z\do\0%
\do\1\do\2\do\3\do\4\do\5\do\6\do\7\do\8\do\9\do\-}%

\usepackage{xcolor}
\definecolor{urlred}{HTML}{660000}

\usepackage{hyperref}
\hypersetup{colorlinks=false}

%\usepackage{mdwlist}     % Änderung der Zeilenabstände bei itemize und enumerate
% \usepackage{draftwatermark} % Wasserzeichen ``Entwurf''
% \SetWatermarkText{Antrag}

\parindent 0pt                 % Absatzeinrücken verhindern
\parskip 12pt                 % Absätze durch Lücke trennen

\setlength{\textheight}{23cm}
\usepackage{fancyhdr}
\pagestyle{fancy}
\fancyhead{} % clear all header fields
\cfoot{}
\lfoot{Zusammenkunft aller Physik-Fachschaften}
\rfoot{www.zapfev.de\\stapf@zapf.in}
\renewcommand{\headrulewidth}{0pt}
\renewcommand{\footrulewidth}{0.1pt}
\newcommand{\gen}{*innen}
\addto{\captionsngerman}{\renewcommand{\refname}{Quellen}}

%%%% Mit-TeXen Kommandoset
\usepackage[normalem]{ulem}
\usepackage{xcolor}
\usepackage{xspace} 

\newcommand{\replace}[2]{
    \sout{\textcolor{blue}{#1}}~\textcolor{blue}{#2}}
\newcommand{\delete}[1]{
    \sout{\textcolor{red}{#1}}}
\newcommand{\add}[1]{
    \textcolor{blue}{#1}}

\newif\ifcomments
\commentsfalse
%\commentstrue

\newcommand{\red}[1]{{\ifcomments\color{red} {#1}\else{#1}\fi}\xspace}
\newcommand{\blue}[1]{{\ifcomments\color{blue} {#1}\else{#1}\fi}\xspace}
\newcommand{\green}[1]{{\ifcomments\color{green} {#1}\else{#1}\fi}\xspace}

\newcommand{\repl}[2]{{\ifcomments{\color{red} \sout{#1}}{\color{blue} {\xspace #2}}\else{#2}\fi}}
%\newcommand{\repl}[2]{{\color{red} \sout{#1}\xspace{\color{blue} {#2}}\else{#2}\fi}\xspace}

\newcommand{\initcomment}[2]{%
	\expandafter\newcommand\csname#1\endcsname{%
		\def\thiscommentname{#1}%
		\definecolor{col}{rgb}{#2}%
		\def\thiscommentcolor{col}%
}}

% initcomment Name RGB-color
\initcomment{Philipp}{0, 0.5, 0}

%\renewcommand{\comment}[1]{{\ifcomments{\color{red} {#1}}{}\fi}\xspace}

\renewcommand{\comment}[2][\nobody]{
	\ifdefined#1
	{\ifcomments{#1 \expandafter\color{\thiscommentcolor}{\thiscommentname: #2}}{}\fi}\xspace
	\else
	{\ifcomments{\color{red} {#2}}{}\fi}\xspace
	\fi
}

\newcommand{\zapf}{ZaPF\xspace}

\let\oldgrqq=\grqq
\def\grqq{\oldgrqq\xspace}

\setlength{\parskip}{.6em}
\setlength{\parindent}{0mm}

%\usepackage{geometry}
%\geometry{left=2.5cm, right=2.5cm, top=2.5cm, bottom=3.5cm}

% \renewcommand{\familydefault}{\sfdefault}




\begin{document}


\hspace{0.87\textwidth}
\begin{minipage}{120pt}
	\vspace{-1.8cm}
	\includegraphics[width=80pt]{logo.pdf}
	\centering
	\small Zusammenkunft aller Physik-Fachschaften
\end{minipage}

\begin{center}
  \huge{Resolution zur Indexmiete}\vspace{.25\baselineskip}\\
  \normalsize
\end{center}
\vspace{1cm}

%%%% Text des Antrages zur veröffentlichung %%%%
Die ZaPF fordert die Bundesregierung und die Landesregierungen auf, in den Gesetzen der Mietpreisbremse die Indexmieten mit den anderen Mietformen gleichzustellen und darüber hinaus gemeinsam schnell Lösungen für die Sicherung von bezahlbarem Wohnraum für Studierende – insbesondere im Kontext Zeiten starker Teuerungen – zu erarbeiten und umzusetzen.\\\\
Während Indexmietverträge bei der Vermietung von Wohnraum an Privatpersonen in den letzten Jahren verhältnismäßig unproblematisch waren und dementsprechend bei stark steigenden Vergleichsmieten und geringer Teuerung wenig Beachtung im öffentlichen Diskurs erfahren haben, so eskaliert die Situation aktuell für viele Studierende angesichts der hohen Inflationsraten doch zusehends.\\\\
Durch die Ausnahme der Indexmiete aus dem seit 2015 in Kraft stehenden Gesetz zur Mietpreisbremse, kann sich der individuelle Mietpreis schnell weit über das Niveau der ortsüblichen Vergleichsmieten hinaus entwickeln. Eine Obergrenze findet die Preisentwicklung erst beim Eintritt in den Bereich des Mietwuchers. Dieser ist für Mietende auf der einen Seite nur schwer nachweis- und rechtlich anfechtbar\footnote{Der Interessenverband Mieterschutz e. V. sieht große Hürden für das Vorgehen von Mietenden gegen Mietwucher: \href{https://www.iv-mieterschutz.de/mietrecht/mietvertrag/miethoehe/}{iv-mieterschutz.de/mietrecht/mietvertrag/miethoehe/}} und liegt auf der anderen Seite vor allem mit 50\% auch weit über den bei Vergleichs- und Staffelmieten durch die Mietpreisbremse maximal zulässigen 10\% Überhöhung gegenüber der ortsüblichen Vergleichsmieten.\\\\
Gerade für Studierende ist dabei ein kostenintensiver Umzug auf einem ohnehin extrem verknappten Wohnungsmarkt in ein neues Mietverhältnis keine Option. Insbesondere angesichts der stark gestiegenen Energiepreise und der damit einhergehenden Notwendigkeit Rücklagen für kommende Nachzahlungen zu bilden, was für sich genommen angesichts der Preisentwicklungen für die meisten Studierenden nahezu unmöglich ist\footnote{Das Konsumklima ist auf einem Allzeittief (\href{https://www.zeit.de/news/2022-07/27/konsumklima-in-deutschland-sinkt-auf-allzeittief}{zeit.de/news/2022-07/27/konsumklima-in-deutschland-sinkt-auf-allzeittief}), gerade für Studierende gibt es wenig bis keinen Spielraum noch stärker bei Konsumausgaben einzusparen.}. Auch ziehen immer weniger Vermietende für ein Mietverhältnis Studierende in Betracht\footnote{Erfahrungswerte aus Berichten von Physikstudierenden aus ganz Deutschland gegenüber ihren Fachschaften.}, da ihre Angst vor unerfüllten Zahlungsverpflichtungen durch die -- angesichts stark steigender Preise -- extrem angespannte finanzielle Situation der Studierenden wächst.\\\\
Durch Indexmietverträge können zur Miete wohnende Studierende sowie andere Mietende in ihrem bestehenden Mietvertrag plötzlich gezwungen sein, eine Mietsumme zu entrichten, die Vermietende nicht einmal bei einer Neuvermietung auf dem Wohnungsmarkt für ihren Wohnraum erzielen könnten. Für ein so knappes und für jeden Menschen essenzielles Rechtsgut wie bezahlbaren Wohnraum, welches entsprechend auch in besonderem Maße schützenswert ist, ist eben das kein akzeptabler Zustand.\\\\
Um größere Verwerfungen zu verhindern, muss jetzt schnell gehandelt werden!

    
\vspace{1cm} 

\vfill
\begin{flushright}
	Verabschiedet am 13. November 2022 \\
	auf der ZaPF in Hamburg.
\end{flushright}

\end{document}
