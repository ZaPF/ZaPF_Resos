\documentclass[DIV=calc]{scrartcl}
\usepackage[utf8]{inputenc}
\usepackage[T1]{fontenc}
\usepackage[ngerman]{babel}
\usepackage{graphicx}
\usepackage[draft, markup=underlined]{changes}
\usepackage{csquotes}
\usepackage{eurosym}

\usepackage{ulem}
%\usepackage[dvipsnames]{xcolor}
\usepackage{paralist}
%\usepackage{fixltx2e}
%\usepackage{ellipsis}
\usepackage[tracking=true]{microtype}

\usepackage{lmodern}              % Ersatz fuer Computer Modern-Schriften
%\usepackage{hfoldsty}

%\usepackage{fourier}             % Schriftart
\usepackage[scaled=0.81]{helvet}     % Schriftart

\usepackage{url}
%\usepackage{tocloft}             % Paket für Table of Contents
\def\UrlBreaks{\do\a\do\b\do\c\do\d\do\e\do\f\do\g\do\h\do\i\do\j\do\k\do\l%
\do\m\do\n\do\o\do\p\do\q\do\r\do\s\do\t\do\u\do\v\do\w\do\x\do\y\do\z\do\0%
\do\1\do\2\do\3\do\4\do\5\do\6\do\7\do\8\do\9\do\-}%

\usepackage{xcolor}
\definecolor{urlred}{HTML}{660000}

\usepackage{hyperref}
\hypersetup{colorlinks=false}

%\usepackage{mdwlist}     % Änderung der Zeilenabstände bei itemize und enumerate
% \usepackage{draftwatermark} % Wasserzeichen ``Entwurf''
% \SetWatermarkText{Antrag}

\parindent 0pt                 % Absatzeinrücken verhindern
\parskip 12pt                 % Absätze durch Lücke trennen

\setlength{\textheight}{23cm}
\usepackage{fancyhdr}
\pagestyle{fancy}
\fancyhead{} % clear all header fields
\cfoot{}
\lfoot{Zusammenkunft aller Physik-Fachschaften}
\rfoot{www.zapfev.de\\stapf@zapf.in}
\renewcommand{\headrulewidth}{0pt}
\renewcommand{\footrulewidth}{0.1pt}
\newcommand{\gen}{*innen}
\addto{\captionsngerman}{\renewcommand{\refname}{Quellen}}

%%%% Mit-TeXen Kommandoset
\usepackage[normalem]{ulem}
\usepackage{xcolor}
\usepackage{xspace} 

\newcommand{\replace}[2]{
    \sout{\textcolor{blue}{#1}}~\textcolor{blue}{#2}}
\newcommand{\delete}[1]{
    \sout{\textcolor{red}{#1}}}
\newcommand{\add}[1]{
    \textcolor{blue}{#1}}

\newif\ifcomments
\commentsfalse
%\commentstrue

\newcommand{\red}[1]{{\ifcomments\color{red} {#1}\else{#1}\fi}\xspace}
\newcommand{\blue}[1]{{\ifcomments\color{blue} {#1}\else{#1}\fi}\xspace}
\newcommand{\green}[1]{{\ifcomments\color{green} {#1}\else{#1}\fi}\xspace}

\newcommand{\repl}[2]{{\ifcomments{\color{red} \sout{#1}}{\color{blue} {\xspace #2}}\else{#2}\fi}}
%\newcommand{\repl}[2]{{\color{red} \sout{#1}\xspace{\color{blue} {#2}}\else{#2}\fi}\xspace}

\newcommand{\initcomment}[2]{%
	\expandafter\newcommand\csname#1\endcsname{%
		\def\thiscommentname{#1}%
		\definecolor{col}{rgb}{#2}%
		\def\thiscommentcolor{col}%
}}

% initcomment Name RGB-color
\initcomment{Philipp}{0, 0.5, 0}

%\renewcommand{\comment}[1]{{\ifcomments{\color{red} {#1}}{}\fi}\xspace}

\renewcommand{\comment}[2][\nobody]{
	\ifdefined#1
	{\ifcomments{#1 \expandafter\color{\thiscommentcolor}{\thiscommentname: #2}}{}\fi}\xspace
	\else
	{\ifcomments{\color{red} {#2}}{}\fi}\xspace
	\fi
}

\newcommand{\zapf}{ZaPF\xspace}

\let\oldgrqq=\grqq
\def\grqq{\oldgrqq\xspace}

\setlength{\parskip}{.6em}
\setlength{\parindent}{0mm}

%\usepackage{geometry}
%\geometry{left=2.5cm, right=2.5cm, top=2.5cm, bottom=3.5cm}

% \renewcommand{\familydefault}{\sfdefault}


%\usepackage{draftwatermark}
%\SetWatermarkText{ENTWURF}
%\SetWatermarkScale{1}

\begin{document}

\hspace{0.87\textwidth}
\begin{minipage}{120pt}
	\vspace{-1.8cm}
	\includegraphics[width=80pt]{logo.pdf}
	\centering
	\small Zusammenkunft aller Physik-Fachschaften
\end{minipage}

\begin{center}
  \huge{Verhaltenskodex der ZaPF }\vspace{.25\baselineskip}\\
  \normalsize
\end{center}
\vspace{1cm}

%%%% Metadaten %%%%

%\paragraph{Adressierte:} 

%\paragraph{Antragstellende:} Vorname, Nachname (Uni)
%%%% Text des Antrages zur veröffentlichung %%%%

%\section*{Antragstext}

\begin{quote}
	"Die ZaPF ist ein freies Forum von und für Physikstudika. Sie bietet eine
	sichere Umgebung für Teilnehmika unabhängig ihrer Alter, Geschlechter,
	sexueller Identitäten oder Orientierungen, physischen Erscheinungen und
	Befähigungen, Studiengänge, Lebensumstände sowie politischer oder
	religiöser Überzeugungen. Aus diesem Grund kann diskriminierendes,
	ausschließendes und grenzüberschreitendes Verhalten in jeglicher Form
	nicht toleriert werden."
\end{quote}

Stellungnahme der Zusammenkunft aller Physik-Fachschaften gegen
  Diskriminierung, Ausschließung und grenzüberschreitendes Verhalten,
  ``beschlossen''\footnote{Da das Abschlussplenumsprotokoll aus Wien
    verschwunden ist, kann der Beschluss historisch nicht belegt werden.}
  in Wien.

\vspace{1cm}

Wir wollen die ZaPF gemeinsam in einem respektvollen Miteinander
gestalten.

Wir wollen inklusive, diskriminierungsfreie und offene Kommunikation
miteinander.

Beispielhaft als nicht abgeschlossene Liste:

\begin{itemize}
%\tightlist
\item
  Wir freuen uns über neue Menschen auf der ZaPF und möchten eine
  Umgebung schaffen, in der wir voneinander lernen können.
\item
  Wir schätzen Meinungspluralismus und ermuntern Personen marginalisierter Gruppen, sich aktiv einzubringen.
\item
  Wir kommunizieren respektvoll miteinander, auch wenn wir verschiedener
  Meinung sind.
\item
  Wir lassen einander ausreden und hören uns zu.
\item
  Wir gehen miteinander empathisch um.
\item
  Wir akzeptieren wenn Menschen Fragen nicht beantworten wollen.
\item
  Wir haben körperlichen Kontakt nur mit expliziter Zustimmung,
  Nichtzustimmung wird ohne Begrüngung akzeptiert und nicht hinterfragt.
\item
  Wir schaffen Sicherheit und diskutieren in einer entspannten
  Atmosphäre miteinander.
\item
  Wir ermutigen rücksichtsvolles Fragen.
\item
  Wir wünschen uns couragiertes Verhalten.
\item
  Wir schützen unsere Grenzen und halten uns an die von anderen
  gesetzten Grenzen. Du selbst bestimmt, was für dich grenzüberschreitend
  ist.
\end{itemize}

Die ZaPFika engagieren sich eine Atmosphäre von Sicherheit und
Gewaltlosigkeit für alle zu schaffen. Dies bedeutet insbesondere, dass:

\begin{itemize}
%\tightlist
\item
  wir keine Form von sexualisierter Gewalt, Belästigung oder
  Diskriminierung akzeptieren.
\item
  wir keine Beleidigungen, Trollen, Degradierungen oder persönliche
  Angriffe akzeptieren.
\item
  wir nicht akzeptieren, wenn Menschen anstelle von Argumenten angegriffen werden.
\end{itemize}

\hypertarget{umsetzung}{%
\section*{Umsetzung}\label{umsetzung}}

Das Umsetzen dieses Verhaltenskodexes liegt in der Verantwortung aller
anwesender Personen; Teilnehmika, Helfika, Orga, etc. Wie in allem
zwischenmenschlichem Handeln, kommt es jedoch auch auf der ZaPF zu
Problemen.

Du kannst dich mit Problemen jederzeit an die Vertrauenspersonen wenden.
Im Konfliktfall mit anderen Menschen oder bei persönlichen Problemen,
die nur dich betreffen, kannst du ihnen von deinem Problem erzählen.
Falls du das wünschst, können sie mit anderen Personen vermitteln, für
dich bei der Tagungsleitung um Hilfe bitten oder andere, z.B. externe,
Hilfe organisieren.

Du kannst mit so vielen oder so wenigen Vertrauenspersonen sprechen, wie
du möchtest.

\hypertarget{geltungsbereich}{%
\section*{Geltungsbereich}\label{geltungsbereich}}

Dieser Verhaltenskodex gilt auf allen unmittelbar mit der ZaPF in
Verbindung stehenden Veranstaltungen. Im Speziellen gilt er auf allen
ZaPFen und durch Organe der ZaPF organisierten Veranstaltungen für alle
anwesenden Personen. Außerdem für alle die ZaPF repräsentierenden
Personen während Sie ihre Funktion ausüben.

Mit der Anmeldung zu einer entsprechenden Veranstaltung, spätestens
jedoch mit dem Erscheinen, stimmst du zu, dich an diesen Verhaltenskodex
zu halten.


\vspace{1cm} 

\vfill
\begin{flushright}
	Verabschiedet am 13. November 2022 \\
	auf der ZaPF in Hamburg.
\end{flushright}

\end{document}
