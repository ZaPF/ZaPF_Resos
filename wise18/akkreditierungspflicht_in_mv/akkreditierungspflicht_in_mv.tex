\documentclass[draft,10pt,oneside]{scrartcl}

% Sprache und Encodings
\usepackage[ngerman]{babel}
%\usepackage[T1]{fontenc}
\usepackage[utf8]{inputenc}

% Typographisch interessante Pakete
\usepackage{microtype} % Randkorrektur und andere Anpassungen

% References to Internet and within the document
\usepackage[pdftex,colorlinks=false,
pdftitle={Antrag zur Änderung der Geschäftsordnung für Plenen der ZaPF},
pdfauthor={Jörg Behrmann (FUB), Björn Guth (RWTH)},
pdfcreator={pdflatex},
pdfdisplaydoctitle=true]{hyperref}

% Absaetze nicht Einruecken
\setlength{\parindent}{0pt}
\setlength{\parskip}{2pt}

% Formatierung auf A4 anpassen
\usepackage{geometry}
\geometry{paper=a4paper,left=15mm,right=15mm,top=10mm,bottom=10mm}

\begin{document}

\section*{Resolution zur Akkreditierungspflicht von Studiengängen in Mecklenburg-Vorpommern}

\textbf{Antragsteller*innen:} Liz (Rostock), Björn (RWTH), Jörg (FUB), Jonas (Münster)

\textbf{Adressaten:} Landesregierung Mecklenburg-Vorpommern, Hochschulleitungen der MV-Hochschulen

\subsection*{Antrag}

Die ZaPF möge beschließen:

\begin{quote}
    Die ZaPF betrachtet mit Sorge die Bestrebungen der Landesregierung
    Mecklenburg-Vorpommern die Akkreditierungspflicht für Studiengänge im Zuge
    der Novellierung des Hochschulgesetzes abzuschaffen.

    Die Akkreditierung hat sich als Mittel der Qualitätssicherung bewährt. Sie
    ist ein bundesweiter Standard und europaweit anerkannt. Weiterhin sichert
    sie die Teilhabe verschiedener Parteien, insbesondere der Studierenden, an
    Qualitätssicherungsverfahren und hilft, einen einheitlichen Mindeststandard
    für den Aufbau von Studiengängen deutschlandweit zu etablieren.

    Aufgrund ihrer weiten Verbreitung verlassen sich viele Arbeitgebende auf die
    Existenz akkreditierter Studiengänge. Durch den Wegfall dieses Merkmals in
    Mecklenburg-Vorpommern können Nachteil bei der Arbeitsplatzsuche für
    Abgänger*innen dieser Studiengänge und mecklenburg-vorpommerische
    Hochschulen werden zunehmend unattraktiver für Studieninteressierte und
    Studierende.
    Damit wird mutwillig in Kauf genommen, dass Studierendenzahlen sinken,
    Studierendenmobilität eingeschränkt wird und die Hochschulen an Reputation
    verlieren.

    Aus diesen Gründen fordern wir die Beibehaltung der Akkreditierungspflicht.
\end{quote}

\end{document}
