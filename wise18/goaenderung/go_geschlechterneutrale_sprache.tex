\documentclass[draft,10pt,oneside]{scrartcl}

% Sprache und Encodings
\usepackage[ngerman]{babel}
\usepackage[T1]{fontenc}
\usepackage[utf8]{inputenc}

% Typographisch interessante Pakete
\usepackage{microtype} % Randkorrektur und andere Anpassungen

% References to Internet and within the document
\usepackage[pdftex,colorlinks=false,
pdftitle={Antrag zur Änderung der Geschäftsordnung für Plenen der ZaPF},
pdfauthor={Jörg Behrmann (FUB), Björn Guth (RWTH)},
pdfcreator={pdflatex},
pdfdisplaydoctitle=true]{hyperref}

% Absaetze nicht Einruecken
\setlength{\parindent}{0pt}
\setlength{\parskip}{2pt}

% Formatierung auf A4 anpassen
\usepackage{geometry}
\geometry{paper=a4paper,left=15mm,right=15mm,top=10mm,bottom=10mm}

\begin{document}

\section*{Antrag zur Änderung der Geschäftsordnung für Plenen der ZaPF}

\textbf{Antragsteller:} Stephie (HUB), Hannah (HUB), Fabs (TUB), Manu (Leibzig), Jaco (Uni Wien), Opa (Alumni), Eli (Giessen), Jan (FUB), Jenny (FUB), Stefan (Köln), Paul (Freiburg), Manu (Uni Wien), Rosa (HUB), Dennis (Braunschweig), 
Jörg (FUB), Björn (RWTH)

\subsection*{Antrag}

Hiermit beantragen wir die Geschäftsordnung für Plenen der ZaPF wie folgend zu
ändern:

Ändere alle Vorkommnisse von \glqq{}Teilnehmer\grqq{}, \glqq{}Teilnehmerinnen\grqq{} sowie \glqq{}Teilnehmerinnen und Teilnehmer\grqq{} durch \glqq{}Teilnehmende\grqq{}

Ändere alle Vorkommnisse von \glqq{}Kandidatinnen und Kandidaten\grqq{} zu \glqq{}Kandidierende\grqq{}

Ändere alle Vorkommnisse von \glqq{}Gegenkandidaten\grqq{} zu \glqq{}Gegenkandidierender\grqq{}

Ändere alle Vorkommnisse von \glqq{}Helferinnen und Helfer\grqq{} zu \glqq{}Helfende\grqq{}

Ändere alle Vorkommnisse von \glqq{}Einzelner\grqq{} zu \glqq{}einzelene Personen\grqq{}

Ändere alle Vorkommnisse von \glqq{}Protokollanten\grqq{} zu \glqq{}Protokollführung\grqq{}

Ändere alle Vorkommnisse von \glqq{}er oder sie\grqq{} zu \glqq{}die Person\grqq{}

In 2.3 ändere:
\begin{quote}
    Zu Beginn der Sitzung werden ein oder mehrere Protokollführer bzw.
    Protokollführerinnen gewählt, das Protokoll der Sitzung wird im
    ZaPF-Reader für die folgende ZaPF abgedruckt.
\end{quote}
zu
\begin{quote}
    Zu Beginn der Sitzung wird eine Protokollführung, bestehend aus einer
    oder mehreren Personen, gewählt.
    Das Protokoll der Sitzung wird im ZaPF-Reader für die folgende ZaPF
    abgedruckt.
\end{quote}

Ändere 3.1.5 von:
\begin{quote}
    Die antragsstellende Person muss im Plenum anwesend sein
    oder kann einen Vertreter oder eine Vertreterin benennen und muss dies
    der Sitzungsleitung mitteilen.
    Die Vertreterin oder der Vertreter ist dann die neue antragstellende Person.
\end{quote}
zu:
\begin{quote}
    Die antragsstellende Person muss im Plenum anwesend sein
    oder kann eine Vertretung benennen und muss dies
    der Sitzungsleitung mitteilen.
    Die vertretende ist dann die neue antragstellende Person.
\end{quote}

Ändere 4.1.2 von
\begin{quote}
    Beschlüsse sind nach außen zu tragende *Resolutionen*, die zwingend einen
    Adressaten haben müssen, *Positionspapiere*, die keinen Adressaten haben, (...)
\end{quote}
zu:
\begin{quote}
    Beschlüsse sind nach außen zu tragende *Resolutionen*, die zwingend an
    eine oder mehrere natürliche oder juristische Personen adressiert sein
    müssen, *Positionspapiere*, die an niemanden adressiert sind,
\end{quote}

In 4.1.6 ändere
\begin{quote}
    vom Hauptantragsteller oder von der Hauptantragstellerin
\end{quote}
zu
\begin{quote}
    von den hauptantragstellenden Persone
\end{quote}

Im Anhang zu Resolutionen, Positionspapiere und Selbstverpflichtungen ändere
\begin{quote}
    Adressaten
\end{quote}
zu
\begin{quote}
    adressierten natürlichen und juristischen Personen
\end{quote}
sowie ändere
\begin{quote}
    aber haben keine eigenen Adressaten
\end{quote}
zu
\begin{quote}
    werden aber nicht gesondert verschickt
\end{quote}

\end{document}
