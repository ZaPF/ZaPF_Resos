\hypertarget{resolution-zur-geplanten-novelle-des-berliner-hochschulgesetzes-berlhg}{%
\section{Resolution zur geplanten Novelle des Berliner Hochschulgesetzes
(BerlHG)}\label{resolution-zur-geplanten-novelle-des-berliner-hochschulgesetzes-berlhg}}

\textbf{Antragstellende:} Jenny (FUB), Hannah (HUB), Stephie (HUB),
lobachevsky (FUB), scatty (RWTH)

\textbf{Adressaten:}

\begin{itemize}
\tightlist
\item
  wissenschaftliche Sprecher*innen der Koalitionsparteien (SPD, Die
  Linke, Bündnis 90/Die Grünen) im Berliner Abgeordnetenhaus
\item
  Für Hochschulen zuständiges Mitglied des Berliner Senats samt
  Staatssekretär
\item
  Landesarbeitsgemeinschaften der Koalitionsfraktionen?
\end{itemize}

\hypertarget{resolution}{%
\subsection{Resolution}\label{resolution}}

Im Rahmen der geplanten Hochschulgesetz-Novellierung in Berlin hält die
Zusammenkunft aller Physikfachschaften (ZaPF) die folgenden Punkte für
besonders relevant und sieht sie als (wichtige Grundgedanke)
Grundpfeiler eines neuen Gesetzes:

\hypertarget{studium}{%
\subsubsection{Studium}\label{studium}}

\begin{enumerate}
\def\labelenumi{(\arabic{enumi})}
\item
  Im Zentrum des Studiums steht die Persönlichkeitsentwicklung der
  Studierenden. Ein Studium hat das Ziel, Menschen zum kritischen Denken
  und Hinterfragen anzuregen. Zudem soll der Abschluss des Studiums eine
  Berufsbefähigung gewährleisten. Die derzeitige Hauptzielsetzung einer
  Berufsqualifizierung lehnen wir ab.
\item
  Alle Menschen haben ein Recht auf Bildung und freie Entfaltung. Wir
  lehnen daher Zugangs- und Zulassungsvoraussetzungen jeder Form ab. Die
  dafür notwendigen Kapazitäten an den Hochschulen sind zu schaffen.
  Details siehe
  \href{https://zapfev.de/resolutionen/wise16/Zugangs-Zulassungsbeschraenkung/Reso\%20gegen\%20Zugangs-\%20und\%20Zulassungsbeschraenkungen.pdf}{\emph{Resolution
  gegen Zugangs- und Zulassungbeschränkungen}}
\item
  Im Sinne der freien Entfaltung und der eigenständigen
  Persönlichkeitsentwicklung lehnen wir jede Art der Zwangsberatung von
  Studierenden ab. Eine Beratung ist nur dann sinnvoll, wenn sie
  freiwillig und aus eigener Entscheidung erfolgt. Daher sollen die
  Beratungsangebote beworben werden und die Kapazitäten geschaffen
  werden.
\item
  Derzeit beträgt der Arbeitsaufwand, bei auf die Vorlesungszeit
  konzentrierten Studienfächern, 50 - 60 Stunden pro Woche. Dies ist die
  Folge einer Berechnung von Leistungspunkten an der oberen Grenze des
  in den Akkreditierungsrichtlinien festgelegten Bereiches von 25-30
  Zeitstunden/Leistungspunkt. Für eine Normalisierung der
  Arbeitsbelastung soll der Maßstab von 1 LP = 25 Stunden Arbeit in
  Präsenz- und Selbststudium angewendet werden.
\item
  Momentan bemisst sich die Regelstudienzeit an einem durchschnittlichen
  Arbeitspensum von 30 LP pro Semester. Daraus ergibt sich bei einem
  Bachelor mit 180 Leistungspunkten eine Regelstudienzeit von 3 Jahren.
  Hierbei ist das Wort ``Regel''-Studienzeit irreführend, da es aktuell
  die kürzest mögliche Zeit beschreibt um das Studium abzuschließen und
  nicht die durchschnittlich benötigte Zeit, die höher liegt. Daher
  fordern wir die Leistungspunkte pro Semester bei 20-25LP anzusetzen,
  so dass die Planstudienzeit zwischen 3.5 und 4.5 Jahre beträgt. Dies
  entspricht einer Annäherung an die real absolvierbaren Leistungspunkte
  pro Semester und erleichtert Studierenden im Vollzeitstudium, in einem
  Umfang arbeiten zu können, die den Lebenshaltungskosten gerecht wird.
\item
  Ein Teilzeitstudium ermöglicht das Studieren in Vereinbarkeit mit
  Arbeit, Familie und anderen Lebensbereichen. Diese Möglichkeit müssen
  alle Studierenden erhalten, weshalb das Recht auf ein Teilzeitstudium,
  ohne Begründung, im Berliner Hochschulgesetz verankert werden soll.
  Einzuschätzen, in welcher Zeit man sich welche Menge an Fachwissen
  aneignen kann, ist Teil der Persönlichkeitsentwicklung und muss in
  freier Entscheidung möglich sein.
\item
  Fachsemester definieren die Zeit, in der jemand in einen Studiengang
  bisher insgesamt eingeschrieben war. Bei Wechsel der Universität und
  Immatrikulation in den gleichen Studiengang wird die Person in das
  Fachsemester +1 eingestuft. Beim Wechsel des Studienganges in einen
  fachähnlichen oder fachnahen Studiengang wird die Person in das erste
  oder auf Antrag in ein höheres Fachsemester eingestuft. Die
  Hochschulsemester sind die Summe aller Fachsemester. Ein
  Teilzeitstudium führt zu einem anteiligen Fachsemester entsprechend
  des Verhältnisses der angestrebten Arbeitsbelastung zum
  Vollzeitstudium. Durch diese Abänderung der Handhabung von
  Fachsemestern schaffen wir Probleme beim Hochschulwechsel aus der
  Welt.
\end{enumerate}

\hypertarget{semestergebuxfchren}{%
\paragraph{Semestergebühren}\label{semestergebuxfchren}}

Die ZaPF spricht sich gegen die Erhebungen von Verwaltungs- und
Rückmeldegebühren aus. Wir sehen die Durchführung von
Verwaltungsaufgaben im Rahmen des Studiums wie Immatrikulation,
Prüfungsverwaltungen oder Rückmeldungen als Kernaufgaben der
Hochschulen, die ohne die Erhebung von Gebühren finanziert werden
müssen.

\hypertarget{pruxfcfungen}{%
\subsubsection{Prüfungen}\label{pruxfcfungen}}

\hypertarget{pruxfcfungsan--und--abmeldungen}{%
\paragraph{Prüfungsan- und
-abmeldungen}\label{pruxfcfungsan--und--abmeldungen}}

\begin{quote}
Prüfungsan- und abmeldungen werden von Hochschulen individuell
gehandhabt und dienen oft einem logistischem Zweck. Dies geht teilweise
soweit, dass selbst innerhalb einer Hochschule oft deutliche
Unterschiede zu vermerken sind. Hier stehen die Fristen im Widerspruch
zu Flexibilität und Studierendenfreundlichkeit.

In unseren Augen gibt es keinen Grund, warum Studierende zum Teil
mehrere Wochen vor Prüfungstermin von einer Prüfungsanmeldung
zurücktreten müssen und wir sehen in dieser Form der Handhabung unnötige
Hürden für Studierende. Eine Prüfungsanmeldung soll, falls sie denn
explizit nötig ist, revidierbar sein. Diese Revision sollte so spät wie
möglich vor der Prüfung durchführbar sein.
\end{quote}

\emph{Ausführliche Details zu den Überlegungen der ZaPF siehe
das}\href{https://zapfev.de/resolutionen/sose18/Pruefungsanmeldung/reso_pruefungsanmeldung.pdf}{\emph{Resolution
für einen flexibleren Umgang mit Prüfungsan- und abmeldungen}}

\hypertarget{zwangsexmatrikulation}{%
\paragraph{Zwangsexmatrikulation}\label{zwangsexmatrikulation}}

\begin{quote}
Studierende durch drohende Zwangsexmatrikulation unter Druck zu setzen
ist in unseren Augen unangemessen; es ersetzt selbstverantwortliches und
selbstbestimmtes durch prüfungsorientiertes Studieren und behindert
damit die freie Entfaltung. Die ZaPF spricht sich gegen sämtliche
Regelungen in Studienordnungen aus, welche den Fokus des Studiums von
der Aneignung von Wissen und persönlicher Entwicklung hin zu der
Verhinderung der eigenen Exmatrikulation verschieben. Insbesondere
fordern wir, solche Regelungen aufzuheben oder abzuändern, die auf eine
Zwangsexmatrikulation hinaus laufen können, insbesondere die Begrenzung
der Anzahl von Prüfungsversuchen oder die Zwangsberatungen.
\end{quote}

\emph{Ausführliche Details zu den Überlegungen der ZaPF siehe das}
\href{https://zapfev.de/resolutionen/wise17/Zwangsexmatrikulation/Zwangsexmatrikulation.pdf}{\emph{Resolution
zu Zwangsexmatrikulation}}

\hypertarget{symptompflicht}{%
\paragraph{Symptompflicht}\label{symptompflicht}}

\begin{quote}
Die ZaPF spricht sich gegen die geforderte Angabe von Symptomen auf
Attesten für die Prüfungsunfähigkeitsmeldung aus. Wir fordern,
ausschließlich das Verfahren der Prüfungsunfähigkeitsbscheinigungen
analog zu Arbeitsunfähigkeitsbescheinigungen zu ermöglichen.
\end{quote}

\emph{Ausführliche Details zu den Überlegungen der ZaPF siehe das}
\href{https://zapfev.de/resolutionen/sose17/symptompflicht/PosPapier_Symptompflicht.pdf}{\emph{Positionspapier
zu Symptompflicht auf Attesten}} \emph{und die}
\href{https://zapfev.de/resolutionen/wise17/Pruefungsunfaehigkeit/Pruefungsunfaehigkeitsbescheinigung.pdf}{\emph{Resolution
zu Prüfungsunfähigkeitsbescheinigungen}}

\hypertarget{akkreditierung}{%
\subsubsection{Akkreditierung}\label{akkreditierung}}

Die Qualitätssicherung im Rahmen von (System-)Akkreditierungen muss eine
Rückbindung an die demokratischen Hochschulgremien erfahren, da sonst
Verwaltungsstellen über die Ausgestaltung und Qualitätssicherung von
Studiengängen entscheiden. Die Einbindung der Ausbildungskommissionen
und der Kommissionen für Studium und Lehre muss gesetzlich garantiert
werden.

Neuakkreditierungen sollen mit einer verkürzten Frist von fünf Jahren
gelten.

Alle Gutachter*innen sollen im Bereich Akkreditierung geschult sein --
entweder durch ihre Erfahrung oder durch entsprechende
Fortbildungsmaßnahmen. Bei Akkreditierungen von Lehramtsstudiengängen
darf die Vertretung der Berufspraxis in der Gutachtergruppe nicht durch
einen Vertreter*in der obersten Landesbehörde ersetzt werden, sondern
soll um diese*n ergänzt werden.

\hypertarget{verfasste-studierendenschaft}{%
\subsubsection{Verfasste
Studierendenschaft}\label{verfasste-studierendenschaft}}

\hypertarget{stuxe4rkung-des-zugangs-zu-hochschuleinrichtungen}{%
\paragraph{Stärkung des Zugangs zu
Hochschuleinrichtungen}\label{stuxe4rkung-des-zugangs-zu-hochschuleinrichtungen}}

Verfasste Studierendenschaften müssen für die Erfüllung ihrer
vielfältigen und wichtigen Aufgaben freien Zugang zu den Einrichtungen
der Hochschule haben. Insbesondere die uneingeschränkte Vergabe von
Räumen für Veranstaltungen und der Versand von Mails an die Studierenden
sind Rechte, die im Gesetz garantiert werden müssen. Die Hochschule hat
die verfasste Studierendenschaft durch Zugang zu allgemeinen
Verwaltungseinrichtungen wie Briefversand z.B. an neue Erstsemester zu
unterstützen.

\hypertarget{fachschaftsinitativen}{%
\paragraph{Fachschaftsinitativen}\label{fachschaftsinitativen}}

Viele Hochschulen in Berlin haben als innovatives Konzept das offene
Modell der studentischen Partizipation eingeführt, die sogenannten
Fachschaftsinitativen. Diese niederschwelligen Strukturen sind als
studentische Gestaltungsmöglichkeiten in der akademischen
Selbstverwaltung entsprechend im Gesetz aufzunehmen. Ihnen muss der
Zugang zu den Einrichtungen der Hochschulen garantiert werden,
insbesondere die Bereitstellung von Fachschaftsräume.

\hypertarget{semesterticket}{%
\paragraph{Semesterticket}\label{semesterticket}}

Zugang zum öffentlichen Personennahverkehr ist eine unverzichtbare
Voraussetzung für ein sozialverträgliches Studium. In Anbetracht der
steigenden Lebenshaltungskosten in Berlin sehen wir es als notwendig,
dass ein Semesterticket nicht nur kostengünstig, sondern maximal
kostenneutral ist.

\hypertarget{gremien}{%
\subsubsection{Gremien}\label{gremien}}

\hypertarget{zusammensetzung-und-statusgruppenveto}{%
\paragraph{Zusammensetzung und
Statusgruppenveto}\label{zusammensetzung-und-statusgruppenveto}}

Im Sinne einer demokratischen Hochschule, sind folgende Punkte wichtig
zu beachten:

\begin{quote}
Für die konstruktive Zusammenarbeit aller Statusgruppen in
Universitätsgremien empfiehlt die ZaPF folgende Punkte zu beachten.
\end{quote}

\begin{quote}
\begin{enumerate}
\def\labelenumi{\arabic{enumi}.}
\tightlist
\item
  Grundsätzlich ist es falsch, wenn eine Statusgruppe in einem
  demokratischen Gremium automatisch die Mehrheit besitzt. Vielmehr ist
  es notwendig, dass keine Position übergangen werden kann. Dies kann
  z.B. durch eine paritätische Zusammensetzung oder ein
  Statusgruppen-Vetorecht {[}(siehe Punkt 2){]} sicher gestellt werden.
  Voraussetzung dafür ist, dass die Teilhaberechte Aller gesetzlich
  sichergestellt sind und nicht nur optional gewährt werden.
\end{enumerate}
\end{quote}

\href{https://zapfev.de/resolutionen/sose18/Hochschulgesetze/reso_hsgesetze.pdf}{\emph{Abschnitt
``Gremien'' der Resolution zu Hochschulgesetzen, verabschiedet in
Heidelberg 2018}}

\begin{quote}
\begin{enumerate}
\def\labelenumi{\arabic{enumi}.}
\setcounter{enumi}{1}
\tightlist
\item
  Lehnt eine Statusgruppe geschlossen einen Antrag ab, soll sie das
  Recht haben, ein Veto einzulegen (\emph{Statusgruppenveto}).
  {[}\ldots{}{]}
\item
  Die Mitglieder des Dekanats {[}und des Präsidiums bzw. Rektorats{]}
  dürfen ausschließlich mit beratender Funktion an Gremiumssitzungen
  teilnehmen. {[}Die Leitung von Gremien erfolgt durch einen selbst
  gewählten Vorsitz.{]}
\end{enumerate}
\end{quote}

\href{https://zapf.wiki/Sammlung_aller_Resolutionen_und_Positionspapiere\#Positionspapier_zur_demokratischen_Mitgestaltung_in_Hochschulgremien}{\emph{angelehnt
an das Positionspapier aus Jena 2013}}

\begin{enumerate}
\def\labelenumi{\arabic{enumi}.}
\setcounter{enumi}{3}
\tightlist
\item
  Gewählten Stellvertretungen der Mitglieder aller Gremien darf die
  Anwesenheit auch in nichtöffentlichen Sitzungen des jeweiligen
  Gremiums nicht verwehrt werden.
\end{enumerate}

\hypertarget{kontrollrechte-von-gremienmitgliedern}{%
\paragraph{Kontrollrechte von
Gremienmitgliedern}\label{kontrollrechte-von-gremienmitgliedern}}

Zur Wahrung der demokratischen Grundsätze in der akademischen
Selbstverwaltung sind Kontrollrechte für Gremienmitglieder unabdingbar.
Daher fordern wir das Recht, Berichte sowie Akteneinsicht zu allen in
den Zuständigkeitsbereich des Gremiums und der von ihm gewählten Ämter
fallenden Fragestellungen zu erhalten. Den Gremien muss darüber hinaus
das Recht auf Einholung von Gutachten und Stellungnahmen garantiert
werden.

\hypertarget{gremienvor--beziehungsweise--nachmittag}{%
\paragraph{Gremienvor- beziehungsweise
-nachmittag}\label{gremienvor--beziehungsweise--nachmittag}}

Die ZaPF spricht sich dafür aus, im Hochschulgesetz einen für
Gremienarbeit reservierten Vor- oder Nachmittag zu verankern.
Insbesondere sollen in diesem Zeitraum keine Lehrveranstaltungen
stattfinden. Die Wahl des Zeitraums und des Wochentages soll der
jeweiligen Hochschule überlassen bleiben.

\hypertarget{gleichstellung}{%
\subsubsection{Gleichstellung}\label{gleichstellung}}

Die Forderungen der ZaPF zur Durchsetzung von Gleichstellung an den
Berliner Hochschulen lauten: (1) genderneutrale Formulierungen im
BerlHG. Beispiel: ``Studierende'' statt ``Studentinnen und Studenten''.
(2) dass Hochschulgrade in genderneutraler Form verliehen werden. (3)
eine Änderung der Regelung zur Frauenbeauftragten, so dass die Anzahl
der hauptberuflichen Frauenbeauftragten von der Anzahl der Studierenden
der Hochschule abhängt und mindestens drei hauptberufliche
Frauenbeauftragte an der Hochschule tätig sind. (4) zur Wahl der haupt-
und nebenberuflichen Frauenbeauftragten haben alle Mitglieder der
Hochschule das passive Wahlrecht. Das aktive Wahlrecht setzt den Eintrag
``weiblich'' in der Geburtsurkunde voraus. (5) eine hauptberuflich
beauftragte Person zur Gender-Gleichstellung. (6) eine beauftragte
Person für Studierende mit Behinderungen nach BerlHG pro Fachbereich.
(7) eine beauftragte Person für Studierende mit Kindern, analog zu (5)
pro Fachbereich. (8) Eltern-Kind-Zimmer, die KiTa-Standards erfüllen.
(9) die Möglichkeit, bei der Immatrikulation nicht-binäre Genderoptionen
wählen zu können.

\hypertarget{promotion}{%
\subsubsection{Promotion}\label{promotion}}

\hypertarget{rolle-und-statusgruppe}{%
\paragraph{Rolle und Statusgruppe}\label{rolle-und-statusgruppe}}

Die Promotion dient dem Nachweis der Befähigung zu vertiefter
wissenschaftlicher Arbeit. Sie ist die erste Phase selbständiger
wissenschaftlicher Arbeit.

Wir als ZaPF fordern die Aufnahme dieser Klarstellung in das Berliner
Hochschulgesetz und in Konsequenz die Einordnung aller Promovierender in
die Statusgruppe der wissenschaftlichen Mitarbeitenden.

\hypertarget{bessere-promotionsbedingungen}{%
\paragraph{Bessere
Promotionsbedingungen}\label{bessere-promotionsbedingungen}}

Zur Auflösung der Abhängigkeit von der betreuenden Person fordern wir
die personelle Trennung von Betreuung, Begutachtung und
arbeitsrechtlicher Weisungsbefugnis.

Um dies zu gewährleisten müssen Promovierendenzentren an den Hochschulen
mit den Aufgaben: - Zulassung zur Promotion und Zuordnung der fachlichen
Betreuung, - Finanzielle und arbeitsrechtliche Organisation des
Promotionsvorhabens, - Bestellung der Gutachter der Doktorarbeit, -
Unterstützung und Weiterbildung in Fragen der Lehre, - Überfachliche
Weiterbildung, - Ombudsstelle eingerichtet werden.

Die Lehrtätigkeit Promovierender sind eine der Säulen universitärer
Lehre. Die Qualität dieser Lehrtätigkeit ist daher von besonderer
Wichtigkeit. Aus diesem Grund müssen Promovierende bei der Entwicklung
ihrer didaktischen Fähigkeiten unterstützt werden.

Als Anstellungsverhältnis fordern wir Qualifizierungsstellen mit den
unter Arbeitsbedingungen genannten Standards (siehe
\href{https://protokolle.zapf.in/reso_BerlHG\#Qualifizierungsstellen}{unten}).

\hypertarget{arbeitsbedingungen-in-der-wissenschaft}{%
\subsubsection{Arbeitsbedingungen in der
Wissenschaft}\label{arbeitsbedingungen-in-der-wissenschaft}}

\hypertarget{qualifizierungsstellen}{%
\paragraph{Qualifizierungsstellen}\label{qualifizierungsstellen}}

\emph{Arbeitsverhältnisse auf Qualifizierungsstellen an Hochschulen
werden grundsätzlich auf Basis von 100 Prozent der regulären tariflichen
Arbeitszeit geschlossen. Begründete Ausnahmen hiervon (Teilzeit)
bedürfen der Zustimmung des Arbeitnehmers bzw. der Arbeitnehmerin. Wird
ein Arbeitsverhältnis mit weniger als 100 Prozent der regulären
tariflichen Arbeitszeit geschlossen, so ist eine Mehrarbeit über das
vertraglich festgesetzte Maß unzulässig. Es stehen mindestens 50 Prozent
der regulären tariflichen Arbeitszeit zur Erreichung des
Qualifikationszieles zur Verfügung.}

Für Arbeitsverhältnisse mit einem Qualifikationsziel fordern wir
folgende Standards:

Arbeitsverhältnisse auf Qualifizierungsstellen an Hochschulen werden
grundsätzlich auf Basis von 100 Prozent der regulären tariflichen
Arbeitszeit geschlossen. In begründeten Ausnahmen (Teilzeit) darf
hiervon in Absprache mit den Arbeitnehmenden abgewichen werden. Wird ein
Arbeitsverhältnis mit weniger als 100 Prozent der regulären tariflichen
Arbeitszeit geschlossen, so ist eine Mehrarbeit über das vertraglich
festgesetzte Maß unzulässig. Es stehen mindestens 50 Prozent der
regulären tariflichen Arbeitszeit zur Erreichung des
Qualifikationszieles zur Verfügung.

\hypertarget{dauerstellen}{%
\paragraph{Dauerstellen}\label{dauerstellen}}

Die ZaPF fordert die Schaffung unbefristeter Stellen im
wissenschaftlichen Mittelbau.

\begin{quote}
Nur durch eine deutliche Erhöhung der Anzahl unbefristeter Stellen im
wissenschaftlichen Mittelbau kann es zu einer nachhaltigen
Qualitätssicherung in der Forschung und Lehre, effizientem
Wissenstransfer und einer Steigerung der Attraktivität der Karriere in
der Wissenschaft kommen.
\end{quote}

\href{https://zapfev.de/resolutionen/sose17/mittelbau/mittelbau.pdf}{\emph{Verabschiedet
am 27.05.2017 in Berlin}}

\begin{quote}
Insbesondere müssen Daueraufgaben durch unbefristete Anstellungen
abgedeckt sein.
\end{quote}

\href{https://zapfev.de/resolutionen/wise15/WissZeitVG/Stellungnahme_WiSe15_WissZeitVG.pdf}{\emph{Resolution
``Novelle des Wissenschaftszeitvertragsgesetzes'', beschlossen in
Frankfurt 2015 (Z. 30-35)}}

\hypertarget{gute-wissenschaftliche-praxis}{%
\paragraph{Gute Wissenschaftliche
Praxis}\label{gute-wissenschaftliche-praxis}}

Forschung nach den Prinzipien guter wissenschaftlicher Praxis ist in
allen Bereichen der Wissenschaft anzustreben.

Als konkrete Maßnahme sehen wir die Notwendigkeit der Einrichtung
unabhängiger Ombudsstellen auf Landesebene.

\hypertarget{transparenz-und-gesellschaftliche-verantwortung-der-hochschulen}{%
\subsubsection{Transparenz und Gesellschaftliche Verantwortung der
Hochschulen}\label{transparenz-und-gesellschaftliche-verantwortung-der-hochschulen}}

\hypertarget{transparenz-in-der-drittmittel-finanzierten-forschung}{%
\paragraph{Transparenz in der Drittmittel-finanzierten
Forschung}\label{transparenz-in-der-drittmittel-finanzierten-forschung}}

\begin{quote}
Wir halten Transparenz bei der Durchführung von wissenschaftlichen
Tätigkeiten im Interesse Dritter für notwendig. Deshalb fordert die
ZaPF, dass bei Drittmittelprojekten folgende Angaben jährlich
veröffentlicht werden müssen: 1. Titel des Projekts 2. Hochschule mit
Organisationseinheit 3. Auftraggebende Personen mit Sparte/Handlungsfeld
der Abteilung 4. Projekt- und Vertragslaufzeit 5. Gesamtsumme 6. Angaben
der Geheimhaltungsvereinbarungen oder Publikationsbeschränkungen, u. a.
Art, Dauer und Umfang
\end{quote}

\href{https://zapf.wiki/images/d/dc/Stellungnahme_WiSe15_Transparenz_in_der_Drittmittelforschung.pdf}{\emph{Positionspapier
verabschiedet am 22.11.2015 in Frankfurt mit Streichungen}}

Zusätzlich muss am Projektende ein Abschlussbericht veröffentlicht
werden.

\hypertarget{friedensbindung}{%
\paragraph{Friedensbindung}\label{friedensbindung}}

Im Berliner Hochschulgesetz fehlt bisher die Verpflichtung zur zivilen
Forschung. Diese Verantwortung soll bei den Aufgaben der Hochschule im
Gesetz verankert werden.

\hypertarget{psychologische-erstberatung}{%
\subsubsection{Psychologische
Erstberatung}\label{psychologische-erstberatung}}

Mit dem Studium beginnt ein neuer Lebensabschnitt in dem Menschen mit
neuen Herausforderungen konfrontiert werden. Dies kann zu psychischen
Belastungen führen, die unter Umständen nicht ohne professionelle
Ansprechpersonen bewältigt werden können. Für diese Aufgabe muss die
Hochschule eine geeignete Anlaufstelle mit psychologisch ausgebildeten
Personen einrichten. Zudem sieht die ZaPF die Sensibilisierung im Umgang
mit psychologischen Problemen als wichtiges gesellschaftliches Thema und
unterstützt die Schaffung und Bewerbung von universitären
Beratungsstellen.
