\documentclass[DIV=calc]{scrartcl}
\usepackage[utf8]{inputenc}
\usepackage[T1]{fontenc}
\usepackage[ngerman]{babel}
\usepackage{graphicx}
\usepackage[draft, markup=underlined]{changes}
\usepackage{csquotes}
\usepackage{eurosym}

\usepackage{ulem}
%\usepackage[dvipsnames]{xcolor}
\usepackage{paralist}
\usepackage{fixltx2e}
%\usepackage{ellipsis}
\usepackage[tracking=true]{microtype}

\usepackage{lmodern}              % Ersatz fuer Computer Modern-Schriften
%\usepackage{hfoldsty}

%\usepackage{fourier}             % Schriftart
\usepackage[scaled=0.81]{helvet}     % Schriftart

\usepackage[hyphens]{url}
%\usepackage{tocloft}             % Paket für Table of Contents

\usepackage{xcolor}
\definecolor{urlred}{HTML}{660000}

\usepackage{hyperref}
\hypersetup{colorlinks=false}

%\usepackage{mdwlist}     % Änderung der Zeilenabstände bei itemize und enumerate
% \usepackage{draftwatermark} % Wasserzeichen ``Entwurf''
% \SetWatermarkText{Antrag}

\parindent 0pt                 % Absatzeinrücken verhindern
\parskip 12pt                 % Absätze durch Lücke trennen

\setlength{\textheight}{23cm}
\usepackage{fancyhdr}
\pagestyle{fancy}
\fancyhead{} % clear all header fields
\cfoot{}
\lfoot{Zusammenkunft aller Physik-Fachschaften}
\rfoot{www.zapfev.de\\stapf@zapf.in}
\renewcommand{\headrulewidth}{0pt}
\renewcommand{\footrulewidth}{0.1pt}
\newcommand{\gen}{*innen}
\addto{\captionsngerman}{\renewcommand{\refname}{Quellen}}

%%%% Mit-TeXen Kommandoset
\usepackage[normalem]{ulem}
\usepackage{xcolor}
\usepackage{xspace} 

\newcommand{\replace}[2]{
    \sout{\textcolor{blue}{#1}}~\textcolor{blue}{#2}}
\newcommand{\delete}[1]{
    \sout{\textcolor{red}{#1}}}
\newcommand{\add}[1]{
    \textcolor{blue}{#1}}

\newif\ifcomments
\commentsfalse
%\commentstrue

\newcommand{\red}[1]{{\ifcomments\color{red} {#1}\else{#1}\fi}\xspace}
\newcommand{\blue}[1]{{\ifcomments\color{blue} {#1}\else{#1}\fi}\xspace}
\newcommand{\green}[1]{{\ifcomments\color{green} {#1}\else{#1}\fi}\xspace}

\newcommand{\repl}[2]{{\ifcomments{\color{red} \sout{#1}}{\color{blue} {\xspace #2}}\else{#2}\fi}}
%\newcommand{\repl}[2]{{\color{red} \sout{#1}\xspace{\color{blue} {#2}}\else{#2}\fi}\xspace}

\newcommand{\initcomment}[2]{%
	\expandafter\newcommand\csname#1\endcsname{%
		\def\thiscommentname{#1}%
		\definecolor{col}{rgb}{#2}%
		\def\thiscommentcolor{col}%
}}

% initcomment Name RGB-color
\initcomment{Philipp}{0, 0.5, 0}

%\renewcommand{\comment}[1]{{\ifcomments{\color{red} {#1}}{}\fi}\xspace}

\renewcommand{\comment}[2][\nobody]{
	\ifdefined#1
	{\ifcomments{#1 \expandafter\color{\thiscommentcolor}{\thiscommentname: #2}}{}\fi}\xspace
	\else
	{\ifcomments{\color{red} {#2}}{}\fi}\xspace
	\fi
}

\newcommand{\zapf}{ZaPF\xspace}

\let\oldgrqq=\grqq
\def\grqq{\oldgrqq\xspace}

\setlength{\parskip}{.6em}
\setlength{\parindent}{0mm}

%\usepackage{geometry}
%\geometry{left=2.5cm, right=2.5cm, top=2.5cm, bottom=3.5cm}

% \renewcommand{\familydefault}{\sfdefault}




\begin{document}

\hspace{0.87\textwidth}
\begin{minipage}{120pt}
	\vspace{-1.8cm}
	\includegraphics[width=80pt]{../logo.pdf}
	\centering
	\small Zusammenkunft aller Physik-Fachschaften
\end{minipage}

\begin{center}
  \huge{Resolution: Umdenken in den Universitäten hin zu Open-Source-Lösungen}\vspace{.25\baselineskip}\\
  \normalsize
\end{center}
\vspace{1cm}

%%%% Metadaten %%%%

% \paragraph{Addressierte:} - HRK- Uni Präsidenten- CCC im CC- FZS (interdisziplinäre Reso) - DPG im CC- DFN -Metafa -Landesastenkonferenzen -Deutschen Initiative fuer Netzwerkinformation (DINI) 
 

% \paragraph{Antragstellende:} Jörg (Siegen), Tobi (Düsseldorf), ChrisPi (Heidelberg), Jörg (Alumni)

%%%% Text des Antrages zur veröffentlichung %%%%

% \section*{Antragstext}

Die ZaPF fordert ein Umdenken bei der Softwarenutzung hin zu Open-Source Software. 

Bei der voranschreitenden Digitalisierung der Hochschulen halten wir die Nutzung von unabhängigen und quelloffenen Lösungen für nachhaltiger und vorteilhafter als die Bindung an proprietäre Software. 

Die Nachhaltigkeit liegt zum einen in Synergieeffekten, die aus der gemeinsamen Nutzung folgen, als auch in der digitalen Souveränität\footnote{\url{https://www.cio.bund.de/SharedDocs/Kurzmeldungen/DE/2021/pm_os_plattform.html}} von strategischen Entscheidungen der involvierten Konzerne\footnote{\url{https://home.cern/news/news/computing/migrating-open-source-technologies}}. Nicht jede staatliche oder universitäre Stelle benötigt eigene Lizenzen und jede Weiterentwicklung kann sofort für alle zur Verfügung gestellt werden. Dabei kommen öffentliche Gelder nicht nur einzelnen Unternehmen, sondern der Öffentlichkeit zugute, wie etwa bei den Projekten Open Educational Ressources\footnote{\url{https://www.bildung-forschung.digital/de/open-educational-resources-offen-lizensierte-bildungsmaterialien-in-die-breite-bringen-1799.html}} oder der Nationalen Bildungsplattform\footnote{\url{https://www.bmbf.de/de/neue-bekanntmachung-zum-aufbau-einer-digitalen-bildungsplattform-13790.html}}. Der Anwender kann die Software selbst flexibel weiterentwicklen, an die individuellen Bedürfnisse anpassen und mit den anderen Nutzern solidarisch teilen. Abwärtskompatibilität und die langfristige Weiterentwicklung, insbesondere in Bezug auf Sicherheit, lässt sich mit Open-Source leichter gewährleisten. So können Standards und Schnittstellen einfach und transparent ergänzt werden. Auch können neue rechtliche Vorgaben, insbesondere auch Datenschutzvorgaben, nachträglich implementiert werden.

Durch offene Standards und Schnittstellen ist eine Flexibilität in der Nutzung vorhanden, die mit proprietärer Software nicht erreicht werden kann. Dies erhöht die Interoperabilität und ergibt kreative und flexible neue Einsatzmöglichkeiten. Verschwindende Abhängigkeiten zu Softwareanbietern reduzieren die Kosten für Lizenzen, halten die Entwicklungskosten langfristig kalkulierbar, sowie den Support unabhängig vom kommerziellen Erfolg.

Für den erfolgreichen Umstieg ist erforderlich Schulungen und Serviceportale für Endanwender anzubieten. 
Um Hürden aus Sicht einzelner Anwender zu verringern, müssen bereits zu Beginn gute Alternativen zur aktuell genutzten Software vorliegen. Diese anfänglich hohen Entwicklungs- und Schulungskosten werden sich aber auf lange Sicht relativieren, da der fortlaufende Betrieb unter eigener Führung wie oben beschrieben zu Einsparungen führen wird. 

Durch die Ausbildung mit verschiedenen Open-Source Lösungen werden allgemeine Kompetenzen im Umgang mit entsprechender Software erlernt. Ein späterer Umstieg auf bestehenden Unternehmenssoftware ist ohne weiteres möglich. Zwar können ähnliche Kompetenzen auch mit proprietärer Software erworben werden, aber eine Ausbildung an proprietärer Software stärkt nur den Marktstandard und die Monopolstellung einzelner Softwareunternehmen. Die dynamische Open-Source Umgebung hingegen fördert darüber hinaus die Fähigkeit, sich an neue Programme anzupassen. 

Durch die breite Nutzung von Open-Source Software an Universitäten kann auf Dauer ein gesellschaftlicher Wandel vorangetrieben werden. Hierdurch kann auch in allen Bereichen der Gesellschaft die Nutzung von Open-Source Software verstärkt werden. 

\vspace{1cm} 

\vfill
\begin{flushright}
	Verabschiedet am 23. Mai 2021 \\
	auf der Digital-ZaPF hosted in Rostock.
\end{flushright}

\end{document}
