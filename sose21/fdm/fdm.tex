\documentclass[DIV=calc]{scrartcl}
\usepackage[utf8]{inputenc}
\usepackage[T1]{fontenc}
\usepackage[ngerman]{babel}
\usepackage{graphicx}
\usepackage[draft, markup=underlined]{changes}
\usepackage{csquotes}

\usepackage[normalem]{ulem}
%\usepackage[dvipsnames]{xcolor}
%\usepackage{paralist}
%\usepackage{fixltx2e}
%\usepackage{ellipsis}
\usepackage[tracking=true]{microtype}

\usepackage{lmodern}              % Ersatz fuer Computer Modern-Schriften
%\usepackage{hfoldsty}

%\usepackage{fourier}             % Schriftart
\usepackage[scaled=0.81]{helvet}     % Schriftart

\usepackage{url}
\usepackage{xspace}
\usepackage{xcolor}
\definecolor{urlred}{HTML}{660000}

\usepackage{hyperref}
\hypersetup{colorlinks=false}

%\usepackage{mdwlist}     % Änderung der Zeilenabstände bei itemize und enumerate
% \usepackage{draftwatermark} % Wasserzeichen ``Entwurf''
% \SetWatermarkText{Antrag}

\parindent 0pt                 % Absatzeinrücken verhindern
\parskip 12pt                 % Absätze durch Lücke trennen

\setlength{\textheight}{23cm}
\usepackage{fancyhdr}
\pagestyle{fancy}
\fancyhead{} % clear all header fields
\cfoot{}
\lfoot{Zusammenkunft aller Physik-Fachschaften}
\rfoot{www.zapfev.de\\stapf@zapf.in}
\renewcommand{\headrulewidth}{0pt}
\renewcommand{\footrulewidth}{0.1pt}
\newcommand{\gen}{*innen}
\addto{\captionsngerman}{\renewcommand{\refname}{Quellen}}

%%%% Mit-TeXen Kommandoset

\newcommand{\replace}[2]{
    \sout{\textcolor{blue}{#1}}~\textcolor{blue}{#2}}
\newcommand{\delete}[1]{
    \sout{\textcolor{red}{#1}}}
\newcommand{\add}[1]{
    \textcolor{blue}{#1}}

\newif\ifcomments
\commentsfalse
%\commentstrue

\newcommand{\red}[1]{{\ifcomments\color{red} {#1}\else{#1}\fi}\xspace}
\newcommand{\blue}[1]{{\ifcomments\color{blue} {#1}\else{#1}\fi}\xspace}
\newcommand{\green}[1]{{\ifcomments\color{green} {#1}\else{#1}\fi}\xspace}

\newcommand{\repl}[2]{{\ifcomments{\color{red} \sout{#1}}{\color{blue} {\xspace #2}}\else{#2}\fi}}
%\newcommand{\repl}[2]{{\color{red} \sout{#1}\xspace{\color{blue} {#2}}\else{#2}\fi}\xspace}

\newcommand{\initcomment}[2]{%
	\expandafter\newcommand\csname#1\endcsname{%
		\def\thiscommentname{#1}%
		\definecolor{col}{rgb}{#2}%
		\def\thiscommentcolor{col}%
}}

% initcomment Name RGB-color
\initcomment{Philipp}{0, 0.5, 0}

%\renewcommand{\comment}[1]{{\ifcomments{\color{red} {#1}}{}\fi}\xspace}

\renewcommand{\comment}[2][\nobody]{
	\ifdefined#1
	{\ifcomments{#1 \expandafter\color{\thiscommentcolor}{\thiscommentname: #2}}{}\fi}\xspace
	\else
	{\ifcomments{\color{red} {#2}}{}\fi}\xspace
	\fi
}

\newcommand{\zapf}{ZaPF\xspace}

\let\oldgrqq=\grqq
\def\grqq{\oldgrqq\xspace}

\setlength{\parskip}{.6em}
\setlength{\parindent}{0mm}

\usepackage{geometry}
\geometry{left=2.5cm, right=2.5cm, top=2.5cm, bottom=3.5cm}

% \renewcommand{\familydefault}{\sfdefault}


\begin{document}

\hspace{0.87\textwidth}
\begin{minipage}{120pt}
	\vspace{-1.8cm}
	\includegraphics[width=80pt]{../logo.pdf}
	\centering
	\small Zusammenkunft aller Physik-Fachschaften
\end{minipage}

\vspace{1cm}
\begin{center}
  \huge{Einbindung von Forschungsdatenmanagement\\in der Lehre}\vspace{.25\baselineskip}\\
  \normalsize
\end{center}
\vspace{1cm}

%%%% Metadaten %%%%

%\paragraph{Addressierte:} Akkreditierungsrat (Geschäftsstelle und Mitglieder), systemakkreditierte Hochschulen, PTOs, Physik-Fachschaften 

% \paragraph{Antragstellende:} Philipp (Manitoba/Wuppertal/Alumni), Janice (Münster), Merten (CERN/Göttingen/jDPG/Alumni)

%%%% Text des Antrages zur veröffentlichung %%%%

% \section*{Antragstext}
%Warum ist NFDI für Studis wichtig?
%\item NFDI in Praktikum ansprechen
%*\item im Praktikum an besten zum Lernen
%\item Summer School für FDM veranstalten
%\item Wahlpflichtfach für NFDI im Studium
%\item Aufbau der NFDI Infos: Sollte alle Menschen aus Naturwissenschaften ansprechen
%\item Interdisziplinärer/transdisziplinärer Aufbau 
%\item Die Studierenden sollen wissen, wie man Daten FAIR strukturiert und sie mit Metadaten versehen können.
%*\item Datenstruktur und Metadaten (wissen und ggf anwenden)

Die ZaPF fordert, dass Forschungsdaten im Physikstudium stärker thematisiert werden, um den Anforderungen der fortschreitenden Digitalisierung in Wissenschaft und Gesellschaft gerecht zu werden. Deutschlandweit erarbeiten diverse Zentren an für Forschungsdatenmanagement geeigneten Konzepten und eine Positionierung der Nationalen Forschungsdateninfrastruktur (NFDI) ist absehbar.

Mögliche Anknüpfungspunkte für die entsprechenden Inhalte sind:
\begin{itemize}
	\item Integration in existierende Veranstaltungen, insbes. Praktika
	\item Schaffung von allein stehenden Modulen im Wahl- oder Wahlpflichtbereich
	\item Durchführung von Summer Schools bzw. Blockkursen für Master- und Promotionsstudierende
\end{itemize}

Zur Maximierung der Reichweite erscheinen zur Einbindung solcher Kompetenzen in das bestehende Physik-Curriculum insbesondere Praktika gut geeignet. Die Umsetzung sollte derart erfolgen, dass auch Studierende, die Grund- oder Fortgeschrittenenpraktika im Nebenfach absolvieren, davon profitieren. Geeignete Lehrinhalte sind daher disziplinübergreifend und aus einer transdisziplinären Perspektive zu bewerten.

Um eine angemessene Einbindung zu erreichen, sollten weiter auf Studiengangsebene entsprechende Lernziele definiert werden. Die unten stehenden Vorschläge folgen in der Formulierung der Bloom'schen Taxonomie von Lernzielen im kognitiven Bereich\footnote{z.B. \url{https://paeda-logics.ch/wp-content/uploads/2014/10/Taxonomiestufen_Bloom.pdf}}. Geordnet nach den Kategorien des Hochschulqualifikationsrahmens könnten diese wie folgt lauten:

\paragraph{Bachelor:}
\begin{itemize}
	\item Die Studierenden kennen die FAIR-Prinzipien und Konzepte zu Open Data.
	\item Sie können die FAIR-Prinzipien\footnote{Wilkinson, M., Dumontier, M., Aalbersberg, I. et al. \textit{The FAIR Guiding Principles for scientific data management and stewardship}. Sci Data \textbf{3}, 160018 (2016). \url{https://doi.org/10.1038/sdata.2016.18}}\xspace auf die Struktur ihrer Daten anwenden und sie mit Metadaten versehen.
\end{itemize}
\clearpage

\paragraph{Master:}
\begin{itemize}
	\item Sie können Metadaten analysieren.
	\item Sie verstehen die Relevanz von offenen, FAIRen Daten.
	\item Sie kennen das Konzept FAIRer digitaler Objekte (FDO)\footnote{De Smedt, K., Koureas, D., \& Wittenburg, P. (2020). \textit{FAIR digital objects for science: from data pieces
		to actionable knowledge units}. Publications, \textbf{8(2)}, 21.}.
\end{itemize}

\paragraph{Promotion:}
\begin{itemize}
	\item Sie können (Meta)datensätze für eigene Experimente, Rechnungen, oder anderen Datenformen synthetisieren.
	\item Sie können die Qualität verschiedener digitaler Objekte beurteilen.
\end{itemize}


\vfill
\begin{flushright}
	Verabschiedet am 23. Mai 2021 \\
	auf der Digital-ZaPF hosted in Rostock.
\end{flushright}


%%%% Text zur Begründung im Plenum %%%%

% \section*{Begründung}

% Selbsterklärend.


\end{document}



%%% Local Variables:
%%% mode: latex
%%% TeX-master: t
%%% End:

