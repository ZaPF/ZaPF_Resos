\documentclass[DIV=calc]{scrartcl}
\usepackage[utf8]{inputenc}
\usepackage[T1]{fontenc}
\usepackage[ngerman]{babel}
\usepackage{graphicx}
\usepackage[draft, markup=underlined]{changes}
\usepackage{csquotes}
\usepackage{eurosym}

\usepackage{ulem}
%\usepackage[dvipsnames]{xcolor}
\usepackage{paralist}
\usepackage{fixltx2e}
%\usepackage{ellipsis}
\usepackage[tracking=true]{microtype}

\usepackage{lmodern}              % Ersatz fuer Computer Modern-Schriften
%\usepackage{hfoldsty}

%\usepackage{fourier}             % Schriftart
\usepackage[scaled=0.81]{helvet}     % Schriftart

\usepackage{url}
%\usepackage{tocloft}             % Paket für Table of Contents
\def\UrlBreaks{\do\a\do\b\do\c\do\d\do\e\do\f\do\g\do\h\do\i\do\j\do\k\do\l%
\do\m\do\n\do\o\do\p\do\q\do\r\do\s\do\t\do\u\do\v\do\w\do\x\do\y\do\z\do\0%
\do\1\do\2\do\3\do\4\do\5\do\6\do\7\do\8\do\9\do\-}%

\usepackage{xcolor}
\definecolor{urlred}{HTML}{660000}

\usepackage{hyperref}
\hypersetup{colorlinks=false}

%\usepackage{mdwlist}     % Änderung der Zeilenabstände bei itemize und enumerate
% \usepackage{draftwatermark} % Wasserzeichen ``Entwurf''
% \SetWatermarkText{Antrag}

\parindent 0pt                 % Absatzeinrücken verhindern
\parskip 12pt                 % Absätze durch Lücke trennen

\setlength{\textheight}{23cm}
\usepackage{fancyhdr}
\pagestyle{fancy}
\fancyhead{} % clear all header fields
\cfoot{}
\lfoot{Zusammenkunft aller Physik-Fachschaften}
\rfoot{www.zapfev.de\\stapf@zapf.in}
\renewcommand{\headrulewidth}{0pt}
\renewcommand{\footrulewidth}{0.1pt}
\newcommand{\gen}{*innen}
\addto{\captionsngerman}{\renewcommand{\refname}{Quellen}}

%%%% Mit-TeXen Kommandoset
\usepackage[normalem]{ulem}
\usepackage{xcolor}
\usepackage{xspace} 

\newcommand{\replace}[2]{
    \sout{\textcolor{blue}{#1}}~\textcolor{blue}{#2}}
\newcommand{\delete}[1]{
    \sout{\textcolor{red}{#1}}}
\newcommand{\add}[1]{
    \textcolor{blue}{#1}}

\newif\ifcomments
\commentsfalse
%\commentstrue

\newcommand{\red}[1]{{\ifcomments\color{red} {#1}\else{#1}\fi}\xspace}
\newcommand{\blue}[1]{{\ifcomments\color{blue} {#1}\else{#1}\fi}\xspace}
\newcommand{\green}[1]{{\ifcomments\color{green} {#1}\else{#1}\fi}\xspace}

\newcommand{\repl}[2]{{\ifcomments{\color{red} \sout{#1}}{\color{blue} {\xspace #2}}\else{#2}\fi}}
%\newcommand{\repl}[2]{{\color{red} \sout{#1}\xspace{\color{blue} {#2}}\else{#2}\fi}\xspace}

\newcommand{\initcomment}[2]{%
	\expandafter\newcommand\csname#1\endcsname{%
		\def\thiscommentname{#1}%
		\definecolor{col}{rgb}{#2}%
		\def\thiscommentcolor{col}%
}}

% initcomment Name RGB-color
\initcomment{Philipp}{0, 0.5, 0}

%\renewcommand{\comment}[1]{{\ifcomments{\color{red} {#1}}{}\fi}\xspace}

\renewcommand{\comment}[2][\nobody]{
	\ifdefined#1
	{\ifcomments{#1 \expandafter\color{\thiscommentcolor}{\thiscommentname: #2}}{}\fi}\xspace
	\else
	{\ifcomments{\color{red} {#2}}{}\fi}\xspace
	\fi
}

\newcommand{\zapf}{ZaPF\xspace}

\let\oldgrqq=\grqq
\def\grqq{\oldgrqq\xspace}

\setlength{\parskip}{.6em}
\setlength{\parindent}{0mm}

%\usepackage{geometry}
%\geometry{left=2.5cm, right=2.5cm, top=2.5cm, bottom=3.5cm}

% \renewcommand{\familydefault}{\sfdefault}




\begin{document}

\hspace{0.87\textwidth}
\begin{minipage}{120pt}
	\vspace{-1.8cm}
	\includegraphics[width=80pt]{../logo.pdf}
	\centering
	\small Zusammenkunft aller Physik-Fachschaften
\end{minipage}

\begin{center}
  \huge{Resolution zu Studierendenwerke}\vspace{.25\baselineskip}\\
  \normalsize
\end{center}
\vspace{1cm}

%%%% Metadaten %%%%

%\paragraph{Adressierte:} Alle Landesregierungen + die Ressors die das angeht, deutsches Studentenwerk, Studierendenrat des DSW (sprecherinnen@studentenwerke.de), alle Studwerke, fzs, MeTaFa


%\paragraph{Antragstellende:} Christian(Marburg), Jan(Tübingen)

%%%% Text des Antrages zur veröffentlichung %%%%

%\section*{Antragstext}
Die ZaPF fordert die Studierendenwerke auf, ihrer Verantwortung nachzukommen, ein bezahlbares studentisches Leben zu ermöglichen.

Des Weiteren fordert die ZaPF die Landesregierungen auf, die Studierendenwerke finanziell stärker zu unterstützen.
Nur so ist es möglich, Belastungen für Studierende durch weitere Preissteigerung abzumildern.
Des Weiteren hierhinsetzen fordert die ZaPF die Studierendenwerke auf, ihrer Verantwortung nachzukommen, ein bezahlbares studentisches Leben zu ermöglichen.

Mit dem Hintergrund der steigenden Lebensmittel-(8,4\%) und Energiepreise (35,3\%) durch eine Inflation von aktuell 7,4\%\footnote{Statistisches Bundesamt:
\url{https://www.destatis.de/DE/Themen/Wirtschaft/Preise/Verbraucherpreisindex/_inhalt.html}}, sieht die ZaPF dringenden Handlungsbedarf. Im Gegensatz dazu ist im BAföGsatz bis jetzt lediglich eine Erhöhung von 5\% vorgesehen, was in Anbetracht der Teuerungsrate nicht ausreichend ist. Schon jetzt leben ca. 30\% der Studierenden unter der Armutsgrenze\footnote{\url{https://www.der-paritaetische.de/alle-meldungen/armut-im-studium-30-prozent-aller-studierenden-leben-in-armut/}}. Daher sollten die Länder ihre Studentenwerke so unterstützen, dass es in Essenseinrichtungen und Wohneinrichtungen keine weiteren rapiden Preiserhöhungen gibt.

Schon in diesem Jahr hat es in einigen Mensen bereits einen starken Anstieg der Preise gegeben.
Für viele Studierende ist die Mensa aber ein täglicher Anlaufpunkt. Dabei muss es möglich sein sich gesund und preiswert zu versorgen. Regionale Produkte und verschiedene Ernährungsformen (vegtarisch, vegan) sind ebenso wichtig wie eine gute Verfügbarkeit im studentischen Alltag. 
\newpage
Eine weitere Aufgabe der Studierendenwerke ist die Bereitstellung von bezahlbarem Wohnraum für Studierende. Die Preise für Wohnfläche haben sich in den letzte Jahren stetig erhöht, insbesondere in Ballungsräumen. Mit der Wiederaufnahme der Präsenzlehre an den Universitäten ist auch die Nachfrage nach Wohnraum wieder deutlich gestiegen. Es müssen Maßnahmen ergriffen werden, um die zunehmende Belastung der Studierenden durch weiter steigende Wohnkosten zu verhindern.

Die Preise für Wohnen und Ernährung müssen für \textbf{alle} Studierenden bezahlbar bleiben!






%\newpage
%\section*{Begründung}


\vspace{1cm} 

\vfill
\begin{flushright}
	Verabschiedet am 07. Juni 2022 \\
	auf der ZaPF in Bochum.
\end{flushright}

\end{document}
