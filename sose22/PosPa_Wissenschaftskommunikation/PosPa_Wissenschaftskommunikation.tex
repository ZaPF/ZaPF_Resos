\documentclass[DIV=calc]{scrartcl}
\usepackage[utf8]{inputenc}
\usepackage[T1]{fontenc}
\usepackage[ngerman]{babel}
\usepackage{graphicx}
\usepackage[draft, markup=underlined]{changes}
\usepackage{csquotes}
\usepackage{eurosym}

\usepackage{ulem}
%\usepackage[dvipsnames]{xcolor}
\usepackage{paralist}
\usepackage{fixltx2e}
%\usepackage{ellipsis}
\usepackage[tracking=true]{microtype}

\usepackage{lmodern}              % Ersatz fuer Computer Modern-Schriften
%\usepackage{hfoldsty}

%\usepackage{fourier}             % Schriftart
\usepackage[scaled=0.81]{helvet}     % Schriftart

\usepackage{url}
%\usepackage{tocloft}             % Paket für Table of Contents
\def\UrlBreaks{\do\a\do\b\do\c\do\d\do\e\do\f\do\g\do\h\do\i\do\j\do\k\do\l%
\do\m\do\n\do\o\do\p\do\q\do\r\do\s\do\t\do\u\do\v\do\w\do\x\do\y\do\z\do\0%
\do\1\do\2\do\3\do\4\do\5\do\6\do\7\do\8\do\9\do\-}%

\usepackage{xcolor}
\definecolor{urlred}{HTML}{660000}

\usepackage{hyperref}
\hypersetup{colorlinks=false}

%\usepackage{mdwlist}     % Änderung der Zeilenabstände bei itemize und enumerate
% \usepackage{draftwatermark} % Wasserzeichen ``Entwurf''
% \SetWatermarkText{Antrag}

\parindent 0pt                 % Absatzeinrücken verhindern
\parskip 12pt                 % Absätze durch Lücke trennen

\setlength{\textheight}{23cm}
\usepackage{fancyhdr}
\pagestyle{fancy}
\fancyhead{} % clear all header fields
\cfoot{}
\lfoot{Zusammenkunft aller Physik-Fachschaften}
\rfoot{www.zapfev.de\\stapf@zapf.in}
\renewcommand{\headrulewidth}{0pt}
\renewcommand{\footrulewidth}{0.1pt}
\newcommand{\gen}{*innen}
\addto{\captionsngerman}{\renewcommand{\refname}{Quellen}}

%%%% Mit-TeXen Kommandoset
\usepackage[normalem]{ulem}
\usepackage{xcolor}
\usepackage{xspace} 

\newcommand{\replace}[2]{
    \sout{\textcolor{blue}{#1}}~\textcolor{blue}{#2}}
\newcommand{\delete}[1]{
    \sout{\textcolor{red}{#1}}}
\newcommand{\add}[1]{
    \textcolor{blue}{#1}}

\newif\ifcomments
\commentsfalse
%\commentstrue

\newcommand{\red}[1]{{\ifcomments\color{red} {#1}\else{#1}\fi}\xspace}
\newcommand{\blue}[1]{{\ifcomments\color{blue} {#1}\else{#1}\fi}\xspace}
\newcommand{\green}[1]{{\ifcomments\color{green} {#1}\else{#1}\fi}\xspace}

\newcommand{\repl}[2]{{\ifcomments{\color{red} \sout{#1}}{\color{blue} {\xspace #2}}\else{#2}\fi}}
%\newcommand{\repl}[2]{{\color{red} \sout{#1}\xspace{\color{blue} {#2}}\else{#2}\fi}\xspace}

\newcommand{\initcomment}[2]{%
	\expandafter\newcommand\csname#1\endcsname{%
		\def\thiscommentname{#1}%
		\definecolor{col}{rgb}{#2}%
		\def\thiscommentcolor{col}%
}}

% initcomment Name RGB-color
\initcomment{Philipp}{0, 0.5, 0}

%\renewcommand{\comment}[1]{{\ifcomments{\color{red} {#1}}{}\fi}\xspace}

\renewcommand{\comment}[2][\nobody]{
	\ifdefined#1
	{\ifcomments{#1 \expandafter\color{\thiscommentcolor}{\thiscommentname: #2}}{}\fi}\xspace
	\else
	{\ifcomments{\color{red} {#2}}{}\fi}\xspace
	\fi
}

\newcommand{\zapf}{ZaPF\xspace}

\let\oldgrqq=\grqq
\def\grqq{\oldgrqq\xspace}

\setlength{\parskip}{.6em}
\setlength{\parindent}{0mm}

%\usepackage{geometry}
%\geometry{left=2.5cm, right=2.5cm, top=2.5cm, bottom=3.5cm}

% \renewcommand{\familydefault}{\sfdefault}




\begin{document}

\hspace{0.87\textwidth}
\begin{minipage}{120pt}
	\vspace{-1.8cm}
	\includegraphics[width=80pt]{../logo.pdf}
	\centering
	\small Zusammenkunft aller Physik-Fachschaften
\end{minipage}

\begin{center}
  \huge{Positionspapier Wissenschaftskommunikation}\vspace{.25\baselineskip}\\
  \normalsize
\end{center}
\vspace{1cm}

%%%% Metadaten %%%%

%\paragraph{Adressierte:} : 

%\paragraph{Antragstellende:} Matthias (Würzburg), Helena (Wien), Stefan (Wien) et. al.

%%%% Text des Antrages zur veröffentlichung %%%%

\subsection*{Ethik und Verantwortung in der Wissenschaftskommunikation}

Forschung ist immer im Kontext ihrer Anwendung, Durchführung und Finanzierung zu sehen. Diese Punkte beinhalten oft eine ethische Dimension, seien es Rüstungsmittel, Tierversuche oder Gentechnik. Die ZaPF wirkt darauf hin, dass dieser ethische Kontext stets berücksichtigt wird.

\subsubsection*{Klarstellung von Nachhaltigkeit als ethische Dimension}

Die Nachhaltigkeit, wie wir sie hier als Begriff verwenden, ist nach dem Brundtland-Report definiert als „eine Entwicklung, die die Bedürfnisse der Gegenwart befriedigt, ohne zu riskieren, dass künftige Generationen ihre eigenen Bedürfnisse nicht befriedigen können“\footnote{Report of the World Commission on Environment and Development; Our Common Future, United Nations, 1987}.

Die Verpflichtung zum Handeln nach einem Nachhaltigkeitsbegriff ist mittlerweile gesamtgesellschatlich, und auch in der wissenschaftlichen Community weitgehend konsens, so zum Beispiel die Unterstützenden von Scientists for Future\footnote{
Charta von Scientists for Future;
Selbstverständnis, \url{https://de.scientists4future.org/ueber-uns/charta/, 2022-06-06 13:00}}. Zum Handeln nach dieser Maxime bedarf es, wie auch die United Nations (UN) beschreibt, der Unterstützung von "Environmentally Sound Technologies"\footnote{Why Does Technology Matter, United Nations, Environment Programme, \url{https://www.unep.org/explore-topics/technology/why-does-technology-matter, 2022-06-06 13:00}}

Wissenschaft ist, sofern sie das Ziel der Nachhaltigkeit verfolgt, verpflichtet sich mit den ökologischen Folgen der eigenen Forschung eingehend zu beschäftigen. 

Wenn wir uns als Wissenschaftler dieser Verantwortung mit Konsequenz nähern wollen, müssen wir diese Konsequenz besonders uns selbst gegenüber zeigen. Dies bedingt das festlegen ethischer Richtlinien in der Frage der Nachhaltigkeit und zu klären was diese Konsequenz für uns bedeutet.

\subsubsection*{Mitbetrachtung von ethischem Kontext bei jeglicher Kommunikation von Forschung}

Da, wie zu Anfang genannt, Forschung immer in Kontext zu setzen ist, muss dieser auch in ihrer Kommunikation Betrachtung finden. So muss auch in der wissenschaftlichen Diskussion, z.B. Fachvorträgen, eine Beschäftigung mit dem Kontext und den möglichen Folgen der Technologie stattfinden. Diese Betrachtung muss von vornherein Teil der Diskussion sein. So muss zum Beispiel bei einem Vortrag über ethisch fragwürdige Themen bereits bei der Planung des Vortrags eine Beschäftigung mit dieser Dimension geplant werden.

Beispiele sind etwa Dual-Use Forschung, oder Fracking. Beide sind Technologien, die von verantwortungsvollen Forschenden nicht unreflektiert diskutiert werden sollten.

Die ZaPF fordert daher, bei jeder Kommunikation von Wissenschaft und Technologie eine Befassung mit ihrer ethischen Dimension. Sei die Kommunikation auch nur gegenüber fachnahen Personen, etwa innerhalb eines physikalischen Instituts.

%\newpage
%\section*{Begründung}


\vspace{1cm} 

\vfill
\begin{flushright}
	Verabschiedet am 07. Juni 2022 \\
	auf der ZaPF in Bochum.
\end{flushright}

\end{document}
