\documentclass[DIV=calc]{scrartcl}
\usepackage[utf8]{inputenc}
\usepackage[T1]{fontenc}
\usepackage[ngerman]{babel}
\usepackage{graphicx}
\usepackage[draft, markup=underlined]{changes}
\usepackage{csquotes}
\usepackage{eurosym}

\usepackage{ulem}
%\usepackage[dvipsnames]{xcolor}
\usepackage{paralist}
\usepackage{fixltx2e}
%\usepackage{ellipsis}
\usepackage[tracking=true]{microtype}

\usepackage{lmodern}              % Ersatz fuer Computer Modern-Schriften
%\usepackage{hfoldsty}

%\usepackage{fourier}             % Schriftart
\usepackage[scaled=0.81]{helvet}     % Schriftart

\usepackage{url}
%\usepackage{tocloft}             % Paket für Table of Contents
\def\UrlBreaks{\do\a\do\b\do\c\do\d\do\e\do\f\do\g\do\h\do\i\do\j\do\k\do\l%
\do\m\do\n\do\o\do\p\do\q\do\r\do\s\do\t\do\u\do\v\do\w\do\x\do\y\do\z\do\0%
\do\1\do\2\do\3\do\4\do\5\do\6\do\7\do\8\do\9\do\-}%

\usepackage{xcolor}
\definecolor{urlred}{HTML}{660000}

\usepackage{hyperref}
\hypersetup{colorlinks=false}

%\usepackage{mdwlist}     % Änderung der Zeilenabstände bei itemize und enumerate
% \usepackage{draftwatermark} % Wasserzeichen ``Entwurf''
% \SetWatermarkText{Antrag}

\parindent 0pt                 % Absatzeinrücken verhindern
\parskip 12pt                 % Absätze durch Lücke trennen

\setlength{\textheight}{23cm}
\usepackage{fancyhdr}
\pagestyle{fancy}
\fancyhead{} % clear all header fields
\cfoot{}
\lfoot{Zusammenkunft aller Physik-Fachschaften}
\rfoot{www.zapfev.de\\stapf@zapf.in}
\renewcommand{\headrulewidth}{0pt}
\renewcommand{\footrulewidth}{0.1pt}
\newcommand{\gen}{*innen}
\addto{\captionsngerman}{\renewcommand{\refname}{Quellen}}

%%%% Mit-TeXen Kommandoset
\usepackage[normalem]{ulem}
\usepackage{xcolor}
\usepackage{xspace} 

\newcommand{\replace}[2]{
    \sout{\textcolor{blue}{#1}}~\textcolor{blue}{#2}}
\newcommand{\delete}[1]{
    \sout{\textcolor{red}{#1}}}
\newcommand{\add}[1]{
    \textcolor{blue}{#1}}

\newif\ifcomments
\commentsfalse
%\commentstrue

\newcommand{\red}[1]{{\ifcomments\color{red} {#1}\else{#1}\fi}\xspace}
\newcommand{\blue}[1]{{\ifcomments\color{blue} {#1}\else{#1}\fi}\xspace}
\newcommand{\green}[1]{{\ifcomments\color{green} {#1}\else{#1}\fi}\xspace}

\newcommand{\repl}[2]{{\ifcomments{\color{red} \sout{#1}}{\color{blue} {\xspace #2}}\else{#2}\fi}}
%\newcommand{\repl}[2]{{\color{red} \sout{#1}\xspace{\color{blue} {#2}}\else{#2}\fi}\xspace}

\newcommand{\initcomment}[2]{%
	\expandafter\newcommand\csname#1\endcsname{%
		\def\thiscommentname{#1}%
		\definecolor{col}{rgb}{#2}%
		\def\thiscommentcolor{col}%
}}

% initcomment Name RGB-color
\initcomment{Philipp}{0, 0.5, 0}

%\renewcommand{\comment}[1]{{\ifcomments{\color{red} {#1}}{}\fi}\xspace}

\renewcommand{\comment}[2][\nobody]{
	\ifdefined#1
	{\ifcomments{#1 \expandafter\color{\thiscommentcolor}{\thiscommentname: #2}}{}\fi}\xspace
	\else
	{\ifcomments{\color{red} {#2}}{}\fi}\xspace
	\fi
}

\newcommand{\zapf}{ZaPF\xspace}

\let\oldgrqq=\grqq
\def\grqq{\oldgrqq\xspace}

\setlength{\parskip}{.6em}
\setlength{\parindent}{0mm}

%\usepackage{geometry}
%\geometry{left=2.5cm, right=2.5cm, top=2.5cm, bottom=3.5cm}

% \renewcommand{\familydefault}{\sfdefault}




\begin{document}

\hspace{0.87\textwidth}
\begin{minipage}{120pt}
	\vspace{-1.8cm}
	\includegraphics[width=80pt]{../logo.pdf}
	\centering
	\small Zusammenkunft aller Physik-Fachschaften
\end{minipage}

\begin{center}
  \huge{Resolution zur niederschwelligen Bereitstellung von Menstruationsprodukten an Hochschulen}\vspace{.25\baselineskip}\\
  \normalsize
\end{center}
\vspace{1cm}

%%%% Metadaten %%%%
%
%\paragraph{Adressierte:} Hochschulrektorkonferenz (HRK), Kultusministerkonferenz (KMK), BMBWF (Bundesministerium für Bildung, Wissenschaft und Forschung -Österreich), BMBF, fzs,  Landes-ASten-Konferenzen
%
%
%\paragraph{Antragstellende:} Virgina (Oldenburg), Jasemin (Münster), Tom (Jena)

%%%% Text des Antrages zur veröffentlichung %%%%


%\section*{Antragstext}
%Die ZaPF schließt sich folgendem Antrag der KoMa an und beauftragt den StAPF diesen an die oben stehenden Adressat*innen zu versenden:\\
%\vspace{1em}

%\textbf{Resolution zur niederschwelligen Bereitstellung von Menstruationsprodukten an Hochschulen}\\


Sehr geehrte Adressat*innen,\\
\\
Für viele Studierende gibt es heutzutage weiterhin Probleme, während ihrer Menstruation am Hochschulbetrieb und insbesondere an Lehrveranstaltungen teilzunehmen, da sie sich beispielsweise keine Menstruationsprodukte leisten können oder die Menstruation überraschend einsetzt. Neben den gesellschaftlichen Auswirkungen birgt dies auch gesundheitliche Risiken wie gravierende Infektionsgefahren\footnote{\url{https://www.rki.de/DE/Content/Infekt/EpidBull/Merkblaetter/Ratgeber_Staphylokokken_MRSA.html}} "Oft müssen Menstruierende sich [in solchen Notsituationen] mit Klopapier oder anderen unhygienischen Alternativen aushelfen, bis sie die Zeit haben, nach Hause zu gehen oder sich Menstruationsartikel zu kaufen. Wenn Menschen eine starke Periode haben, reicht Klopapier nicht aus und die Lehrveranstaltung kann gar nicht besucht werden."\footnote{\url{https://www.fzs.de/2021/06/17/kostenlose-menstruationsprodukte-in-allen-bildungseinrichtungen/}} Abhilfe können niederschwellige Angebote an kostenlosen Menstruationsprodukten schaffen, wie sie zum Beispiel in den Universitäten Regensburg, Bonn, Graz und Wien bereits angeregt/geschaffen wurden. 

Deshalb unterstützen wir, die 86. Konferenz der deutschsprachigen Mathematikfachschaften, nachdrücklich den zitierten offenen Brief des freien Zusammenschlusses von Student*innenschaften zu kostenlosen Menstruationsartikeln in öffentlichen Bildungseinrichtungen. Bis die Finanzierung länderweit getragen wird, sollen die Hochschulen sich dieser annehmen. Im Detail fordern wir
\begin{itemize}
    \item die kostenfreie Bereitstellung auf allen Toiletten - im Sinne der Inklusion \& Diversität auch über Damentoiletten hinaus,
    \item die Sicherstellung der ständigen Verfügbarkeit auf den Toiletten, vorzugsweise in allen Toilettenkabinen und
    \item die Zurverfügungstellung von Mülleimern und Hygienebeuteln zur Entsorgung in allen Toilettenkabinen.
    
\end{itemize}

Die Finanzierung dessen darf nicht auf die Studierendenschaft zurückfallen, da es sich bei Menstruationsprodukten genauso um Grund-Hygienebedarf handelt, wie etwa bei Toilettenpapier.
Die aktuelle Situation ist nicht tragbar.\\


%\section*{Begründung}
%Am 29.05.2022 hat die KoMa den folgenden Antrag zu der Bereitstellung von Menstruationsprodukten an Universitäten beschlossen. Dies ist ein wichtiges Thema, welches die ZaPF auch unterstützen sollte. Wir möchten deswegen diesen Beschluss auch von Seiten der ZaPF teilen.

%\newpage
%\section*{Begründung}


\vspace{1cm} 

\vfill
\begin{flushright}
	Verabschiedet am 07. Juni 2022 \\
	auf der ZaPF in Bochum.
\end{flushright}

\end{document}
