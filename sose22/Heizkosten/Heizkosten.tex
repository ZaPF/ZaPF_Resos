\documentclass[DIV=calc]{scrartcl}
\usepackage[utf8]{inputenc}
\usepackage[T1]{fontenc}
\usepackage[ngerman]{babel}
\usepackage{graphicx}
\usepackage[draft, markup=underlined]{changes}
\usepackage{csquotes}
\usepackage{eurosym}

\usepackage{ulem}
%\usepackage[dvipsnames]{xcolor}
\usepackage{paralist}
\usepackage{fixltx2e}
%\usepackage{ellipsis}
\usepackage[tracking=true]{microtype}

\usepackage{lmodern}              % Ersatz fuer Computer Modern-Schriften
%\usepackage{hfoldsty}

%\usepackage{fourier}             % Schriftart
\usepackage[scaled=0.81]{helvet}     % Schriftart

\usepackage{url}
%\usepackage{tocloft}             % Paket für Table of Contents
\def\UrlBreaks{\do\a\do\b\do\c\do\d\do\e\do\f\do\g\do\h\do\i\do\j\do\k\do\l%
\do\m\do\n\do\o\do\p\do\q\do\r\do\s\do\t\do\u\do\v\do\w\do\x\do\y\do\z\do\0%
\do\1\do\2\do\3\do\4\do\5\do\6\do\7\do\8\do\9\do\-}%

\usepackage{xcolor}
\definecolor{urlred}{HTML}{660000}

\usepackage{hyperref}
\hypersetup{colorlinks=false}

%\usepackage{mdwlist}     % Änderung der Zeilenabstände bei itemize und enumerate
% \usepackage{draftwatermark} % Wasserzeichen ``Entwurf''
% \SetWatermarkText{Antrag}

\parindent 0pt                 % Absatzeinrücken verhindern
\parskip 12pt                 % Absätze durch Lücke trennen

\setlength{\textheight}{23cm}
\usepackage{fancyhdr}
\pagestyle{fancy}
\fancyhead{} % clear all header fields
\cfoot{}
\lfoot{Zusammenkunft aller Physik-Fachschaften}
\rfoot{www.zapfev.de\\stapf@zapf.in}
\renewcommand{\headrulewidth}{0pt}
\renewcommand{\footrulewidth}{0.1pt}
\newcommand{\gen}{*innen}
\addto{\captionsngerman}{\renewcommand{\refname}{Quellen}}

%%%% Mit-TeXen Kommandoset
\usepackage[normalem]{ulem}
\usepackage{xcolor}
\usepackage{xspace} 

\newcommand{\replace}[2]{
    \sout{\textcolor{blue}{#1}}~\textcolor{blue}{#2}}
\newcommand{\delete}[1]{
    \sout{\textcolor{red}{#1}}}
\newcommand{\add}[1]{
    \textcolor{blue}{#1}}

\newif\ifcomments
\commentsfalse
%\commentstrue

\newcommand{\red}[1]{{\ifcomments\color{red} {#1}\else{#1}\fi}\xspace}
\newcommand{\blue}[1]{{\ifcomments\color{blue} {#1}\else{#1}\fi}\xspace}
\newcommand{\green}[1]{{\ifcomments\color{green} {#1}\else{#1}\fi}\xspace}

\newcommand{\repl}[2]{{\ifcomments{\color{red} \sout{#1}}{\color{blue} {\xspace #2}}\else{#2}\fi}}
%\newcommand{\repl}[2]{{\color{red} \sout{#1}\xspace{\color{blue} {#2}}\else{#2}\fi}\xspace}

\newcommand{\initcomment}[2]{%
	\expandafter\newcommand\csname#1\endcsname{%
		\def\thiscommentname{#1}%
		\definecolor{col}{rgb}{#2}%
		\def\thiscommentcolor{col}%
}}

% initcomment Name RGB-color
\initcomment{Philipp}{0, 0.5, 0}

%\renewcommand{\comment}[1]{{\ifcomments{\color{red} {#1}}{}\fi}\xspace}

\renewcommand{\comment}[2][\nobody]{
	\ifdefined#1
	{\ifcomments{#1 \expandafter\color{\thiscommentcolor}{\thiscommentname: #2}}{}\fi}\xspace
	\else
	{\ifcomments{\color{red} {#2}}{}\fi}\xspace
	\fi
}

\newcommand{\zapf}{ZaPF\xspace}

\let\oldgrqq=\grqq
\def\grqq{\oldgrqq\xspace}

\setlength{\parskip}{.6em}
\setlength{\parindent}{0mm}

%\usepackage{geometry}
%\geometry{left=2.5cm, right=2.5cm, top=2.5cm, bottom=3.5cm}

% \renewcommand{\familydefault}{\sfdefault}




\begin{document}

\hspace{0.87\textwidth}
\begin{minipage}{120pt}
	\vspace{-1.8cm}
	\includegraphics[width=80pt]{../logo.pdf}
	\centering
	\small Zusammenkunft aller Physik-Fachschaften
\end{minipage}

\begin{center}
  \huge{Resolution zum Heizkostenzuschuss}\vspace{.25\baselineskip}\\
  \normalsize
\end{center}
\vspace{1cm}

%%%% Metadaten %%%%

%\paragraph{Adressierte:} Deutschen Studierendenwerke (DSW), Bildungspolitische Sprecher aller Parteien im deutschen Bundestag, Sozialpolitische Sprecher aller Parteien im deutschen Bundestag, MeTaFa, fzs
%
%
%\paragraph{Antragstellende:} Peter (Alumni),  Kai (JLU Gießen)

%%%% Text des Antrages zur veröffentlichung %%%%

%\section*{Antragstext}
Die kürzlich beschlossenen Entlastungspakete der Bundesregierung sollen helfen, die Last der gestiegenen Energiepreise abzumildern. Einkommenssteuerzahler erhalten dafür die Energiepreispauschale.\footnote{\url{https://www.bundesfinanzministerium.de/Content/DE/Gesetzestexte/Gesetze_Gesetzesvorhaben/Abteilungen/Abteilung_IV/20_Legislaturperiode/2022-05-27-StEntlastG2022/4-Verkuendetes-Gesetz.pdf?__blob=publicationFile&v=2}} BAföG-Beziehende werden durch den Heizkostenzuschuss entlastet.\footnote{\url{https://www.gesetze-im-internet.de/heizkzuschg/BJNR069800022.html}}
Studierende, welche hingegen kein BAföG beziehen, was mittlerweile mit steigender Tendenz auf über 89 \% von ihnen zutrifft
\footnote{\url{https://www-genesis.destatis.de/genesis/online} Zahlen der BaFÖG beziehenden Studierenden, \url{https://www.destatis.de/DE/Presse/Pressemitteilungen/2020/12/PD20_497_213.html}}, erhalten durch diese Aufteilung gar keine Unterstützung und bleiben auf den gestiegenen Energiekosten sitzen.\\
Die ZaPF fordert daher die Bundesregierung auf, diesen Missstand zu beheben und auch für Studierende außerhalb des BAföGs einen angemessenen finanzielle Entlastung zu schaffen.




%\section*{Begründung}
%Vor dem Hintergrund der stark gestiegenen Energiepreise hat die Bundesregierung aktuell mit den Gesetzen zur Energiepreispauschale und dem Heizkostenzuschuss Entlastungen auf den Weg gebracht. Die Energiepreispauschale wird dabei an Steuerzahlende ausgeschüttet, den Heizkostenzuschuss bekommen vor allem Wohngeld- und BAföG-Empfangende. Studierende liegen mit ihrem Einkommen in aller Regel unter dem Grundfreibetrag der Einkommessteuer. Gleichzeitig beziehen nur 11\% von ihnen überhaupt BAföG. Der absolute Großteil der Studierenden fällt also bei der aktuell von der Bundesregierung vorgesehenen Unterteilung komplett raus. Und das dem Umstand zum Trotze, dass gerade die Studierenden es sind, die in der aktuellen Situation ohnehin schon überproportional mehrbelastet sind.
%\section*{Begründung}
%Vor dem Hintergrund der stark gestiegenen Energiepreise hat die Bundesregierung aktuell mit den Gesetzen zur Energiepreispauschale und dem Heizkostenzuschuss Entlastungen auf den Weg gebracht. Die Energiepreispauschale wird dabei an Steuerzahlende ausgeschüttet, den Heizkostenzuschuss bekommen vor allem Wohngeld- und BAföG-Empfangende. Studierende liegen mit ihrem Einkommen in aller Regel unter dem Grundfreibetrag der Einkommessteuer. Gleichzeitig beziehen nur 11\% von ihnen überhaupt BAföG. Der absolute Großteil der Studierenden fällt also bei der aktuell von der Bundesregierung vorgesehenen Unterteilung komplett raus. Und das dem Umstand zum Trotze, dass gerade die Studierenden es sind, die in der aktuellen Situation ohnehin schon überproportional mehrbelastet sind.

%\newpage
%\section*{Begründung}


\vspace{1cm} 

\vfill
\begin{flushright}
	Verabschiedet am 07. Juni 2022 \\
	auf der ZaPF in Bochum.
\end{flushright}

\end{document}
