\documentclass[DIV=calc]{scrartcl}
\usepackage[utf8]{inputenc}
\usepackage[T1]{fontenc}
\usepackage[ngerman]{babel}
\usepackage{graphicx}
\usepackage[draft, markup=underlined]{changes}
\usepackage{csquotes}

\usepackage{ulem}
%\usepackage[dvipsnames]{xcolor}
\usepackage{paralist}
\usepackage{fixltx2e}
%\usepackage{ellipsis}
\usepackage[tracking=true]{microtype}
\usepackage[a4paper,top=3cm,bottom=2cm,left=3cm,right=3cm]{geometry}

\usepackage{lmodern}              % Ersatz fuer Computer Modern-Schriften
%\usepackage{hfoldsty}

%\usepackage{fourier}             % Schriftart
\usepackage[scaled=0.81]{helvet}     % Schriftart

\usepackage{url}
\usepackage{tocloft}             % Paket für Table of Contents

%\usepackage{xcolor}
\usepackage{ulem,xspace,xcolor}
\definecolor{urlred}{HTML}{660000}

\usepackage{hyperref}
\hypersetup{
    colorlinks=true,
    linkcolor=black,    % Farbe der internen Links (u.a. Table of Contents)
    urlcolor=black,    % Farbe der url-links
    citecolor=black} % Farbe der Literaturverzeichnis-Links

\usepackage{mdwlist}     % Änderung der Zeilenabstände bei itemize und enumerate
\usepackage{draftwatermark} % Wasserzeichen ``Entwurf''
\SetWatermarkText{}

\parindent 0pt                 % Absatzeinrücken verhindern
\parskip 12pt                 % Absätze durch Lücke trennen

\setlength{\textheight}{23cm}
\usepackage{fancyhdr}
\pagestyle{fancy}
\fancyhead{} % clear all header fields
\cfoot{}
\lfoot{Zusammenkunft aller Physik-Fachschaften}
\rfoot{www.zapfev.de\\stapf@zapf.in}
\renewcommand{\headrulewidth}{0pt}
\renewcommand{\footrulewidth}{0.1pt}
\newcommand{\gen}{*innen}
\addto{\captionsngerman}{\renewcommand{\refname}{Quellen}}

%%%% Mit-TeXen Kommandoset
\usepackage[normalem]{ulem}
\usepackage{xcolor}

\usepackage{enumitem}
\setenumerate[1]{label=\thesection.\arabic*.}

\def\zapf{ZaPF}

\newif\ifcomments
\commentsfalse
\commentstrue

\begin{document}
  \hspace{0.87\textwidth}
  \begin{minipage}{120pt}
  	\vspace{-1.8cm}
  	\includegraphics[width=80pt]{../logo.pdf}
  	\centering
  	\small Zusammenkunft aller Physik-Fachschaften
  \end{minipage}
  \begin{center}
    \huge{Positionspapier zu FAIR und Open Data im physikalischen Praktikum}\vspace{.25\baselineskip}\\
  	\normalsize
  \end{center}
  \vspace{1cm}
    %
    % \begin{center}
    %     \normalsize
    % \end{center}
    % \vspace{1cm}

%%%% Text des Antrags %%%%

Die ZaPF spricht sich dafür aus, dass Studierende im Rahmen der Praktika bereits ab dem Bachelor-Studium lernen, 
wie Daten nach den FAIR-Prinzipien \footnote{\url{https://www.go-fair.org/fair-principles/}} erhoben, gespeichert und veröffentlicht werden. 
Diese Kompetenz ist Voraussetzung für datenbasierte Zusammenarbeit, etwa im Kontext der NFDI-Initiative, in der die ZaPF große Chancen sieht \footnote{\url{https://zapfev.de/resolutionen/wise19/nfdi/nfdi.pdf}}. 
Sie sind unter anderem in den Arbeitsgebieten der Einbindung in die Lehre und Ausbildung der Anträge der NFDI-Konsortien \textit{NFDI4Phys} \footnote{\url{https://www.dfg.de/download/pdf/foerderung/programme/nfdi/absichtserklaerungen_2020/2021_nfdi_4phys.pdf}},\textit{PUNCH4NFDI} \footnote{\url{https://www.dfg.de/download/pdf/foerderung/programme/nfdi/absichtserklaerungen_2020/2020_nfdi_punch4nfdi.pdf}} oder \textit{FAIRMAT} \footnote{\url{https://doi.org/10.5281/zenodo.4074973}}
und vielen anderen formuliert. \\ 

Studierende sollten nach Möglichkeit schrittweise in die vollständige, nachvollziehbare und korrekte Daten- und Metadatenerfassung eingeführt werden.
Darüber hinaus sollen Studierende den Umgang mit langfristig verwalteten Datenspeichern, soge-nannten Repositorien, kennenlernen, um dadurch in die langfristige Speicherung von Forschungsdateneingeführt zu werden. 
Neben offenen Repositorien wie Zenodo \footnote{\url{https://www.zenodo.org}} können dies auch universitätseigene Datenrepositorien wie z.B. \footnote{Beispielsweise gibt es an der Universität Göttingen das Portal GRO.Data: \url{https://data.goettingen-research-online.de/}} sein. 
Um den Wert offener Datenquellen kennenzulernen, sollen diese Repositorien auch dazu verwendet werden, fremde Daten nachzunutzen. 
Hierbei sollte insbesondereauf Aspekte wie Lizenzen oder korrekte Datenzitation eingegangen werden.
In Grund- und Anfänger:innenpraktika ist diese Praxis ein "Grundstein für gutes wissenschaftliches Arbeiten" \footnote{\url{https://zapfev.de/resolutionen/sose17/Praktika/PosPapier_Praktika.pdf}}, 
wohingegen sie besonders in umfangreicheren Fortgeschrittenenpraktika als Vorbereitung auf "die aktuelle Laborpraxis" und "das korrekte wissenschaftliche Arbeiten" \footnote{\url{https://zapfev.de/resolutionen/sose19/Fortgeschrittenenpraktika/Fortgeschrittenenpraktika.pdf}} hilfreich ist.
Bisher sind die Daten, die im (Anfänger:innen-)Praktikum erhoben werden, in der Regel weder nachvollziehbar, noch vollständig gespeichert. 
Somit kann der Lerneffekt des Praktikums in Bezug auf Datenerfassung und -speicherung als eher gering eingeschätzt werden. 
Dabei bieten die Praktika oft die einzige Möglichkeit, den Umgang mit (Forschungs-)Daten vor Forschungarbeiten im Rahmen des Studiums kennenzulernen.
Konkreter gesprochen sollen Studierende in den Praktika die Möglichkeit haben, Daten sowohl von ihren Kommiliton:innen, als auch der früheren Kohorten einzusehen und mit auszuwerten, 
um Messfehler besser einschätzen zu können und die Diskussion von Versuchsergebnissen mit einer größerenSicherheit durchführen zu können. 
Außerdem sollen Studierende die Möglichkeit erhalten, in einem Semester aufgenommene Daten eines Versuchs zu speichern 
und diese Daten dann im darauffolgenden Semester gemeinsam mit einem größeren Datensatz auszuwerten. 
Zusätzlich kann dann zwischen verschiedenen Ausgangsbedingungen und Aufbauten verglichen und der Einfluss verschiedener Parameter untersucht werden.
Ein wichtiger Teil nachhaltiger Forschung ist, dass Forschungsdaten nicht nur langfristig verfügbar, sondern vor allen Dingen für andere Wissenschaftler:innen verständlich und nutzbar bleiben. 
In letzter Konsequenz profitieren von einem Praktikum, das nach den FAIR-Prinzipien gestaltet wird, neben den Studierenden auch die Forschungsgruppen. 
Ein abwechslungreicherer Praktikumsbetrieb motiviert hierbei auch die Studierenden zusätzlich. 
Schlussendlich ernten die Forschungsgruppen, denen die Studierenden während ihrer Projektphasen angehören und so bereits wissen, wie man mit Forschungsdaten umgeht, die Erfolge dieses Konzepts. \\

Gerade im aufkeimenden "Post-Truth-Zeitalter" kommt Open Science eine kaum zu überschätzende gesellschaftliche Bedeutung zu \footnote{\url{https://doi.org/10.1007/978-981-15-4276-3_2}}. 
Wissenschaftler:innen von morgen sollten deshalb so früh wie möglich an das Konzept herangeführt werden, um so "Open Science" als "Science just done right" zu verinnerlichen \footnote{\url{https://zapfev.de/resolutionen/wise18/Reso_Open_Science/Resolution_Open_Science.pdf}}.

\vfill
    \begin{flushright}
      Verabschiedet am 15.11.2020 \\auf der Online-ZaPF hosted in Garching.
    \end{flushright}
\end{document}
