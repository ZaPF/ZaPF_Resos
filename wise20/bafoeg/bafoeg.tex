\documentclass[DIV=calc]{scrartcl}
\usepackage[utf8]{inputenc}
\usepackage[T1]{fontenc}
\usepackage[ngerman]{babel}
\usepackage{graphicx}
\usepackage[draft, markup=underlined]{changes}
\usepackage{csquotes}
\usepackage{eurosym}
\usepackage{ulem}
%\usepackage[dvipsnames]{xcolor}
\usepackage{paralist}
\usepackage{fixltx2e}
%\usepackage{ellipsis}
\usepackage[tracking=true]{microtype}

\usepackage{lmodern}              % Ersatz fuer Computer Modern-Schriften
%\usepackage{hfoldsty}

%\usepackage{fourier}             % Schriftart
\usepackage[scaled=0.81]{helvet}     % Schriftart

\usepackage[hyphens]{url}
\usepackage{hyperref}
\usepackage{tocloft}             % Paket für Table of Contents

\usepackage{xcolor}
\definecolor{urlred}{HTML}{660000}

\hypersetup{
    colorlinks=true,    
    linkcolor=black,    % Farbe der internen Links (u.a. Table of Contents)
    urlcolor=black,    % Farbe der url-links
    citecolor=black} % Farbe der Literaturverzeichnis-Links

\usepackage{mdwlist}     % Änderung der Zeilenabstände bei itemize und enumerate
\usepackage{draftwatermark} % Wasserzeichen ``Entwurf'' 
\SetWatermarkText{}

\parindent 0pt                 % Absatzeinrücken verhindern
\parskip 12pt                 % Absätze durch Lücke trennen

\setlength{\textheight}{23cm}
\usepackage{fancyhdr}
\pagestyle{fancy}
\fancyhead{} % clear all header fields
\cfoot{}
\lfoot{Zusammenkunft aller Physik-Fachschaften}
\rfoot{www.zapfev.de\\stapf@zapf.in}
\renewcommand{\headrulewidth}{0pt}
\renewcommand{\footrulewidth}{0.1pt}
\newcommand{\gen}{*innen}
\addto{\captionsngerman}{\renewcommand{\refname}{Quellen}}

%%%% Mit-TeXen Kommandoset
\usepackage[normalem]{ulem}
\usepackage{xcolor}

\newcommand{\replace}[2]{
    \sout{\textcolor{blue}{#1}}~\textcolor{blue}{#2}}
\newcommand{\delete}[1]{
    \sout{\textcolor{red}{#1}}}
\newcommand{\add}[1]{
    \textcolor{blue}{#1}}


\begin{document}
    \hspace{0.87\textwidth}
    \begin{minipage}{120pt}
        \vspace{-1.8cm}
      \includegraphics[width=80pt]{../logo.pdf}
        \centering
        \small Zusammenkunft aller Physik-Fachschaften
    \end{minipage}
    \begin{center}
        \Huge{Forderungskatalog an eine BAföG-Novellierung}\vspace{.25\baselineskip}\\
        \normalsize
    \end{center}
    \vspace{1cm}

\section*{Inhaltsverzeichnis}
\begin{enumerate}
    \item Automatische Anpassung der BAföG-Sätze
    \item Wegfall der maximalen Förderungsdauer
    \item Elternunabhängiges BAföG für alle
    \item Realistische Wohngeldpauschale
    \item Bundesweit einheitliche Online-Beantragung
    \item Vorläufiger Antrag
    \item Abschaffung des Leistungsnachweises
    \item Streichung der Altershöchstgrenze
    \item Honorierung von Ehrenämtern
\end{enumerate}

\newpage

Im Rahmen der letzten Überarbeitung des BAföG beschrieb der Deutsche Bundestag den Grundgedanken hinter dem Gesetz wie folgt:

\glqq Das Ziel des Bundesausbildungsförderungsgesetzes (BAföG) ist es, jungen Menschen unabhängig von ihrer sozialen und wirtschaftlichen Situation eine Ausbildung zu ermöglichen. Das BAföG ist das Instrument, um Chancengleichheit in der Bildungsbiografie sicherzustellen.\grqq{}\footnote{\url{https://www.bundesrat.de/SharedDocs/drucksachen/2019/0201-0300/zu216-19.pdf?__blob=publicationFile&v=1}}

Dieser Zielsetzung stimmen wir prinzipell zu, sehen dies allerdings aktuell nicht erfüllt. Aus diesem Grund fordern wir die Bundesregierung sowie alle bundespolitischen Parteien dazu auf, sich für eine Novellierung des BAföG einzusetzen, welche dem ausgegebenen Ziel entspricht.

\large{Wir fordern:}

\section{Automatische Anpassung der BAföG-Sätze}

Die Förder- und Freibeträge des BAföG müssen in regelmäßigen Abständen evaluiert und an die aktuelle Situation angepasst werden. Hierbei sollten alle relevanten Faktoren wie Inflation oder Lebenshaltungskosten einbezogen werden.

Wir sprechen uns an dieser Stelle für eine jährliche Aktualisierung aus, welche nicht durch eine Fehlplanung des Haushaltes oder anderen Ausreden verschoben werden darf. 

Entsprechend der 21. Sozialerhebung\footnote{\url{http://www.sozialerhebung.de/download/21/Soz21_hauptbericht.pdf}} deckt das BAföG lediglich 12\% der monatlichen Einkünfte der Studierenden, was eine Reduktion von 7\% gegenüber 2012 ist. Auch die Novellierung im Jahr 2019 hat diesen Missstand nachweislich nicht beheben können.

Studierende dürfen nicht darauf angewiesen sein, sich den Lebensunterhalt durch Nebeneinkünfte finanzieren zu müssen! Vor allem die Corona-Krise hat dies deutlich gemacht. Die Sozialerhebung zeigte, dass über 40\% der Studierenden, welche einen Job haben, diesen als \glqq notwendig für den Lebensunterhalt\grqq{} bezeichnen. Diese Studierenden sollten frei von finanziellen Sorgen studieren können.

Wenn die Entscheidung über die Weiterführung eines Studiums von den eigenen Einkünften abhängt, fördert dies die soziale Ungerechtigkeit nicht nur im Moment, sondern auch für kommende Generationen.

\section{Wegfall der maximalen Förderungsdauer}

Die Berechnung der maximalen Förderungsdauer über die sogenannte Regelstudienzeit verfehlt die Realität von vielen Studierenden, da nur ein Bruchteil das Studium in dieser Zeit abschließt. Dies liegt daran, dass die Regelstudienzeit sich auf die minimale Semesteranzahl bezieht, in der Studierende einen Studiengang absolvieren können sollten. Sie ist eine Messgröße für die Studierbarkeit und als solche eine Anforderung an die Hochschulen, die keine Relevanz für eine BAföG-Forderung haben sollte.

Die tatsächliche Studiendauer hängt von vielen individuellen Faktoren ab, die sich nicht in einem einfachen Parameter festhalten lassen können. Auf viele dieser Faktoren können die Studierenden keinen Einfluss nehmen. Wir fordern daher die Abschaffung einer maximalen Förderungsdauer. Das Kriterium für eine Förderung sollte alleinig die Immatrikulation sein.

\section{Elternunabhängiges BAföG für alle}

Entsprechend des eingangs genannten Zitats des Bundestages\footnote{\url{https://www.bundesrat.de/SharedDocs/drucksachen/2019/0201-0300/zu216-19.pdf?__blob=publicationFile&v=1}} zum BAföG darf die soziale Herkunft keinen Einfluss auf die Bildungschancen haben. In vielen Fällen verwehren die Vergabekriterien eine Förderung, obwohl Studierende auf BAföG angewiesen wären. Um diesen Missstand zu beheben, sollte das BAföG für alle elternunabhängig werden.

Dies würde die Studierendenwerke entlasten, da Einzelfälle nicht aufwendig berechnet werden müssen und die Nachbearbeitung durch Widerspruchsfälle entfällt. 
Studierenden mit nur einem Elternteil oder Eltern, die das Studium nicht finanzieren wollen, bringt elternunabhängiges BAföG eine Erleichterung, da sie keinen aufwendigen Rechtsweg einschlagen müssen. Diese sind meist zeitintensiv und psychisch belastend, wodurch unter anderem die Qualität des Studiums leidet.

Die Maßnahme würde auch Studierenden der Mittelschicht zu Gute kommen, bei denen das Einkommen der Eltern häufig nur zu einer geringen oder gar keinen Förderung führt. Allerdings sind die Eltern trotz der theoretischen Berechnungen durch das BAföG praktisch nicht in der Lage, ihre Kinder ausreichend zu unterstützen, da beispielsweise negatives Einkommen nicht in die Berechnung eingeht.

\section{Realistische Wohngeldpauschale}

Eine Wohngeldpauschale, welche nur knapp über den durchschnittlich gezahlten Mieten liegt (324\euro\footnote{\url{https://www.bafög.de/de/-13-bedarf-fuer-studierende-230.php}} vs. 323\euro\footnote{\url{http://www.sozialerhebung.de/download/21/Soz21_hauptbericht.pdf}}) ist für eine große Zahl von Studierenden nicht ausreichend und somit realitätsfern. Die Aussage von Bildungsministerin Karliczek, \glqq man [müsse] ja nicht in die teuersten Städte gehen\grqq{}\footnote{\url{https://www.spiegel.de/lebenundlernen/uni/anja-karliczek-bilanz-einer-unsichtbaren-bildungsministerin-a-1242275.html}}, unterstreicht, dass in diesem Zusammenhang eine Selektierung aufgrund der sozialen Herkunft geschieht. Diese Selektierung widerspricht dem Grundgedanken des BAföG, weshalb wir eine Anpassung der Wohngeldpauschale fordern, welche die Lebensrealität von Studierenden widerspiegelt.

\section{Bundesweit einheitliche Online-Beantragung}

Das bestehende Vergabeverfahren über die regionalen Ämter für Ausbildungförderung sorgt für große (Qualitäts-)Unterschiede bei der Beratung und der Bearbeitungsdauer. Auch gibt es keinen einheitlichen Umgang mit den Kriterien für eine mögliche BAföG-Verlängerung, wie zum Beispiel das Berücksichtigen von Ehrenamtsbescheinigungen. Dies schadet auch der Transparenz der Antragsprüfung und wirkt für Studieninteressierte wie Studierende gleichermaßen abschreckend.

Um diese Probleme zu beheben und eine bundesweite Gleichberechtigung bei der Förderung sicher zu stellen, muss das Antragsverfahren vereinheitlicht, vereinfacht und digitalisiert werden. Der erste Schritt in diese Richtung wurde durch die Einführung des Online-Antragsassistenten bereits getätigt.

Das Deutsche Studentenwerk formuliert jedoch das Ziel, \glqq einen bundesweit einheitlichen e-Antrag, einen e-Bescheid und eine e-Akte beim BAföG zu haben\grqq{}\footnote{\url{https://www.studentenwerke.de/de/content/digitalisierung-des-baf\%C3\%B6g-neuer-online}}.

Diese seien für eine einheitliche Anwendung der BAföG-Regelungen notwendig. Weiterhin wird eine Reduzierung der Anforderungen im BAföG-Gesetz gefordert, um das Verfahren zu verschlanken. Wir schließen uns diesen Forderungen an.
\newpage
\section{Vorläufiger Antrag}

Im Rahmen der Digitalisierung des Antragsverfahren sollte die Chance ergriffen werden, weitere Vereinfachungen für potentielle Studierende vorzunehmen. Darum fordern wir die Möglichkeit, einen vorläufigen BAföG-Antrag ohne Bindung an einen Studienort zu stellen. 

Dieser Antrag soll die grundsätzliche Förderfähigkeit feststellen und kann von einer zentralen Stelle bereits während der Suche nach einem Studienplatz bearbeitet werden. Dies sorgt für schnellere Klarheit bei der Finanzierungssituation und die Vermeidung von Finanzierungslücken am Beginn des Studium.

\section{Abschaffung des Leistungsnachweises}

Die Forderung eines Leistungsnachweises in der Mitte des Studiums sollte abgeschafft werden, da dies zu unnötigem Aufwand für Studierende, Hochschulen und Ämtern führt. Der weitere Studienerfolg ist anhand eines solchen Leistungsnachweises nur schwer abschätzbar. Außerdem belastet das Ende einer Förderung bei nur minimal längerer Studiendauer die Studierenden zusätzlich.

\section{Streichung der Altershöchstgrenze}

Die Weiterbildung und Weiterentwicklung von Menschen ist ein stetiger Prozess, welcher sehr wichtig und unterstützenswert ist. Dieser sollte ohne bürokratische Hürden gefördert werden. 

Beim Einschlagen eines zweiten oder dritten Bildungsweges sollten Menschen keine finanzielle Engpässe oder Existenzängste befürchten müssen. Daher fordern wir eine Abschaffung der Altershöchstgrenzen für die BAföG-Förderung. 

\newpage
\section{Honorierung von Ehrenämtern}

Ehrenamtliche Tätigkeiten sind eine Bereicherung der Gesellschaft und der Persönlichkeit. Sie verdienen neben einer sozialen Anerkennung auch die Honorierung durch das BAföG. Ein zeitlicher Druck bei der Beendigung des Studiums führt dazu, dass sich Studierende weniger ehrenamtlich engagieren. 

Solange eine generelle Beschränkung der Förderungsdauer besteht, fordern wir deswegen dass eine nachgewiesene ehrenamtliche Tätigkeit von mindestens einem Jahr zu einer Verlängerung der Förderungsdauer von bis zu einem Studienjahr führt (vergleiche hierzu auch §32(6), Gesetz über die Hochschulen in Baden-Württemberg (Landeshochschulgesetz - LHG), Vom 1. Januar 2005).

\vspace{1cm}

\vfill
\begin{flushright}
Verabschiedet am 15.11.2020 \\
auf der Online-ZaPF hosted in Garching.
\end{flushright}

\end{document}