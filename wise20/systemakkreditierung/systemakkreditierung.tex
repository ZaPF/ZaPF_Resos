\documentclass[DIV=calc]{scrartcl}
\usepackage[utf8]{inputenc}
\usepackage[T1]{fontenc}
\usepackage[ngerman]{babel}
\usepackage{graphicx}
\usepackage[draft, markup=underlined]{changes}
\usepackage{csquotes}

\usepackage{ulem}
%\usepackage[dvipsnames]{xcolor}
\usepackage{paralist}
\usepackage{fixltx2e}
%\usepackage{ellipsis}
\usepackage[tracking=true]{microtype}
\usepackage[a4paper,top=3cm,bottom=2cm,left=3cm,right=3cm]{geometry}

\usepackage{lmodern}              % Ersatz fuer Computer Modern-Schriften
%\usepackage{hfoldsty}

%\usepackage{fourier}             % Schriftart
\usepackage[scaled=0.81]{helvet}     % Schriftart

\usepackage{url}
\usepackage{tocloft}             % Paket für Table of Contents

%\usepackage{xcolor}
\usepackage{ulem,xspace,xcolor}
\definecolor{urlred}{HTML}{660000}

\usepackage{hyperref}
\hypersetup{
    colorlinks=true,
    linkcolor=black,    % Farbe der internen Links (u.a. Table of Contents)
    urlcolor=black,    % Farbe der url-links
    citecolor=black} % Farbe der Literaturverzeichnis-Links

\usepackage{mdwlist}     % Änderung der Zeilenabstände bei itemize und enumerate
\usepackage{draftwatermark} % Wasserzeichen ``Entwurf''
\SetWatermarkText{}

\parindent 0pt                 % Absatzeinrücken verhindern
\parskip 12pt                 % Absätze durch Lücke trennen

\setlength{\textheight}{23cm}
\usepackage{fancyhdr}
\pagestyle{fancy}
\fancyhead{} % clear all header fields
\cfoot{}
\lfoot{Zusammenkunft aller Physik-Fachschaften}
\rfoot{www.zapfev.de\\stapf@zapf.in}
\renewcommand{\headrulewidth}{0pt}
\renewcommand{\footrulewidth}{0.1pt}
\newcommand{\gen}{*innen}
\addto{\captionsngerman}{\renewcommand{\refname}{Quellen}}

%%%% Mit-TeXen Kommandoset
\usepackage[normalem]{ulem}
\usepackage{xcolor}

\usepackage{enumitem}
\setenumerate[1]{label=\thesection.\arabic*.}

\def\zapf{ZaPF}

\newif\ifcomments
\commentsfalse
\commentstrue

\begin{document}
  \hspace{0.87\textwidth}
  \begin{minipage}{120pt}
  	\vspace{-1.8cm}
  	\includegraphics[width=80pt]{../logo.pdf}
  	\centering
  	\small Zusammenkunft aller Physik-Fachschaften
  \end{minipage}
  \begin{center}
    \huge{Unterstützung des Positionspapiers: Qualitätsberichte systemakkreditierter Hochschulen}\vspace{.25\baselineskip}\\
  	\normalsize
  \end{center}
  \vspace{1cm}
    %
    % \begin{center}
    %     \normalsize
    % \end{center}
    % \vspace{1cm}

%%%% Text des Antrags %%%%

Die ZaPF als Pooltragende Organisation beschließt das Positionspapier des studentischen Akkreditierungspools
\footnote{\url{https://www.studentischer-pool.de}} und trägt es somit mit. \\

\section*{Einleitung}
Sowohl in den European Standards and Guidelines (ESG) als auch in der Musterrechtsverordnung gemäß Artikel 4 (Abs. 1-4) Studienakkreditierungsstaatsvertrag (MRVO) ist festgelegt, 
dass Akkreditierungsberichte inkl. Akkreditierungsentscheidungen veröffentlicht werden müssen. 
Dies bezieht sich ausdrücklich auch auf die internen Verfahren von systemakkreditierten Hochschulen, die in diesem Punkt der Transparenz nicht hinter der Programmakkreditierung zurückfallen dürfen. 
Der Akkreditierungsrat hat in seinem Beschluss vom 17.09.2019 weitere Hinweise erarbeitet und verfügbar gemacht, wie systemakkreditierte Hochschulen ihre sogenannten  Qualitätsberichte zu veröffentlichen haben und definiert Ansprüche an jene Qualitätsberichte. 
Laut diesem Beschluss ist es spätestens ab dem 30.09.2020 nur noch in Verbindung mit einem Qualitätsbericht möglich, den eigenen Studiengang in die Akkreditierungsdatenbank einzutragen.
Die Teilnehmenden des 47. Poolvernetzungstreffens des studentischen Akkreditierungspools sprechen sich dafür aus, dass von systemakkreditierten Hochschulen diese Qualitätsberichte nun eingefordert werden und begrüßen den Beschluss des Akkreditierungsrates vom 17.09.2019. 
Da es seitens einzelner Hochschulen den Wunsch gibt, die Anforderungen an Qualitätsberichte zu verändern und der inhaltliche Mehrwert dieser Berichte stark angezweifelt wurde, soll dieses Papier die studentische Position darstellen. 

\section*{Mehrwert von Qualitätsberichten}
Es ist eine fundamentale Frage der Transparenz sowie Vergleichbarkeit zwischen verschiedenen Hochschulen, dass auch systemakkreditierte Hochschulen ihre Akkreditierungsberichte und -entscheidungen in nachvollziehbarer, umfassender und zugänglicher Form veröffentlichen. 
Die aktuelle Situation, dass einzelne systemakkreditierte Hochschulen ihre Akkreditierungsberichte der Öffentlichkeit vollständig vorenthalten, ist intransparent und inakzeptabel. 
Aktuell werden Akkreditierungsberichte von programmakkreditierten Studiengängen öffentlich zugänglich gemacht; hierin wird transparent mit Mängeln, Verbesserungspotenzialen und Maßnahmen umgegangen. 
Einzelne systemakkreditierte Hochschulen haben dadurch einen vermeintlichen Vorteil, weil sie eigene Verbesserungspotentiale von Studiengängen nicht veröffentlichen. 
Studieninteressierte, Studierende, ArbeitgeberInnen und auch die Öffentlichkeit haben aus unserer Sicht jedoch ein Anrecht darauf, dass auch systemakkreditierte Hochschulen ihrer Veröffentlichungspflicht nachkommen. 
Dabei sind Mindestkriterien für Qualitätsberichte entscheidend, damit diese eine vergleichbare Aussagekraft haben und es damit Vergleichbarkeit zwischen den intern akkreditierten und programmakkreditierten Studiengängen geben kann. 
Dazu gehört unseres Erachtens auch, dass GutachterInnen im Qualitätsbericht benannt werden, das abschließende Akkreditierungsergebnis einsehbar ist und etwaige Sondervoten ausgewiesen werden. 
Dies widerspricht nicht der Heterogenität unserer Hochschullandschaft, führt aber zu klaren, fairen und gleichen Regeln für alle Hochschulen. 
Freiheitsgrade innerhalb der Qualitätsberichte können als Chance genutzt werden, um die eigenen Maßnahmen und Follow-Ups des Studiengangs darzustellen und somit die eigene Qualitätsentwicklung in ihrer Wichtigkeit zu unterstreichen. 
Qualitätsberichte können so als Instrument der Sichtbarmachung eigener Bemühungen um Qualitätsverbesserungen dienen und Studieninteressierten aufzeigen, dass es neben Werbematerialien auf Hochglanzpapier 
auch einen Prozess der stetigen Weiterentwicklung des Studiengangs gibt und zeigt Möglichkeiten auf, sich selbst zu beteiligen. 
Insbesondere Studierende, die bereits Studienerfahrung gesammelt haben, beispielsweise indem sie bereits einen Bachelorabschluss an einer anderen Hochschule erworben haben, 
suchen gezielt nach bestimmten Informationen. 
Die Qualitätsberichte lassen sich als Basis für verschiedene Zwecke und diverse Adressaten verwenden. 
Sie möchten selbst nachlesen können, wie bspw. die Studierbarkeit, Studienorganisation oder Vereinbarkeit mit Familienaufgaben in einem Studiengang von unabhängigen Expert*innen geprüft und bewertet wurde. 
Zudem können die Qualitätsberichte einen Überblick über gute Praktiken innerhalb der verschiedenen systemakkreditierten Hochschulen 
und der Vielfalt der Qualitätssicherungssysteme geben und können als Grundlage für eine systematische Analyse der Entwicklungen der internen Verfahren dienen (vgl. ESG 3.4 Thematic analysis Standard: Agencies should regularly publish reports that describe and analyse the general findings of their external quality assurance activities). 
Vor diesem Hintergrund erwarten wir von systemakkreditierten Hochschulen mehr Selbstvertrauen in die eigenen Prozesse und einem transparenten Umgang mit eigenen Verbesserungspotentialen und entsprechenden Maßnahmen. 
Es ist aus unserer Sicht eine Chance auf eine positive Außendarstellung, wenn systemakkreditierte Hochschulen entsprechende Qualitätsberichte veröffentlichen. 

\section*{Fazit}
Akkreditierungsentscheidungen müssen innerhalb aller Systeme bereits jetzt aussagekräftig dokumentiert werden. 
Die Dokumentation trägt zur kontinuierlichen Qualitätssicherung und -weiterentwicklung bei. 
Die Berichte sind gemäß dem Beschluss des Akkreditierungsrats vom 17.09.2019 für alle Stakeholder zugänglich zu veröffentlichen. 
Zusammenfassend ist die Veröffentlichungspflicht von Qualitätsberichten und die konsequente Umsetzung der bereits beschlossenen Mindestkriterien unumgänglich, 

\begin{itemize}
  \item weil bereits jetzt eine verbindliche Rechtsgrundlage für die Dokumentation besteht, 
  \item weil Qualitätsberichte ein unverzichtbarer Teil der Legitimation von Akkreditierungsentscheidungen sind.
  
  \item weil Transparenz die Basis eines jeden guten und funktionierenden QM-Systems ist, 
  \item und weil sie Vergleichbarkeit zwischen Hochschulen, Studiengängen und QM-Systemen ermöglicht.
\end{itemize}

\vfill
    \begin{flushright}
      Verabschiedet am 15.11.2020 \\auf der Online-ZaPF hosted in Garching.
    \end{flushright}
\end{document}
