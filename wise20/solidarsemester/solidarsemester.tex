\documentclass[a4paper]{scrartcl}
\usepackage[utf8]{inputenc}
\usepackage[T1]{fontenc}
\usepackage[ngerman]{babel}
\usepackage{graphicx}
\usepackage{csquotes}

\usepackage{ulem}
%\usepackage[dvipsnames]{xcolor}
\usepackage{paralist}
\usepackage{fixltx2e}
%\usepackage{ellipsis}
\usepackage[tracking=true]{microtype}

\usepackage{lmodern}              % Ersatz fuer Computer Modern-Schriften
%\usepackage{hfoldsty}

\usepackage[scaled=0.81]{helvet}     % Schriftart

\usepackage{url}
\usepackage{tocloft}             % Paket für Table of Contents

\usepackage{xcolor}
\definecolor{urlred}{HTML}{660000}

\usepackage{hyperref}
\hypersetup{
    colorlinks=true,
    linkcolor=black,    % Farbe der internen Links (u.a. Table of Contents)
    urlcolor=black,    % Farbe der url-links
    citecolor=black} % Farbe der Literaturverzeichnis-Links

\usepackage{mdwlist}     % Änderung der Zeilenabstände bei itemize und enumerate

\parindent 0pt                 % Absatzeinrücken verhindern
\parskip 12pt                 % Absätze durch Lücke trennen

\setlength{\textheight}{23cm}
\usepackage{fancyhdr}
\pagestyle{fancy}
\fancyhead{} % clear all header fields
\cfoot{}
\lfoot{Zusammenkunft aller Physik-Fachschaften}
\rfoot{www.zapfev.de\\stapf@zapf.in}
\renewcommand{\headrulewidth}{0pt}
\renewcommand{\footrulewidth}{0.1pt}
\newcommand{\gen}{*innen}
\addto{\captionsngerman}{\renewcommand{\refname}{Quellen}}


\begin{document}
    \hspace{0.87\textwidth}
    \begin{minipage}{120pt}
        \vspace{-1.8cm}
        \includegraphics[width=80pt]{logo.png}
        \centering
        \small Zusammenkunft aller Physik-Fachschaften
    \end{minipage}
    \begin{center}
        \huge{Positionspapier zum Solidarsemester-Bündnis}\vspace{.25\baselineskip}\\
        \normalsize
    \end{center}
    \vspace{1cm}

Die ZaPF bekräftigt ihre Unterstützung und aktive Teilnahme am Bündnis Solidarsemester und seine Bestrebungen, die vielfältigen negativen Folgen der Corona-Pandemie für Studierende durch solidarische Maßnahmen des Bundes und der Hochschulen zu mildern. Insbesondere trägt sie die zentralen Forderungen des Bündnis im Wintersemester 2020 mit:
\begin{itemize}
\item Eine sofortige Öffnung des BAföGs und/oder eine Anpassung der Überbrückungshilfe an die Bedürfnisse der Studierenden
\item Die Verlängerung aller Prüfungs-, Studien- und Studienfinanzierungsfristen (unter anderem Regelungen zu Freiversuchen, Regel- und Maximalstudienzeit) um mindestens drei Semester.
\item Eine sofortige deutliche Aufstockung des Lehrpersonals und die Entfristung von bestehendem Personal zur Verbesserung der Lehrbedingungen.
\item Die Bereitstellung von öffentlicher, datensicherer Open Source Software, der Verleih von angemessener Hardware und ein gesicherter Zugang zu Lernarbeitsplätzen für alle, die diese benötigen.
\item Die Aussetzung des Finanzierungsnachweises für internationale Studierende und die generelle Zulassung von Selbstständigkeit und Freelancing
\item Internationalen Student:innen, die nicht einreisen können, kurzfristig Immatrikulation und Studienteilnahme online ermöglichen.
\end{itemize}
Die ZaPF wird sich vertreten durch den StAPF weiterhin aktiv an der Arbeit im Bündnis beteiligen und so herausarbeiten, wie sie die Umsetzung dieser Forderungen am besten unterstützen kann.

\vspace*{\fill}
\begin{flushright}
Verabschiedet am 15.11.2020\\
auf der Online-ZaPF hosted in Garching
\end{flushright}

\end{document}