\documentclass[a4paper]{scrartcl}
\usepackage[utf8]{inputenc}
\usepackage[T1]{fontenc}
\usepackage[ngerman]{babel}
\usepackage{graphicx}
\usepackage{csquotes}

\usepackage{ulem}
%\usepackage[dvipsnames]{xcolor}
\usepackage{paralist}
\usepackage{fixltx2e}
%\usepackage{ellipsis}
\usepackage[tracking=true]{microtype}

\usepackage{lmodern}              % Ersatz fuer Computer Modern-Schriften
%\usepackage{hfoldsty}

\usepackage[scaled=0.81]{helvet}     % Schriftart

\usepackage[hyphens]{url}
\usepackage{tocloft}             % Paket für Table of Contents

\usepackage{xcolor}
\definecolor{urlred}{HTML}{660000}

\usepackage{hyperref}
\hypersetup{
    colorlinks=true,
    linkcolor=black,    % Farbe der internen Links (u.a. Table of Contents)
    urlcolor=black,    % Farbe der url-links
    citecolor=black} % Farbe der Literaturverzeichnis-Links

\usepackage{mdwlist}     % Änderung der Zeilenabstände bei itemize und enumerate


\parindent 0pt                 % Absatzeinrücken verhindern
\parskip 12pt                 % Absätze durch Lücke trennen

\setlength{\textheight}{23cm}
\usepackage{fancyhdr}
\pagestyle{fancy}
\fancyhead{} % clear all header fields
\cfoot{}
\lfoot{Zusammenkunft aller Physik-Fachschaften}
\rfoot{www.zapfev.de\\stapf@zapf.in}
\renewcommand{\headrulewidth}{0pt}
\renewcommand{\footrulewidth}{0.1pt}
\newcommand{\gen}{*innen}
\addto{\captionsngerman}{\renewcommand{\refname}{Quellen}}


\begin{document}
    \hspace{0.87\textwidth}
    \begin{minipage}{120pt}
        \vspace{-1.8cm}
        \includegraphics[width=80pt]{logo.png}
        \centering
        \small Zusammenkunft aller Physik-Fachschaften
    \end{minipage}
    \begin{center}
        \huge{Resolution zur Novellierung des Bayerischen Hochschulgesetzes}\vspace{.25\baselineskip}\\
        \normalsize
    \end{center}
    \vspace{1cm}

Im Rahmen der geplanten Hochschulrechtsreform in Bayern äußert sich die ZaPF zu den Eckpunkten der Bayerischen Staatsregierung. Wir beziehen uns im Folgenden auf den Stand des Eckpunktepapiers vom 20. Oktober 2020.\footnote{\url{https://stmwk.bayern.de/download/20668_MRV-Novellierung-des-Bayerischen-Hochschulrechts-Eckpunkte-Hochschulrechtsreform_final_20102020.pdf}}

\section{Einleitung}

Die ZaPF lehnt die von der Bayerischen Staatsregierung vorgelegten Eckpunkte zur geplanten Hochschulrechtsreform in Bayern entschieden ab.

In den Eckpunkten spiegelt sich der dringende gesellschaftliche Bedarf wider, dass Hochschulen sich den epochalen Herausforderungen stellen und aktiv in gesellschaftlichen Prozessen mitwirken, was konkret an den Themen \glqq Nachhaltigkeit\grqq{} und \glqq Gleichberechtigung\grqq{} sowie dem großen \glqq soziale[n], technologische[n], ökonomische[n], ökologische[n] und kreative[n]\grqq{} Wert der Hochschulen festgemacht wird.

Was die Bayerische Landesregierung plant, ist allerdings vor allem ein einseitiges gesellschaftliches Mitwirken der Hochschulen im Dienst von Arbeitgeber:inneninteressen. Dies soll durch einen starken Fokus auf vermehrte Gründungsaktivität und weitgehende Entdemokratisierung zugunsten von Top-Down-Management realisiert werden.

Da Probleme wie Klimakrise und gesellschaftliche Spaltung allerdings offensichtlich Lösungen im Sinne der gesamten Gesellschaft erfordern, müssen diese auch durch Einbeziehung und Mitbestimmung möglichst vieler Interessensgruppen erarbeitet werden. Insbesondere die geplante Entdemokratisierung der Hochschulen ist dafür jedoch im besten Fall kontraproduktiv.

Die gleichzeitig geplanten Studiengebühren für ausländische Studierende halten Studierende aus dem nicht-europäischen Ausland von einem Studium in Deutschland ab, bedienen rassistische Ressentiments und stehen der Lösung globaler Probleme im Weg.

\section{Organisation und Struktur der Hochschulen}

Für die gesellschaftliche Verantwortung der Hochschulen ist es aus Sicht der ZaPF unerlässlich, die freiheitliche demokratische Grundordnung unserer Gesellschaft im Leben der Hochschulen und damit insbesondere in deren innerer Organisation abzubilden.

Aus diesem Grund ist die Gruppenhochschule die gegenwärtig in allen Bundesländern gesetzlich vorgesehene Organisationsform. Durch die Beteiligung von Statusgruppen an demokratischen Prozessen wird sichergestellt, dass alle Hochschulangehörigen Einfluss auf die Gestaltung des Hochschullebens haben.

In der ersatzlosen Aufhebung des Abschnitts II \glqq Aufbau und Organisation der Hochschulen\grqq{} (Artikel 19 bis 41) im BayHSchG zugunsten eines weitestgehend auf Eigenverantwortung basierenden Aufbaus sehen wir eine klare Bedrohung für den demokratischen Charakter der Hochschule und damit die Grundlage des modernen Hochschullebens. So ist in der geplanten Novelle die Mitbestimmung von Studierenden, wissenschaftlichen Mitarbeiter:innen oder überhaupt von Statusgruppen in keinster Weise sichergestellt. Zwar sieht das Eckpunktepapier die Gewährleistung eines \glqq angemessen[en] Einfluss[es] der  Träger[:innen] der Wissenschaftsfreiheit\grqq{} nach Maßgabe der ständigen Rechtsprechung des Bundesverfassungsgerichtes vor, legt jedoch nicht dar, in welcher Weise dieses Ziel mit den vorliegenden Eckpunkten in Einklang zu bringen ist.

Weiterhin kritisieren wir scharf, dass das Eckpunktepapier keine Wahlen für Führungspositionen oder Gremien in den Hochschulen vorsieht. Wir fordern daher, Gremienstrukturen unter gerechter Beteiligung aller Statusgruppen und Fachbereiche weiterhin gesetzlich vorzuschreiben und die Demokratisierung der Hochschulen noch auszubauen. Die Aufsicht der Hochschulleitung muss ebenfalls durch demokratisch gewählte Gremien erfolgen - die Möglichkeit der Abberufung von Hochschulpräsident:innen durch den:die Wissenschaftsminister:in halten wir für undemokratisch und lehnen sie ab.

Wir fordern daher die demokratisch zusammengesetzten Gremien der Hochschulen, insbesondere die akademischen Senate, auf, entschieden Stellung gegen die geplante Hochschulrechtsreform zu beziehen.

Die geplante \glqq Modernisierung\grqq{} des Berufungsrechts schlägt dieselbe undemokratische Richtung ein, weswegen wir sie strikt ablehnen. Hier sollen mit sogenannten  \glqq Findungsverfahren\grqq{} eine Alternative zu herkömmlichen Berufungsverfahren geschaffen werden, bei denen sogar an \glqq geeignete Personen [...] mit konkreten Berufungsangebot[en]\grqq{} herangetreten werden kann. Dies ermöglicht, intransparente Berufungsverfahren ohne Beachtung der Interessen aller Statusgruppen durchzuführen, da die Zusammensetzung der Findungskommissionen nicht festgelegt ist. Zudem würde der Wettbewerbscharakter einer Berufung zerstört, was einem meritokratischen Entscheidungsverfahren abträglich ist.

Auf der Ebene der Studierendenvertretung begrüßt die ZaPF, dass die bayerische Staatsregierung den Mehrwert von landesweiter Vernetzung und Vertretung anerkennt und einen Landesstudierendenbeirat gesetzlich verankern möchte. Leider wird diesem Gremium durch die Verweigerung einer eigenen Rechtspersönlichkeit die Arbeit unnötig erschwert, wodurch sich zeigt, dass das Einbringen von studentischen Interessen, ähnlich wie bei anderen Statusgruppen, durch die Staatsregierung offensichtlich nicht gewollt ist.

Wir beharren deswegen auf unserer Forderung, dass Studierende sich auch in Bayern in Verfassten Studierendenschaften organisieren können müssen. Dazu gehört für uns eine eigene Rechtspersönlichkeit, die Möglichkeit, Beiträge von ihren Mitgliedern zu erheben sowie deren politische Bildung und staatsbürgerliches Verantwortungsbewusstsein zu fördern. Wir sind der Überzeugung, dass eine eigenständig handelnde Studierendenvertretung nicht nur eine demokratische Notwendigkeit ist, sondern auch eine Bereicherung der vielfältigen Hochschulgesellschaft bewirken würde.

Nicht alle vorgeschlagenen Änderungen am hochschulpolitischen Entscheidungsprozess sind negativ zu betrachten. Der Verzicht des Wissenschaftsministeriums auf das zwingende Einvernehmen bei der Einrichtung von Studiengängen und die Übertragung des Berufungsrechts an die Hochschulen beschleunigt bürokratische Prozesse und baut überflüssige Hürden bei der Gestaltung der wissenschaftlichen Arbeit an den Hochschulen ab. Verbunden mit dem grundsätzlichen Erhalt der Gremienstrukturen hätte dies sogar eine Demokratisierung der Prozesse zur Folge, da nun alle Statusgruppen an jedem Schritt beteiligt wären.

\section{Wissenschaft im Ideal der zweckfreien Erkenntnis}

Die ZaPF begrüßt die Akzeptanz des Ideals der zweckfreien Erkenntnis als Paradigma für Forschung und Lehre.
Es muss die Freiheit der Hochschulangehörigen bestehen, durch wissenschaftliche Arbeit in Foschung und Lehre auch Wissen zu gewinnen und zu vermitteln, welches keinen ökonomischen Mehrwert in den Augen der beteiligten Akteure und insbesondere der Hochschulleitungen darstellt.
Gleichzeitig ist anzuerkennen, dass für Hochschulen auch die Zusammenarbeit mit der Wirtschaft notwendig ist, um anwendungsbezogene Forschung zu einem gesellschaftlichen Nutzen zu führen. Allerdings darf hierbei das obige Ideal nicht zugunsten von wirtschaftlichen Interessen oder Exzellenzbestrebungen vernachlässigt werden. Die Gefahr, dass genau dies in Zukunft eintreten wird, sehen wir durch die Eckpunkte leider jedoch akut gegeben. In der Gesamtheit führen die ökonomischen Aspekte der geplanten Hochschulrechtsreform zu einer Ausrichtung der Hochschulen an den Interessen von Arbeitgeber:innen, wobei die Betroffenen in den Hochschulen selbst überwiegend Arbeitnehmer:innen sind. Dies lehnen wir ab und fordern, die wirtschaftlichen Aktivitäten der Hochschulen an den Interessen der Gesamtheit ihrer Mitglieder auszurichten.

Indikativ für die geplanten Maßnahmen ist die Schaffung eines Gründungsfreisemesters, welches gleichrangig, wenn nicht durch seine verlängerte Dauer sogar höherwertig, zum herkömmlichen Forschungsfreisemester stehen soll. Dies verurteilen wir, da es Hochschullehrer:innen motiviert, vermehrt wirtschaftlichen statt wissenschaftlichen Aktivitäten nachzugehen und gleichzeitig Potential für eine Ungleichbehandlung von weniger anwendungs- bzw. industrienahen Forschungsgebieten birgt.

\newpage Eine ähnliche Problematik sehen wir bei der Einführung eines Gesamtlehrdeputats, welches es ermöglicht, die Lehrbelastung ungleich auf Hochschullehrer:innen zu verteilen. Hierbei würde eine Verschiebung der Lehre weg von forschungs- oder wirtschaftsstarken Lehrstühlen in verstärkter Weise stattfinden, obgleich diese erfahrungsgemäß schon geschieht. Dies sorgt zusätzlich dafür, dass wissenschaftlich relevante und innovative Ideen keinen oder nur eingeschränkten Eingang in die Lehre finden, was den oben erwähnten Idealen zuwider läuft.

Diesen Effekt verstärken auch die geplanten Forschungsprofessuren. Es stellt sich die Frage, ob es sich hierbei noch um eine Professur handelt, die ursprünglich Hochschullehrer:innen auszeichnet. Zudem ist an den HaWen ein Ungleichgewicht zwischen den auf Lehre ausgerichteten regulären HaW-Professuren und den geplanten Forschungsprofessuren zu erwarten. Eine Stärkung der Forschung ist viel mehr durch eine Verbesserung der Arbeitsbedingungen des zahlenmäßig deutlich größeren Mittelbaus, sowie durch Mitbestimmung und dem Abbau von persönlichen Abhängigkeitsbeziehungen möglich.

Wir fordern daher die Fixierung eines Lehrdeputats für Hochschullehrer:innen und darüber hinaus, überdurchschnittliche Leistung in der Lehre und freiwillige didaktischer Weiterbildung angemessen zu belohnen. So könnte zum Beispiel eine äquivalente finanzielle Motivation für herausragende Lehre wie für herausragende Forschung geschaffen werden.

Eine weitere problematische Entwicklung für die Qualität der Hochschullehre sehen wir in den Bestrebung hinsichtlich des Berufungsrechts, nämlich in der fixierten Möglichkeit der Gruppenleitung als Qualifikationsweg. Es ist höchst fragwürdig, aus der Leitung einer (Nachwuchs-)Forschungsgruppe grundsätzlich die Befähigung zur eigenständigen Lehre abzuleiten. Vielmehr sollte weiterhin individuell festgestellt werden, ob eine habilitationsäquivalente Leistung vorliegt.

\section{Internationalisierung}

Die ZaPF vertritt die Meinung, dass die Internationalisierung der Hochschulen in erster Linie einen wissenschaftlichen und globalgesellschaftlichen Mehrwert haben sollte. Darüber hinaus bietet sie natürlich ein großes Potential für die Gesellschaft und Wirtschaft des Freistaates Bayern, was im Eckpunktpapier auch anerkannt wird. Da sie aber jeglichen dieser Aspekte zuwiderlaufen, spricht sich die ZaPF gegen die Möglichkeit der Erhebung von Studiengebühren für Nicht-EU-Ausländer:innen und die verpflichtende Verankerung von Deutschkursen in fremdsprachigen Studiengängen aus. Beide Maßnahmen sind klar diskriminierend gegenüber internationalen Studierenden und dürfen auf keinen Fall Einzug in die Hochschulen finden.

Studiengebühren jeglicher Art stellen eine erhebliche Bildungsbarriere dar, die wir im Sinne der Gleichberechtigung entschieden ablehnen. Die Einführung solcher Gebühren erzeugt für bereits in Bayern immatrikulierte Studierende bei schlechter Umsetzung zusätzliche massive Hürden, ihr Studium erfolgreich abzuschließen. Im speziellen Fall der einseitigen Gebühren für Studierende aus Nicht-EU-Staaten verhindern diese internationalen Austausch, der ein Kernelement des wissenschaftlichen Diskurses darstellt. Sie hindern außerdem Menschen aus wirtschaftlich schwachen Ländern am Erreichen eines Hochschulabschlusses, obwohl gute Bildung ein entscheidender Aspekt der Entwicklungshilfe ist.

Es ist darüber hinaus am Beispiel Baden-Württemberg zu beobachten, dass diese Form der Studiengebühren nicht einmal das erhoffte wirtschaftliche Ziel erreicht und deutlich geringere Einnahmen erzielte als im Vorfeld berechnet wurde.\footnote{Quelle: Stellungnahme LAK BW vom 28.10.2018, \url{https://www.kontextwochenzeitung.de/fileadmin/content/kontext_wochenzeitung/dateien/398/20181028_stellungnahme_lak_gebuehren.pdf}}

Die ZaPF unterstützt ausdrücklich, dass internationalen Studierenden die Möglichkeiten gegeben werden, die deutsche Sprache im Rahmen ihres Studiums zu erlernen. Diese Weiterbildung verpflichtend in explizit fremdsprachige Studiengänge zu integrieren halten wir jedoch für den falschen Weg. Dies würde nicht nur eine Ungleichbehandlung und damit einen Chancennachteil für internationale Studierende bedeuten, es steht auch der Attraktivität solcher Studiengänge massiv entgegen. Diese bieten explizit auch die Möglichkeit, sich ohne das Ziel eines dauerhaften Aufenthalts in Deutschland auf höchstem Niveau zu bilden, was wir im Sinne des interkulturellen Austauschs unterstützen. 

Um die sprachliche Weiterbildung attraktiv und fair in fremdsprachige Studiengänge einzubinden, schlagen wir stattdessen vor, einen sprachlichen Wahlpflichtbereich in die entsprechenden Studien- und Prüfungsordnungen zu integrieren. So können internationale Studierende ohne Nachteil im Rahmen ihres Studiums Deutsch lernen oder beispielsweise Kurse zu Wissenschaftsenglisch einbringen. Um die fachliche Spezialisierung nicht zu behindern, sollte dieser Bereich jedoch keine Mindestgröße haben.

\vspace*{\fill}
\begin{flushright}
	Verabschiedet am 15.11.2020\\
	auf der Online-ZaPF hosted in Garching
\end{flushright}

\end{document}