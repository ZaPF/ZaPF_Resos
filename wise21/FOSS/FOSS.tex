\documentclass[DIV=calc]{scrartcl}
\usepackage[utf8]{inputenc}
\usepackage[T1]{fontenc}
\usepackage[ngerman]{babel}
\usepackage{graphicx}
\usepackage[draft, markup=underlined]{changes}
\usepackage{csquotes}
\usepackage{eurosym}

\usepackage{ulem}
%\usepackage[dvipsnames]{xcolor}
\usepackage{paralist}
\usepackage{fixltx2e}
%\usepackage{ellipsis}
\usepackage[tracking=true]{microtype}

\usepackage{lmodern}              % Ersatz fuer Computer Modern-Schriften
%\usepackage{hfoldsty}

%\usepackage{fourier}             % Schriftart
\usepackage[scaled=0.81]{helvet}     % Schriftart

\usepackage{url}
%\usepackage{tocloft}             % Paket für Table of Contents

\usepackage{xcolor}
\definecolor{urlred}{HTML}{660000}

\usepackage{hyperref}
\hypersetup{colorlinks=false}

%\usepackage{mdwlist}     % Änderung der Zeilenabstände bei itemize und enumerate
% \usepackage{draftwatermark} % Wasserzeichen ``Entwurf''
% \SetWatermarkText{Antrag}

\parindent 0pt                 % Absatzeinrücken verhindern
\parskip 12pt                 % Absätze durch Lücke trennen

\setlength{\textheight}{23cm}
\usepackage{fancyhdr}
\pagestyle{fancy}
\fancyhead{} % clear all header fields
\cfoot{}
\lfoot{Zusammenkunft aller Physik-Fachschaften}
\rfoot{www.zapfev.de\\stapf@zapf.in}
\renewcommand{\headrulewidth}{0pt}
\renewcommand{\footrulewidth}{0.1pt}
\newcommand{\gen}{*innen}
\addto{\captionsngerman}{\renewcommand{\refname}{Quellen}}

%%%% Mit-TeXen Kommandoset
\usepackage[normalem]{ulem}
\usepackage{xcolor}
\usepackage{xspace} 

\newcommand{\replace}[2]{
    \sout{\textcolor{blue}{#1}}~\textcolor{blue}{#2}}
\newcommand{\delete}[1]{
    \sout{\textcolor{red}{#1}}}
\newcommand{\add}[1]{
    \textcolor{blue}{#1}}

\newif\ifcomments
\commentsfalse
%\commentstrue

\newcommand{\red}[1]{{\ifcomments\color{red} {#1}\else{#1}\fi}\xspace}
\newcommand{\blue}[1]{{\ifcomments\color{blue} {#1}\else{#1}\fi}\xspace}
\newcommand{\green}[1]{{\ifcomments\color{green} {#1}\else{#1}\fi}\xspace}

\newcommand{\repl}[2]{{\ifcomments{\color{red} \sout{#1}}{\color{blue} {\xspace #2}}\else{#2}\fi}}
%\newcommand{\repl}[2]{{\color{red} \sout{#1}\xspace{\color{blue} {#2}}\else{#2}\fi}\xspace}

\newcommand{\initcomment}[2]{%
	\expandafter\newcommand\csname#1\endcsname{%
		\def\thiscommentname{#1}%
		\definecolor{col}{rgb}{#2}%
		\def\thiscommentcolor{col}%
}}

% initcomment Name RGB-color
\initcomment{Philipp}{0, 0.5, 0}

%\renewcommand{\comment}[1]{{\ifcomments{\color{red} {#1}}{}\fi}\xspace}

\renewcommand{\comment}[2][\nobody]{
	\ifdefined#1
	{\ifcomments{#1 \expandafter\color{\thiscommentcolor}{\thiscommentname: #2}}{}\fi}\xspace
	\else
	{\ifcomments{\color{red} {#2}}{}\fi}\xspace
	\fi
}

\newcommand{\zapf}{ZaPF\xspace}

\let\oldgrqq=\grqq
\def\grqq{\oldgrqq\xspace}

\setlength{\parskip}{.6em}
\setlength{\parindent}{0mm}

%\usepackage{geometry}
%\geometry{left=2.5cm, right=2.5cm, top=2.5cm, bottom=3.5cm}

% \renewcommand{\familydefault}{\sfdefault}




\begin{document}

\hspace{0.87\textwidth}
\begin{minipage}{120pt}
	\vspace{-1.8cm}
	\includegraphics[width=80pt]{../logo.pdf}
	\centering
	\small Zusammenkunft aller Physik-Fachschaften
\end{minipage}

\begin{center}
  \huge{Resolution zur freien Open Source Software}\vspace{.25\baselineskip}\\
  \normalsize
\end{center}
\vspace{1cm}

%%%% Metadaten %%%%

% \paragraph{Addressierte:} Landesregierung NRW, die Landtagsfraktionen von CDU, FDP, Grünen, SPD und der Linken NRW

% \paragraph{Antragstellende:} Bernhard & Sebastian (Uni Köln)

%%%% Text des Antrages zur veröffentlichung %%%%

% \section*{Antragstext}

Die ZaPF unterstützt die Initiative Public Code der Free Software Foundation Europe und befürwortet den offenen Brief\footnote{\url{https://publiccode.eu/de/openletter/}}. 
Der StAPF nimmt die Kommunikation mit den Organisator*innen der Kampagne auf und unterzeichnet im Namen der ZaPF.

Für eine demokratischere Hochschule braucht es auch demokratische und damit freie Open Source Software (FOSS), die Partizipation fördert. Die Freiheit der Forschung und Lehre muss konsequent in Software fortgesetzt werden.\\
Durch FOSS würden Forschende entlastet werden, wissenschaftlichen Institutionen Geld einsparen und doppelte Arbeit vermieden werden. Darüber hinaus kann man sich von Quasi-Monopolen lösen und ist nicht deren Willkür \footnote{\url{https://home.cern/news/news/computing/migrating-open-source-technologies}} ausgesetzt. \\
Nicht überall ist benötigte Software frei verfügbar oder es mangelt an Lizenzen. Der Einsatz von FOSS löst unserer Meinung nach das Problem, indem die Notwendigkeit der Lizenzbeschaffung entfällt; dadurch können soziale und finanzielle Hürden abgebaut werden.\\
Gerade Studierende haben bei FOSS die Möglichkeit Fehler selbst zu beheben und die Software nach ihren Wünschen anzupassen. Es ist ganz im Sinne einer positiven Bildungsphilosophie, sich an der Lösung des Problems zu beteiligen und daraus zu lernen um somit ein tieferes Verständnis erlangen zu können.\\
Wie wir auch schon in einer früheren Resolution \footnote{\url{https://zapf.wiki/images/4/4e/Resolution_SoSe2021_opensource.pdf}} festgehalten haben, fordern wir den vermehrten Einsatz von FOSS im universitären Alltag.
Wir bitten die Fachschaften und das ZKI, die Kampagne zu verbreiten und zu unterstützen.

%\vspace{1cm} 

\vfill
\begin{flushright}
	Verabschiedet am 14. November 2021 \\
	auf der ZaPF in Göttingen.
\end{flushright}

\end{document}
