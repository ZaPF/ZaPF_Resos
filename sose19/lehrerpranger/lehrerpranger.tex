\documentclass[draft,10pt,oneside]{scrartcl}

% Sprache und Encodings
\usepackage[ngerman]{babel}
%\usepackage[T1]{fontenc}
\usepackage[utf8]{inputenc}

% Typographisch interessante Pakete
\usepackage{microtype} % Randkorrektur und andere Anpassungen

% References to Internet and within the document
\usepackage[pdftex,colorlinks=false, pdftitle={Antrag zur Änderung der
Geschäftsordnung für Plenen der ZaPF}, pdfauthor={Jörg Behrmann (FUB), Björn
Guth (RWTH)}, pdfcreator={pdflatex}, pdfdisplaydoctitle=true]{hyperref}

% Absaetze nicht Einruecken
\setlength{\parindent}{0pt} \setlength{\parskip}{2pt}

% Formatierung auf A4 anpassen
\usepackage{geometry}
\geometry{paper=a4paper,left=15mm,right=15mm,top=10mm,bottom=10mm}

% Zeilenumbrüche
\usepackage{hyphenat}
%\hyphenation{Baden-Württemberg}

\begin{document}

\section*{Resolution zur Verurteilung von Onlineplattformen zur Denunziation
von Lehrenden}

\textbf{Antragsteller:} Jörg Behrmann (FUB), Björn Guth (RWTH), Hannah (HUB)

\textbf{Adressaten:} Deutscher Philologenverband, GEW, Fachverband Didaktik der
DPG, AG Schule der DPG

\subsection*{Antrag}

Die ZaPF möge beschließen:

\begin{quote} Die Freiheit von Forschung und Lehre ist der zentrale
    Grundpfeiler wissenschaftlicher Arbeit. Dazu gehört insbesondere die
    Fähigkeit frei von Angst und Repressionen die eigene Lehre zu gestalten.

    Die selbsternannte \glqq{}Alternative für Deutschland\grqq{} (AfD) hat diese
    Freiheit in der jüngeren Vergangenheit in mehreren Bundesländern
    (z.B. Ba\-den-\allowbreak Würt\-tem\-berg, Berlin, Hamburg, Rheinland-Pfalz
    und Sachsen) unter dem Deckmantel des Neutralitätsgebots angegriffen und
    Denunziationsplattformen für Lehrende eingerichtet, die sich der AfD
    gegenüber kritisch äußern. Dies erzeugt ein Milieu der Angst, dem wir uns
    entgegenstellen müssen.

    Die Schule stellt einen in besonderem Maße schützenswerten Raum dar, in dem
    die Werte der freiheitlich-demokratischen Grundordnung aktiv vorgelebt
    werden müssen.  Die Mär von der Absolutheit des Neutralitätsgebots an
    Schulen steht dabei im krassen Widerspruch zum Beutelsbacher
    Konsens\footnote{https://www.bpb.de/die-bpb/51310/beutelsbacher-konsens}.
    Durch diese Praktiken wird die Freiheit der Lehre im Allgemeinen und die
    Schule im speziellen in einem nicht hinnehmbarem Maße angegriffen.

    Aus diesen Gründen verurteilt die ZaPF die Einrichtungen solcher
    Denunziationsplattformen aufs Schärfste und ruft alle Lehrenden zum Protest
    gegen diese Praktiken auf.

    Mögliche Protestformen umfassen:

    \begin{itemize}
        \item Kollektive Selbstanzeigen wie zum Beispiel gesehen an der
            Lina-\allowbreak Morgenstern-\allowbreak Schule in
            Ber\-lin-\allowbreak Kreuz\-berg
        \item Nutzung der Auskunftsrechte nach Artikel 15 der
            Datenschutzgrundverordnung (DSGVO) sowie Geltentmachung des Rechts
            auf Löschung nach Artikel 17 der DSGVO
        \item Beschwerde beim Landesdatenschutzbeauftragten
    \end{itemize}
\end{quote}

\subsection*{Begründung}

Die Begründung ist dem Antrag zu entnehmen.

Wir sehen uns verantwortlich, da die Pranger der AfD uns direkt betreffen, über
einen Pranger für Profs den ein baden-württenbergischer AfD-Abgeordneter
gestartet hat, sowie dadurch, dass Lehramts-ZaPFika in ihrem Praxissemester
direkt betroffen sind.

\end{document}
