\documentclass[draft,10pt,oneside]{scrartcl}

% Sprache und Encodings
\usepackage[ngerman]{babel}
%\usepackage[T1]{fontenc}
\usepackage[utf8]{inputenc}

% Typographisch interessante Pakete
\usepackage{microtype} % Randkorrektur und andere Anpassungen

% References to Internet and within the document
\usepackage[pdftex,colorlinks=false, pdftitle={Antrag zur Änderung der
Geschäftsordnung für Plenen der ZaPF}, pdfauthor={Jörg Behrmann (FUB), Björn
Guth (RWTH)}, pdfcreator={pdflatex}, pdfdisplaydoctitle=true]{hyperref}

% Absaetze nicht Einruecken
\setlength{\parindent}{0pt} \setlength{\parskip}{2pt}

% Formatierung auf A4 anpassen
\usepackage{geometry}
\geometry{paper=a4paper,left=15mm,right=15mm,top=10mm,bottom=10mm}

% Zeilenumbrüche
\usepackage{hyphenat}
%\hyphenation{Baden-Württemberg}

\begin{document}

\section*{Verfasste Studierendenschaften}

\textbf{Antragsteller:} Jörg Behrmann (FUB), Björn Guth (RWTH)

\textbf{Adressaten:} alle Fraktionen aller deutschen Landtage

\subsection*{Antrag}

Die ZaPF möge beschließen:

\begin{quote}
  Aufgrund jüngster Hetze gegen und Forderungen zur Abschaffung von verfassten
  Studierendenschaften\footnote{u.a. Drucksache 7/3844 des Landtages von Sachsen
    Anhalt
    (\url{https://www.landtag.sachsen-anhalt.de/fileadmin/files/drs/wp7/drs/d3844aan.pdf}),
  "Zwangsmitgliedschaft in der Studierendenschaft abschaffen" der Jungen
  Liberalen NRW
  (\url{https://julis-nrw.de/beschlusssammlung/zwangsmitgliedschaft-in-der-studierendenschaft-abschaffen/}),
  Grundsatzprogramm der Jungen Union von 2012}
  wollen wir unseren Standpunkt aus dem Wintersemester 2009/2010 bekräftigen und
  erneuern.

  Die Zusammenkunft aller Physikfachschaften (ZaPF) unterstützt Verfasste
  Studierendenschaften vorbehaltlos und fordert weiterhin die Wiedereinführung
  der verfassten Studierendenschaft in Bayern. Bestrebungen die Rechte
  verfasster Studierendenschaften zu beschneiden oder sie gar abzuschaffen
  lehnen wir ab.

  Zur Vertretung und Wahrung der Rechte und Interessen aller Studierenden sind
  frei gewählte Vertretungen Verfasster Studierendenschaften notwendig. Zur
  Ausübung dieser Funktion sollen sie insbesondere mit folgenden Rechten
  ausgestattet sein:
  \begin{enumerate}
  \item Sich selbst eine Satzung zu geben,
  \item Beiträge zu erheben und ihre Finanzen selbst zu verwalten,
  \item sowie sich politisch zu äußern.
  \end{enumerate}

  Darüber hinaus stellt die verfasste Studierendenschaft eine
  Solidargemeinschaft dar. Eine Möglichkeit sich aus dieser zu lösen, wie sie in
  zwei Bundesländern traurigerweise existiert, verurteilen wir und fordern
  weiterhin eine verpflichtende Mitgliedschaft.

\end{quote}

\subsection*{Begründung}

An unserem Standpunkt hat sich seit 2010 nichts verändert. Stärker geworden sind
nur die Rufe das Rad der Zeit zurückzudrehen und Errungenschaften wieder
abzuschaffen. Sachsen-Anhalt und Sachsen haben Möglichkeiten geschaffen aus der
verfassten Studierendenschaft auszutreten und Bayern hat sie immernoch nicht
wieder eingeführt und selbst Baden-Württemberg möchte die gerade eingeführten
verfassten Studierendenschaften wieder in ihren Rechten beschneiden. Es ist Zeit
unsere Meinung wieder zu unterstreichen.

\end{document}

%%% Local Variables:
%%% mode: latex
%%% TeX-master: t
%%% End:
